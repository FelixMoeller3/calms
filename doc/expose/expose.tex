\documentclass[expose, en]{thesis}

\thesissetup{%
  name={Felix Möller}, %
  id={2287872}, %
  mail={utywh@student.kit.edu}, %
  titlede={/}, %
  titleen={Continual Active Learning for Effective Model Stealing Attacks}, %
  thesis=BSc, %MSc
  registration={January 6th, 2023}, %
  submission={April 28th, 2023}, %
  degree={Informatik}, %
  supervisor={Jun.-Prof. Dr. Christian Wressnegger},
  cosupervisor={Prof. Dr. Thorsten Strufe}
}

\usepackage{lipsum}
\usepackage{multicol}
\usepackage{multirow}
\usepackage{pgfgantt}
\usepackage{pifont}
\usepackage{tabularx}
\usepackage{tikz}
\usepackage{wasysym}
\usepackage{xcolor}


\begin{document}
\maketitle


\section{Introduction}
%Some intro to the topic: What is it about? What is the specific problem
%that should be addressed in this work?
Machine Learning, especially Deep Learning, is a resource-intensive task. Modern models can have millions of parameters
 and require a lot of data to be trained. This makes model development difficult for small companies or individuals.
 Big-tech companies like Google, Amazon and Microsoft however have access to a lot of computing power which they
 now monetize by offering Machine-Learning as a Service (MLaaS). MLaaS-Plattforms allow users to train their models in the cloud
 and query them for predictions. While MLaaS makes Machine Learning more accessible it also poses a security risk. Numerous works
 have shown that it is possible to steal models from MLaaS-Plattforms by consecutively querying the target model, i.e. the model which 
 runs on the cloud and can be queried \cite{tramer2016stealing} \cite{reith2019efficiently}. The queries are then used to train a local model, also called the substitute model which is
 intended to approximate the target model as good as possible.
 The aim of this bachelor thesis is to further improve upon the Model Stealing process. Most recent works in the area of Model Stealing
 have made use of Active Learning to improve the quality of the substitute model and ultimately reduce the number of queries needed to approximate
 the target model sufficiently. However, in each iteration of the Active Learning process, an entirely new model has to be trained. This thesis aims to
 combine Active Learning with Continual Learning to speed up the training process of the substitute model and make it approximate the target model
 better.
\subsection{Motivation}
%What is the motivation for this work? Why should this topic be
%elaborated?
The main motivation for this work is that there are still opportunities to further improve the Model Stealing process.
Namely, Continual Learning has not been brought into the context of Model Stealing yet (to the best of our knowledge).
Continual Learning is different from traditional Machine Learning in that it aims to learn new tasks while retaining knowledge
of old tasks. That way, when the distribution of the data changes, the existing model can be adapted to the new distribution
instead of having to train a new model from scratch as it would be the case in traditional Machine Learning.
Given that Continual Learning has been shown to be effective in other domains, it is reasonable to assume that it can
lead to significant improvements in the Model Stealing process as well.

\subsection{Scope}
%Since you cannot solve all problems, what is the particular scope of
%your research? Which methods/approaches should be applied or
%implemented?
The particular scope of this thesis is to implement and evaluate a continual Active Learning approach for Model Stealing.
More concretely, the Active Learning strategies CoreSet \cite{sener2017active}, LC \cite{lewis1995sequential}, BALD \cite{gal2017deep} 
and BADGE \cite{ash2019deep} should be implemented in combination
with the Continual Learning strategies MAS \cite{aljundi2018memory}, EWC \cite{kirkpatrick2017overcoming}, ALASSO \cite{park2019continual}
and IMM \cite{lee2017overcoming}. These combinations of Active Learning and Continual Learning should 
then be transferred to the Model Stealing process and evaluated on the MNIST,CIFAR-10, CIFAR-100 and (optionally) ImageNet datasets.
The focus of the evaluation should be to test how well Deep Neural Networks for image classification can be extracted using
this continual Active Learning approach.

\section{Related Work}
The related work has to be divided into three parts: The first part is focused on Model Stealing,
the second part is focused on Active Learning and the third part is focused on Continual Learning.

\subsection{Model Stealing}
For the Model Stealing and the Active Learning part, \cite{chandrasekaran2020exploring} draws important
conclusions between Model Stealing and Active Learning. It therefore follows that the Active Learning
domain should be closely followed by researchers in the Model Stealing domain and vice versa because the 
two domains are closely related. \cite{pal2020activethief} provide a more practical approach by developing
a model extraction framework for Deep Neural Networks based on Active Learning. Their framework which is called
ActiveThief is able to extract a model simply by providing an unlabeled thief dataset. The use of Active Learning to
select which points to label next showed to be effective because it significantly reduced the number of queries
needed to extract the model. Given that Model Stealing can have tremendous negative consequences, there has been research on
how to prevent it. One approach which is called Protecting against DNN Model Stealing Attacks (PRADA) \cite{juuti2019prada},
aims to detect Model Stealing attacks by observing the distribution of the queried data and raising an alarm when it deviates
significantly from the distribution that would be expected from a legitimate user. Rather than passively detecting Model Stealing
attacks, another approach named Prediction Poisoning \cite{orekondy2019prediction} actively adds noise to the model's predictions. 


\subsection{Active Learning}
% Talk about CoreSet, LC, BALD and BADGE
For the Active Learning part we will focus on the Active Learning approaches that will be used in the evaluation. These are
CoreSet \cite{sener2017active}, LC \cite{lewis1995sequential}, Bayesian Active Learning by disagreement (BALD) \cite{gal2017deep} 
and Batch Active Learning by Diverse Gradient Embeddings (BADGE) \cite{ash2019deep}. The CoreSet strategy \cite{sener2017active}
proposes that the model should be learned by the subset which will be able to classify the remaining points best. LC Active Learning \cite{lewis1995sequential}
iteratively labels the samples where the current model is most uncertain of the classes they belong to. The third strategy called BALD is a specific Bayesian
Active Learning strategy that makes use of Monte Carlo Dropout \cite{gal2016dropout} to approximate the uncertainty of the model. The final approach,
BADGE computes the gradient embedding of the current model's output layer and uses them to extend the labeled dataset.

\subsection{Continual Learning}
% Talk about MAS, EWC, ALASSO and IMM
As for the Active Learning part, we will focus on Related Work on the Continual Learning approaches that will be used in the evaluation.
These are Memory Aware Synapses (MAS) \cite{aljundi2018memory}, Elastic Weight Consolidation (EWC) \cite{kirkpatrick2017overcoming},
asymmetric loss approximation with single-side overestimation (ALASSO) \cite{park2019continual} and 
Incremental Moment Matching (IMM) \cite{lee2017overcoming}. The first Continual Learning strategy, MAS \cite{aljundi2018memory}, continually computes the importance of
a neural network's parameter via the gradient of the learned function with respect to that parameter. When learning a new task, changes to the important parameters
of the network are penalized, which enables the learning of new tasks without forgetting previously learned ones. The second strategy, EWC \cite{kirkpatrick2017overcoming},
uses a similar approach, however the importance of the parameters is computed by the Fisher Information Matrix. The third approach, ALASSO \cite{park2019continual},
uses a loss function that overestimates the loss in areas where the true loss cannot be observed and behaves like a regular loss function in areas where it can.
The final approach, IMM \cite{lee2017overcoming}, uses Bayesian Neural Networks to match the moments of tasks' posterior distributions.
A more detailed overview, including even more Continual Learning approaches, can be found in \cite{mundt2020wholistic}. They provide a good summary of the existing
Continual Learning approaches and how they link together with Active Learning. 


\section{Evaluation}
For the evaluation the datasets CIFAR-10, CIFAR-100, MNIST and optionally ImageNet will be used. The Active Learning strategies that will be used are
CoreSet \cite{sener2017active}, LC \cite{lewis1995sequential}, BALD \cite{gal2017deep} and BADGE. The Continual Learning strategies that will be used are MAS \cite{aljundi2018memory},
EWC \cite{kirkpatrick2017overcoming}, ALASSO \cite{park2019continual} and IMM \cite{lee2017overcoming}. The baseline of our evaluation will be the ActiveThief
framework proposed in \cite{pal2020activethief}. If there is enough time to do so, then we will also evaluate how our Continual Learning approach performs
when Model Stealing defense mechanisms like \cite{juuti2019prada} and \cite{orekondy2019prediction} are used.

\clearpage
\section{Schedule}

% THIS IS AN EXAMPLE, I.E. ADAPT THE TASKS TO YOUR NEEDS

\ifthenelse{\equal{\mythesistype}{BSc}}
{ % for BSc Thesis
  \begin{figure}[h!]
    \centering
    \begin{ganttchart}[hgrid, vgrid,]{1}{16}
      \gantttitle{Week}{16} \\
      \gantttitlelist{1,...,16}{1} \\
      \ganttbar{\nameref{sec:task1}}{1}{5} \\
      \ganttbar{Implement Model Stealing workflow}{6}{7} \\
      \ganttbar{Run experiments}{8}{10} \\
      \ganttmilestone{Status Presentation}{8} \\
      \ganttbar{Rehearsal + Representation-based AL}{10}{12} \\
      \ganttbar{Write thesis}{13}{16} \\
      \ganttmilestone{Submit first draft}{14} \\
      \ganttbar{Print and Submit}{16}{16} \\
      \ganttmilestone{Final Presentation}{16}
    \end{ganttchart}
    \caption{Schedule of the Bachelor's thesis}
    \label{fig:timeschedule}
  \end{figure}
}
{% for MSc Thesis
  \begin{figure}[htbp]
    \centering
    \begin{ganttchart}[hgrid, vgrid,]{1}{24}
      \gantttitle{Week}{24} \\
      \gantttitlelist{1,...,24}{1} \\
      \ganttbar{\nameref{sec:task1}}{1}{4} \\
      \ganttbar{\nameref{sec:task2}}{5}{9} \\
      \ganttmilestone{Status Presentation}{6} \\
      \ganttbar{\nameref{sec:task3}}{7}{17} \\
      \ganttbar{\nameref{sec:task4}}{18}{23} \\
      \ganttbar{Print and Submission}{24}{24} \\
      \ganttmilestone{Final Presentation}{24}
    \end{ganttchart}
    \caption{Schedule of the Master's thesis}
    \label{fig:timeschedule}
  \end{figure}
}


\subsection{Combine Active Learning with Continual Learning strategy}
\label{sec:milestone1}

The first large milestone is to combine Active Learning with Continual Learning. Currently, the
problem is that Active Learning strategies train a new model from scratch in each iteration. Given that these models
are often Deep Neural Networks, this is a time and resource consuming process. To solve this problem,
Continual Learning strategies should be used to train a single model which can be updated in each iteration
of the Active Learning process. Doing exactly this will be the goal of the first milestone.

\subsubsection{Implement AL + CL strategies} 
\label{sec:task1}
The goal of this task is to implement the Active Learning strategies CoreSet \cite{sener2017active}, LC \cite{lewis1995sequential}, BALD \cite{gal2017deep}
 and BADGE \cite{ash2019deep} as well as the Continual Learning strategies MAS \cite{aljundi2018memory},
 EWC \cite{kirkpatrick2017overcoming}, ALASSO \cite{park2019continual} and IMM \cite{lee2017overcoming}. Since the chair's framework has already implemented
 the four Active Learning strategies, we will only need to implement the Continual Learning strategies. For the
implementation of each method, one needs to check whether the papers which introduced the method provide an implementation. If they
do then it needs to be tested whether it works as intended. If it does not work as intended or no implementation
is provided, then the methods need to be implemented from scratch.
\paragraph{Must-haves:}
\begin{itemize}
    \item Implement MAS Continual Learning strategy
    \item Implement EWC Continual Learning strategy
    \item Implement ALASSO Continual Learning strategy
    \item Implement IMM Continual Learning strategy
\end{itemize}

\paragraph{Nice-to-haves:}
\begin{itemize}
    \item Potentially implement more Active Learning and/or Continual Learning strategies
\end{itemize}


\subsection{Transfer Continual Active Learning to Model Stealing}
Transferring the Continual Active Learning approach to the Model Stealing domain will be the
second milestone. The goal of this step is to speed up the process of Model Stealing. This shall be achieved by updating
the current substitute model in each iteration of the Active Learning process instead of training a new model from scratch.
For this to work we need to integrate the Continual Active Learning strategies into the ActiveThief \cite{pal2020activethief} framework.
After the new Model Stealing workflow is implemented we can run experiments to determine how well our new approach
performs. To run the experiments, we will use the datasets CIFAR-10, CIFAR-100 and MNIST . If there is enough 
time we will also run experiments on ImageNet and we will see how effective Model Stealing defense methods like
 \cite{orekondy2019prediction} and \cite{juuti2019prada} are against our new approach. Another nice-to-have would be to
insert Exemplar Rehearsal \cite{gepperth2016bio} and Representation-based Active Learning \cite{ren2021survey} into the ActiveThief framework
and see how well they perform.

\subsubsection{Implement Model Stealing workflow and run experiments}
\label{sec:task2}

\paragraph{Must-haves:}
\begin{itemize}
    \item Integrate the Continual Active Learning strategies into ActiveThief framework
    \item Well-formatted output for Model Stealing workflow (like in training phase of Tensorflow models)
    \item Run experiments on CIFAR-10 dataset
    \item Run experiments on CIFAR-100 dataset
    \item Run experiments on MNIST dataset
\end{itemize}

\paragraph{Nice-to-haves:}
\begin{itemize}
  \item Run experiments on ImageNet dataset
  \item Run experiments with Model Stealing defense mechanisms (PRADA \cite{juuti2019prada} and Prediction Poisoning \cite{orekondy2019prediction})
  \item Insert Exemplar Rehearsal into Continual Learning framework
  \item Insert Representation-based Active Learning into Active Learning framework
  \item Run experiments with Exemplar Rehearsal Continual Learning and Representation-based Active Learning
\end{itemize}


\newpage
\bibliographystyle{plain}
\bibliography{expose}


\newpage
\section*{Implementation}

This document is the foundation for the implementation of the
\emph{\thesistype} and describes the contents and goals that must be fulfilled.
Please consider the general \emph{\thesistype} requirements of your examination
regulations. If you are in doubt please contact the
\href{https://www.informatik.kit.edu/sul.php}{examination office}.

\subsubsection*{Task Definition and Supervision}
\vspace*{5mm}
\begin{tabularx}{\textwidth}{@{}XX}
    \mysupervisor & \rule[-0.5ex]{\linewidth}
    {1pt}\newline\itshape(Date, Signature)\\[5ex]
\end{tabularx}

\vspace*{10mm}

\subsubsection*{Student}
\vspace*{5mm}
\begin{tabularx}{\textwidth}{@{}XX}
    \myname & \rule[-0.5ex]{\linewidth}{1pt}\newline\itshape(Date, Signature)
\end{tabularx}

\end{document}
