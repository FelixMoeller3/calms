\documentclass[expose, en]{thesis}

\thesissetup{%
  name={Foo Bar}, %
  id={1337}, %
  mail={foo@student.kit.edu}, %
  titlede={Mein Titel}, %
  titleen={My Title}, %
  thesis=BSc, %MSc
  registration={1. February 2020}, %
  submission={31. July}, %
  degree={Informatik}, %
  supervisor={Jun.-Prof. Dr. Christian Wressnegger},
  cosupervisor={Prof. Dr. N.N}
}

\usepackage{lipsum}
\usepackage{multicol}
\usepackage{multirow}
\usepackage{pgfgantt}
\usepackage{pifont}
\usepackage{tabularx}
\usepackage{tikz}
\usepackage{wasysym}
\usepackage{xcolor}


\begin{document}
\maketitle


\section{Introduction}
Some intro to the topic: What is it about? What is the specific problem
that should be addressed in this work?

\subsection{Motivation}
What is the motivation for this work? Why should this topic be
elaborated?

\subsection{Scope}
Since you cannot solve all problems, what is the particular scope of
your research? Which methods/approaches should be applied or
implemented?


\section{Related Work}
Describe the state of the art with \num{5}-\num{10} references from the
field of research and add the corresponding
cites~\cite{website:google}.


\section{Evaluation}
How can the required data be obtained/acquired? How can the results of
the topic be evaluated?

\clearpage
\section{Schedule}

% THIS IS AN EXAMPLE, I.E. ADAPT THE TASKS TO YOUR NEEDS

\ifthenelse{\equal{\mythesistype}{BSc}}
{ % for BSc Thesis
  \begin{figure}[h!]
    \centering
    \begin{ganttchart}[hgrid, vgrid,]{1}{12}
      \gantttitle{Week}{12} \\
      \gantttitlelist{1,...,12}{1} \\
      \ganttbar{\nameref{sec:task1}}{1}{4} \\
      \ganttbar{\nameref{sec:task2}}{5}{9} \\
      \ganttmilestone{Status Presentation}{6} \\
      \ganttbar{\nameref{sec:task3}}{7}{11} \\
      \ganttbar{Print and Submit}{12}{12} \\
      \ganttmilestone{Final Presentation}{12}
    \end{ganttchart}
    \caption{Schedule of the Bachelor's thesis}
    \label{fig:timeschedule}
  \end{figure}
}
{% for MSc Thesis
  \begin{figure}[htbp]
    \centering
    \begin{ganttchart}[hgrid, vgrid,]{1}{24}
      \gantttitle{Week}{24} \\
      \gantttitlelist{1,...,24}{1} \\
      \ganttbar{\nameref{sec:task1}}{1}{4} \\
      \ganttbar{\nameref{sec:task2}}{5}{9} \\
      \ganttmilestone{Status Presentation}{6} \\
      \ganttbar{\nameref{sec:task3}}{7}{17} \\
      \ganttbar{\nameref{sec:task4}}{18}{23} \\
      \ganttbar{Print and Submission}{24}{24} \\
      \ganttmilestone{Final Presentation}{24}
    \end{ganttchart}
    \caption{Schedule of the Master's thesis}
    \label{fig:timeschedule}
  \end{figure}
}

Describe each task (work package) in detail. Additionally, every work package
contains a list of must-haves and nice-to-haves that are checked off at the end
of the thesis. Do not forget to address the time schedule
(\fig{fig:timeschedule}) above and give a brief overview of the tasks in this
section.

\subsection{Milestone 1: Collect data}
\label{sec:milestone1}

Give an overview of what tasks are necessary to reach that milestone.

\subsubsection{Setup Crawler}
\label{sec:task1}
\lipsum[1]\\

\paragraph{Must-haves:}
\begin{itemize}
    \item X
    \item Y
    \item Z
\end{itemize}

\paragraph{Nice-to-haves:}
\begin{itemize}
    \item X
    \item Y
    \item Z
\end{itemize}

\subsubsection{Gather data}
\label{sec:task2}
\lipsum[1]

\subsection{Milestone 2}

Group all your tasks in individual milestones.

\subsubsection{Do more work}
\label{sec:task3}
\lipsum[1]

\paragraph{Must-haves:}
\begin{itemize}
    \item X
    \item Y
    \item Z
\end{itemize}

\paragraph{Nice-to-haves:}
\begin{itemize}
    \item X
    \item Y
    \item Z
\end{itemize}


\subsubsection{Do even more work}
\label{sec:task4}
\lipsum[1]

\paragraph{Must-haves:}
\begin{itemize}
    \item X
    \item Y
    \item Z
\end{itemize}

\paragraph{Nice-to-haves:}
\begin{itemize}
    \item X
    \item Y
    \item Z
\end{itemize}
\clearpage

\subsection{Milestone 3: Finalize Thesis}

Probably it would be best to schedule writing across the entire duration
of the thesis.

\subsubsection{Fix Gantt Width}
\label{sec:task5}
\lipsum[1]

\paragraph{Must-haves:}
\begin{itemize}
    \item X
    \item Y
    \item Z
\end{itemize}

\paragraph{Nice-to-haves:}
\begin{itemize}
    \item X
    \item Y
    \item Z
\end{itemize}


\newpage
\bibliographystyle{plain}
\bibliography{expose}


\newpage
\section*{Implementation}

This document is the foundation for the implementation of the
\emph{\thesistype} and describes the contents and goals that must be fulfilled.
Please consider the general \emph{\thesistype} requirements of your examination
regulations. If you are in doubt please contact the
\href{https://www.informatik.kit.edu/sul.php}{examination office}.

\subsubsection*{Task Definition and Supervision}
\vspace*{5mm}
\begin{tabularx}{\textwidth}{@{}XX}
    \mysupervisor & \rule[-0.5ex]{\linewidth}
    {1pt}\newline\itshape(Date, Signature)\\[5ex]
\end{tabularx}

\vspace*{10mm}

\subsubsection*{Student}
\vspace*{5mm}
\begin{tabularx}{\textwidth}{@{}XX}
    \myname & \rule[-0.5ex]{\linewidth}{1pt}\newline\itshape(Date, Signature)
\end{tabularx}

\end{document}
