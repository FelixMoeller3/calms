%% LaTeX2e class for student theses
%% thesis.tex
%% 
%% Karlsruhe Institute of Technology
%% Institute for Program Structures and Data Organization
%% Chair for Software Design and Quality (SDQ)
%%
%% Dr.-Ing. Erik Burger
%% burger@kit.edu
%%
%% See https://sdq.kastel.kit.edu/wiki/Dokumentvorlagen
%%
%% Version 1.3.6, 2022-09-28

%% Available page modes: oneside, twoside
%% Available languages: english, ngerman
%% Available modes: draft, final (see README)
\documentclass[oneside, english]{sdqthesis}

%% ---------------------------------
%% | Information about the thesis  |
%% ---------------------------------

%% Name of the author
\author{Felix Möller}

%% Title (and possibly subtitle) of the thesis
\title{Continual Active Learning for Effective Model Stealing Attacks}

%% Type of the thesis 
\thesistype{Bachelor's Thesis}

%% Change the institute here, ``KASTEL'' is default
% \myinstitute{Institute for \dots}

%% You can put a logo in the ``logos'' directory and include it here
%% instead of the SDQ logo
% \grouplogo{myfile}
%% Alternatively, you can disable the group logo
% \nogrouplogo

%% The reviewers are the professors that grade your thesis
\reviewerone{Jun.-Prof. Dr. Christian Wressnegger}
\reviewertwo{Prof. Dr. Thorsten Strufe}

%% The advisors are PhDs or Postdocs
\advisorone{M.Sc. Yilin Ji}
%% The second advisor can be omitted
\advisortwo{}

%% Please enter the start end end time of your thesis
\editingtime{10. January 2023}{10. May 2023}

\settitle

%% --------------------------------
%% | Bibliography                 |
%% --------------------------------

%% Use biber instead of BibTeX, see README
%\usepackage[nonumberlist,toc]{glossaries} 
\usepackage{glossaries} 
\makeglossaries



\newacronym{vae}{VAE}{Variational Autoencoder}
\newacronym{ewc}{EWC}{Elastic Weight Consolidation}
\newacronym{mas}{MAS}{Memory Aware Synapses}
\newacronym{imm}{IMM}{Incremental Moment Matching}
\newacronym{alasso}{ALASSO}{Asymmetric Loss Approximation by Single-Side Overestimation}
\newacronym{a-gem}{A-GEM}{Averaged Gradient Episodic Memory}
\newacronym{bald}{BALD}{Bayesian Active Learning by Disagreement}
\newacronym{badge}{BADGE}{Batch Active Learning by Diverse Gradient Embeddings}
\newacronym{lc}{LC}{Least Confidence}
\newacronym{cnn}{CNN}{Convolutional Neural Network}
\newacronym{si}{SI}{Synaptic Intelligence}
\newacronym{kl}{KL}{Kullback-Leibler}


% \newglossaryentry{Cold Start}{
%     % Explain difference to warm start
%     name={Cold Start},
%     plural={Cold Starts},
%     description={Cold Start is a term used in Active Learning literature and }
% }

% \newglossaryentry{Batch Size}{
%     name={Abc},
%     plural={Def},
%     description={GHI}
% }

% \newglossaryentry{ActiveThiefConv}{
%     name={ActiveThiefConv},
%     plural={ActiveThiefConvs},
%     description={proprietary Convolutional Neural Network Architecture used in the ActiveThief framework. ActiveThiefConv consists of 2,3 or 4 convolution blocks followed  by a fully connected layer. The architecture is shown in TODO }
%     %\cite{pal2020activethief}.
% }

% \newglossaryentry{SmallImagenet}{
%     name={SmallImagenet},
%     plural={SmallImagenets},
%     description={A subset of the ImageNet TODO.}
% }
\usepackage{dsfont}
\usepackage{hyperref}
\usepackage{tabularx}
\usepackage[citestyle=numeric,style=numeric,backend=biber]{biblatex}
\usepackage{algorithm}
\usepackage{algpseudocode}
%\renewcommand{\algorithmicrequire}{\textbf{Input:}}
\renewcommand{\algorithmicrequire}{\textbf{Input:}}
\newcommand{\algorithmicoutput}{\textbf{return}}
\newcommand{\return}{\item[\algorithmicoutput]}
\usepackage{amsmath}
\usepackage{amsthm}
\usepackage{makecell}
\newtheorem{theorem}{Theorem}
\usepackage{commath}
\usepackage{diagbox}
\usepackage{multirow}
\DeclareMathOperator*{\argmax}{\arg\!\max}
\DeclareMathOperator*{\argmin}{\arg\!\min}
\addbibresource{thesis.bib}

%% ====================================
%% ====================================
%% ||                                ||
%% || Beginning of the main document ||
%% ||                                ||
%% ====================================
%% ====================================
\begin{document}

%% Set PDF metadata
\setpdf

%% Set the title
\maketitle

%% The Preamble begins here
\frontmatter

\input{sections/declaration.tex}

\setcounter{page}{1}
\pagenumbering{roman}

%% ----------------
%% |   Abstract   |
%% ----------------
 
%% For theses written in English, an abstract both in English
%% and German is mandatory.
%%
%% For theses written in German, a German abstract is sufficient.
%%
%% The text is included from the following files:
%% - sections/abstract

\includeabstract
\printglossaries
\chapter{Notation}
\label{ch:notation}

In this chapter, we introduce the notation used throughout this thesis. The variables introduced in
the table below are used in the following chapters to describe the algorithms and experiments. They have
the meaning assigned to them in the table below unless explicitly stated otherwise.

\begin{tabularx}{\textwidth}{l X}
    \toprule
    Symbol & Definition \\
    \midrule
    $x_i$ & The $i^{th}$ data point in a dataset \\ \addlinespace
    $y_i$ & The label of the given data point $x_i$, also known as $y(x_i)$ \\ \addlinespace
    $X$ & The set of data points in a dataset \\ \addlinespace
    $Y$ & The set of labels in a dataset \\ \addlinespace
    $b$ & The number of samples within a batch in active learning \\ \addlinespace
    $U$ & The unlabeled pool in pool-based active learning\\ \addlinespace
    $L$ & The labeled pool in pool-based active learning\\ \addlinespace
    $\mathcal{L}$ & The loss function \\ \addlinespace
    $P$ & Patterns per experience, a hyperparameter of the \gls{a-gem} algorithm \\ \addlinespace
    $S$ & Samples drawn from memory to compute the reference gradients in \gls{a-gem} \\ \addlinespace
    $\theta$ & The parameters of a neural network \\ \addlinespace
    $O$ & The oracle in active learning \\ \addlinespace
    \bottomrule
\end{tabularx}
\label{tab:notation}

%% ------------------------
%% |   Table of Contents  |
%% ------------------------
\tableofcontents

\listoffigures
\listoftables

%% -----------------
%% |   Main part   |
%% -----------------

\mainmatter

%% LaTeX2e class for student theses
%% sections/content.tex
%% 
%% Karlsruhe Institute of Technology
%% Institute for Program Structures and Data Organization
%% Chair for Software Design and Quality (SDQ)
%%
%% Dr.-Ing. Erik Burger
%% burger@kit.edu
%%
%% Version 1.3.6, 2022-09-28

\chapter{Introduction}
\label{ch:Introduction}

Research in deep learning has produced an abundance of highly influential work in recent years, like
\glspl{cnn} \cite{lecun1998gradient}, transformers \cite{vaswani2017attention} and \glspl{gan} \cite{goodfellow2020generative}.
Machine learning models are becoming more accurate, achieving or surpassing human-level performance in visual
perception and natural language understanding. Because of their high innovation potential, machine learning models are increasingly
being used in real-world applications. To profit from this, many technology
enterprises offer services to ease training, development, and deployment of machine learning applications. These services can
be grouped under the term \gls{mlaas}. \par
While \gls{mlaas} is crucial to increase the accessibility of machine learning, it also poses a security threat. In recent years, numerous research papers
have been published that demonstrate how easily machine learning models can be extracted from \gls{mlaas} services \parencite{papernot2017practical,tramer2016stealing,
reith2019efficiently}. The standard procedure to steal such a machine learning model is to train a local
clone (also called substitute model) of the model hosted on the \gls{mlaas} service (also called target model). The substitute model is trained
using a dataset collected by the attacker. This dataset is also called the thief dataset. Labels for the thief dataset are obtained by querying the target
model on the thief dataset. The process of extracting machine learning models from \gls{mlaas} services is also called model stealing. \par
Because of the dramatic consequences of model stealing attacks, researchers have investigated how to defend against them. Two recent model stealing
defenses are prediction poisoning \cite{orekondy2019prediction} and \gls{prada} \cite{juuti2019prada}. Researchers have proposed attacks that evade
these defense strategies, despite the effort to
defend against model stealing attacks. One notable example is the model
extraction framework ActiveThief \cite{pal2020activethief}, which successfully evades the defense strategy  \gls{prada}. ActiveThief utilizes active
learning to determine which samples of the thief dataset it should query the target model on. Active learning is an intensively studied research
field that aims to minimize the labeling effort in the training process of machine learning models. However, the problem with active learning is
that it needs large amounts of computing resources. In this work, we extend the ActiveThief framework by using continual active learning in the model
extraction process.\par
Continual learning is a research field that aims to make machine learning models more robust against the advent of new data. The major problem
of the classic training procedure is that a model trained on a new task rapidly loses the ability to perform any previous tasks it was trained on.
Research in continual learning aims to develop methods that allow machine learning models to learn new tasks without forgetting the knowledge of old 
tasks. \par
\textbf{Delimitation} \hspace{0.2cm} We conduct all work in this thesis in the context of image classification. Furthermore, we conduct all experiments
with \glspl{cnn}. Therefore, our findings only apply to deep learning within the computer vision domain and may not generalize to other settings. \par
\textbf{Contributions} \hspace{0.2cm} The objective of this thesis is to combine existing approaches from continual learning
with approaches in the active learning domain. More specifically, we focus on regularization-based continual learning methods,
whereas we use uncertainty-based and diversity-based active learning methods. The continual learning methods we use are \gls{ewc} \cite{kirkpatrick2017overcoming},
\gls{mas} \cite{aljundi2018memory}, \gls{imm} \cite{lee2017overcoming} and \gls{alasso} \cite{park2019continual}.
Concerning active learning, we compare random sampling, \gls{lc} \cite{lewis1995sequential}, \gls{badge} \cite{ash2019deep}, \gls{bald} \cite{houlsby2011bayesian} and
CoreSet \cite{sener2017active}.
We analyze how the continual learning methods can speed up the active learning process, which combinations of continual learning and active learning methods
are most effective, and the trade-off between accuracy and speed up. While the focus of this thesis is on regularization-based continual
earning methods, we briefly explore the effectiveness of exemplar rehearsal continual earning using \gls{a-gem} \cite{chaudhry2018efficient}
combined with representation-based active learning in form of \gls{vaal} \cite{sinha2019variational}. Furthermore, we evaluate the performance of a custom Replay strategy.\par
After exploring the effectiveness of the different combinations of continual and active learning methods, we apply these combinations in the model stealing
domain. More specifically, we build upon the model extraction framework ActiveThief \cite{pal2020activethief} and investigate the performance of continual
active learning methods in the model stealing domain.
%% LaTeX2e class for student theses
%% sections/content.tex
%% 
%% Karlsruhe Institute of Technology
%% Institute for Program Structures and Data Organization
%% Chair for Software Design and Quality (SDQ)
%%
%% Dr.-Ing. Erik Burger
%% burger@kit.edu
%%
%% Version 1.3.6, 2022-09-28

\chapter{Background}
\label{ch:Background}

In this chapter we will give an overview of the background knowledge that is needed to understand the following chapters. The topic
of this thesis is at the intersection of the fields of Active Learning, Continual Learning and Model Stealing and therefore we will
give an introduction to each of these fields. 

\section{Active Learning}
\label{sec:ActiveLearning}

%TODO: Hier learner-Beschreibung anpassen
Active Learning is a subfield of Machine Learning that focuses on the problem of how to select the most informative data points to label.
Research in Active Learning is motivated by the fact that data acquisition is often easy because it can be automated whereas labeling is
difficult and time-consuming. Therefore, the goal of research in Active Learning is to develop methods which maximize model performance
with minimal amount of labeled data. A typical Active Learning scenario comprises a learner, an oracle, unlabeled data and labeled data.
The learner is the actual algorithm which selects which data points to label by the oracle. Let $I$ be the instance space (i.e. the set
of all possible data points) and $L$ be the label space (i.e. the set of all possible labels). The oracle $O$ can be seen as the function
\begin{equation}
    O: I \rightarrow L, x \mapsto y(x)
\end{equation}
where $y(x)$ is the true label of the data point $x$, often just referred to as $y$. The unlabeled data $U$ is a subset of $I$ just as the set
of labeled data $L$. At the beginning there are no labeled data points. Often a few labeled data points are randomly sampled from $U$ and labeled
by the oracle to initialize the learning process. A detailed overview on research activities within the Active Learning Domain is presented in
\cite{settles2009active}. Note however that more recent works are not contained in this literature survey because it was last updated in 2009.
In general, Active Learning methods can be divided into three categories:
\begin{itemize}
    \item \textbf{Query Synthesis Active Learning}
    \item \textbf{Pool-based Active Learning}
    \item \textbf{Stream-based Active Learning}
\end{itemize}
The taxonomy mentioned above is composed in a data-centric way. While researchers generally agree on the taxonomy mentioned above,
it is important to note that the effectiveness of an active learning strategy also depends on the type of Machine Learning Model that is used
(e.g. (Convolutional) Neural Networks, Support Vector Machines, etc.) and the type of data that is used (e.g. images or text).
For example, the pool-based Active Learning strategy CoreSet \cite{sener2017active} was specifically designed for CNNs.

\subsection{Query Synthesis Active Learning}
\label{sec:QuerySynthesisActiveLearning}
Query Synthesis Active Learning, also known as Membership Query Synthesis Active Learning, was among the first active learning scenarios
that were proposed \cite{angluin1988queries}. The idea behind this approach is to synthesize data points from the input space, rather than
to sample real data points. Nowadays, this is done by training a generative model (e.g a Generative Adversarial Network) \cite{zhu2017generative}
which learns the distribution of unlabeled data. However, earlier works relied on statistical models such as Gaussian Mixture Models 
\cite{cohn1996active}. The generated queries are then labeled by the oracle and can be used to train the Machine Learning Model. Note that
Query Synthesis is not limited to classification tasks. For example, \cite{cohn1996active} proposed a method to predict the absolute coordinates
of a robot hand when given the joint angles of its mechanical arms as inputs. When the oracle is a human annotator, Query Synthesis Active Learning
researchers have encountered problems in labeling them. Because the generated queries do not show any class-discriminative features, human annotators
struggled to assign any class to them in the survey of \cite{baum1992query}.

% TODO: Hier ein Bild wie QS Active Learning funktioniert einfügen

\subsection{Pool-based Active Learning}
\label{sec:PoolBasedActiveLearning}
Pool-based Active Learning is the most widely studied and used type of Active Learning. The idea behind pool-based Active Learning approaches
is to iteratively select the most informative data points from the current unlabeled pool, query the oracle for their labels and add them to the
labeled pool. Next, the ML model is trained on the current labeled pool. This process is repeated until the query budget is exhausted. A more detailed
explanation can be found in \href{alg:PoolBasedActiveLearning}{Algorithm 1}. Pool-based Active Learning strategies share this structure among each other.
The main difference between them is the informative measure, i.e. the criterion with which they select which data points to label next. Within the 
Pool-based Active Learning category, there are two subcategories: \textbf{Uncertainty-based} sampling and \textbf{Diversity-based} sampling. 
Uncertainty-based sampling strategies select the data points that the model is most uncertain about. Diversity-based sampling strategies on the other
hand aim to select data points that best represent the data distribution in the unlabeled pool. All the active learning strategies that we will use
are pool-based Active Learning Strategies, stemming both from the Uncertainty-based and Diversity-based subcategories.

\begin{algorithm}
    \caption{Pool-based Active Learning} \label{alg:PoolBasedActiveLearning}
    \begin{algorithmic}[1]
        \Require Unlabeled data $U$,Labeled data $L = \emptyset$:, Oracle $O$, Model $M$, budget $B$
        \State Select $k$ data points from $U$ at random, obtain labels by querying $O$ and set $L=\{x_1,\ldots,x_1\}$
        and $U = U \setminus \{x_1,\ldots,x_1\}$ \Comment{Initialization}
        \State Train $M$ on initial labeled set $L$
        \While{label budget $B$ not exhausted}
            \State Select $l$ data points from $U$ predicted to be the most informative by the Active Learning strategy
            \State Set $L= L \cup \{x_i,\ldots,x_l\}$ and $U = U \setminus \{x_i,\ldots,x_l\}$
            \State Train $M$ on labeled set $L$
        \EndWhile
    \end{algorithmic}
\end{algorithm}

\subsection{Stream-based Active Learning}
\label{sec:StreamBasedActiveLearning}
Stream-based Active Learning is closer to Pool-based Active Learning than Query Synthesis Active Learning. It was first introduced by Cohn et al. 
\cite{cohn1994improving}. The main difference between Stream-based Active Learning and Pool-based Active Learning is that data arrives sequentially
instead of having a batch of data points at once. In the stream-based Active Learning scenario, the learner draws a data point from the data source
one at a time. For each data point, the learner can then decide to query the oracle for its label or to discard it. The decision whether to label a
data point can either be made on the basis of its informativeness \cite{dagan1995committee} or its location within the instance space \cite{cohn1994improving}.
In the latter case the learner would label the data point if it is located in a region of the instance space that the learner is not confident about.

\section{Continual Learning}
\label{sec:ContinualLearning}
Continual Learning is a subfield of Machine Learning that aims to solve the problem of catastrophic forgetting. To elaborate further on this problem,
it is important to remember that most machine learning services are deployed in an environment where constant changes occur. To adapt to this, it is
necessary for Machine Learning Models to learn continually, i.e. to learn new tasks without forgetting the knowledge they have acquired in the past.
This is a common case for humans. For example, if a child has once learned to walk, it does not forget how to walk when it learns to ride a bike.
In contrast to human behavior, the performance of Machine Learning Models on old tasks rapidly decreases when they are trained on new tasks. This phenomenon
is known as \enquote{Catastrophic Forgetting} and was already discovered in the early days of Machine Learning research \cite{mccloskey1989catastrophic}.
In more general terms, research in Continual Learning is looking not only to alleviate Catastrophic Forgetting, but to solve the 
\enquote{stabiliy / plasticity dilemma} \cite{carpenter1988art}. The stability / plasticity dilemma refers to the fact that Machine Learning Models
should ideally be stable enough to retain their performance on old tasks while simultaneously being plastic enough to adapt to new tasks. This is a dilemma
because both properties are designed even though they are in conflict with each other. In practice, Machine Learning Models are rather plastic than stable.
This is especially true for Deep Neural Networks, but generally for all Machine Learning Models that are trained by greedily updating their parameters using
gradient descent \cite{mundt2020wholistic}. \\
Continual Learning is mainly used in scenarios where new tasks arrive over time or where the data distribution changes over time. Despite being useful
in these classic scenarios, Continual Learning approaches can also be used in cases where the data cannot be stored for legal reasons or due to memory
constraints, i.e. when the normal batch learning approach with a large training set cannot be applied. 


\subsection{Continual Learning Scenarios}
\label{sec:ContinualLearningScenarios}
Continual Learning is a rapidly evolving research field and terminology and taxonomy are still being established. Among the most important factors to distinguish
is the way in which new tasks arrive. In the following, we will introduce the three typical Continual Learning Scenarios as presented in \cite{van2022three}. 

\subsubsection{Task-Incremental Continual Learning}
\label{sec:TaskIncrementalContinualLearning}
%TODO: Add a figure that shows the difference between the three scenarios. For Task-Incremental learning mention tuple with task id.
In the Task-Incremental setting, a model is informed about which task it will be trained on or which task the data whose label it is supposed to predict belongs to.
The task-information is supplied via a task identifier which is usually an integer. Because the model does not have to infer the task which it is supposed to predict
or learn, it is possible to have task-specific components in the model. For neural networks, this means that there is one output layer per task and for each task the
respective output layer is used while all other layers are shared between all tasks. These classifiers are also known as \enquote{multi-head} classifiers \cite{van2018generative}.
An example would be the following: The first task is to classify images of cats and dogs. The second task is to classify cows and sheep. The model would have one output layer 
used to classify cats and dogs and a second output layer to classify cows and sheep. When training for the first task, only the first output layer is used. When the second task
arrives the model is retrained with the second output layer.  The mapping that is learned is 
\begin{equation}
    f: X \times T \rightarrow Y
\end{equation}
Where $X = \mathbb{R}^{N x N x C}$ is the input space (i.e. all possible input images of size $N x N$ with $C$ channels), $T = \{1,2,\ldots\}$ is the task space (i.e. all possible
tasks that the model can be trained on) and $Y = \mathbb{Z}_{k}$ is the label space with $k$ being the number of possible classes.
% Figure with decision tree for the continual learning scenarios like in https://www.nature.com/articles/s42256-022-00568-3
\subsubsection{Domain-Incremental Continual Learning}
\label{sec:DomainIncrementalContinualLearning}
In the Domain-Incremental setting, task identities are not given during evaluation. Since the underlying task does not change, this is also not necessary. While the structure of the task
stays the same it is rather the distribution of the data that changes. An example of Domain-Incremental Continual Learning would be the classification of digits between 0 and 9 (such as
those in the MNIST dataset) where the digits for one subtask are green and blue for the other. The mapping that is learned is 
\begin{equation}
    f: X \rightarrow Y
\end{equation}
Where $X = \mathbb{R}^{N x N x C}$ is the input space (i.e. all possible input images of size $N x N$ with $C$ channels), and $Y = \mathbb{Z}_{k}$ is the label space with $k$ being the
number of possible classes.

\subsubsection{Class-Incremental Continual Learning}
\label{sec:ClassIncrementalContinualLearning}
Class-Incremental Continual Learning is the most challenging scenario. Like in the Domain-Incremental setting, task identities are not provided at evaluation time, however, this time they
need to be inferred by the model. In this scenario, a classifier would be incrementally exposed to multiple tasks which contain different classes. An example would be again the classification
of digits between 0 and 9, where this time each task contains a disjoint subset of digits. The classifier would be trained on the first task, containing the digits 0 and 1, the second task,
containing the digits 2 and 3 and so on. During evaluation, the classifier would not only have to classify the digits correctly, but also to infer which task they belong to. The mapping that
is learned is 
\begin{equation}
    f: X \rightarrow Y \times T
\end{equation}
Where $X = \mathbb{R}^{N x N x C}$ is the input space (i.e. all possible input images of size $N x N$ with $C$ channels), $T = \{1,2,\ldots\}$ is the task space (i.e. all possible
tasks that the model can be trained on) and $Y = \mathbb{Z}_{k}$ is the label space with $k$ being the number of possible classes.

\subsection{Continual Learning Approaches}
\label{sec:ContinualLearningApproaches}
The Continual Learning approaches which have been proposed so far can be grouped into three categories according to, \cite{parisi2019continual} \cite{mundt2020wholistic} and \cite{zenke2017continual}.
Parisi et al. \cite{parisi2019continual} propose to group Continual Learning approaches into \textbf{regularization}, \textbf{rehearsal} and \textbf{architectural} approaches whereas Zenke et al.
\cite{zenke2017continual} group Continual Learning approaches into \textbf{architectural},\textbf{functional} and \textbf{structural} approaches. In the following, we will stick to the
categorization proposed by Parisi et al. because it is broader and fully encompasses the categorization by Zenke et al. Furthermore, Parisi et al.'s classification has been adopted by
recent continual learning reviews \cite{mundt2020wholistic}.

\subsubsection{Regularization Approaches}
\label{sec:RegularizationApproaches}
Regularization-based approaches to Continual Learning aim to prevent the forgetting of previous tasks by adding a regularization
term to the model's loss function. The regularization term is used as a proxy for how much the performance of the model on previous
tasks will decrease, i.e. a high regularization term indicates that the model will perform poorly on the old tasks with the current
weights and a low regularization term indicates that the model has not lost much knowledge of the old tasks. The way the
regularization term is computed can further be divided into two groups. There are \textbf{structural} approaches which regularize
based on weight changes to the model and there are \textbf{functional} approaches which regularize based on the output of the model.
Notable examples of structural approaches include Elastic Weight Consolidation (EWC) \cite{kirkpatrick2017overcoming},Memory Aware
Synapses (MAS) \cite{aljundi2018memory}, Incremental Moment Matching (IMM) \cite{lee2017overcoming} as well as Asymmetric Loss
Approximation by Single-Side Overestimation (ALASSO) \cite{park2019continual} which is an extension of Synaptic Intelligence (SI)
\cite{zenke2017continual}. All the structural regularization approaches will be covered in more detail in the section on
\href{sec:Related_work:Continual_Learning:Experiments}{the continual learning approaches used in the experiments}. \\
Functional regularization approaches are inspired by knowledge distillation \cite{hinton2015distilling}. They add a distillation
loss to the objective function which is computed based on the prediction of a data sample stored for future use. These data samples
are called soft targets. Li et al. \cite{li2017learning} compute the distillation loss by using the output of the newly arrived task
given by the model trained on the old tasks. The distillation loss they introduce aims to retain the prediction of the old model on
the new task even if the prediction itself may be inaccurate. The approach of Rannen et al. \cite{rannen2017encoder}, called Encoder
Based Lifelong Learning (EBLL) is based on the approach of Li et al., however in EBLL the distillation loss is computed based on
autoencoder reconstructions of old tasks.

\subsubsection{Rehearsal Approaches}
\label{sec:RehearsalApproaches}
Rehearsal approaches to Continual Learning aim to prevent catastrophic forgetting by fitting a model's parameters to the distribution
of an incoming task and all previous tasks simultaneously. Within Rehearsal approaches it is important to remember the trade-off between
model performance and computational cost. In general, the more data from previous tasks is used to train the model, the better the accuracy is. 
However, in order to train on more data, the data has to be stored and fed into the training process which is costly in terms of memory and This can either be done by replaying stored data from previous tasks or by
retraining on generated data drawn from the distribution of previous tasks. Continual Learning Methods that rely on the former idea are
categorized as \textbf{exemplar rehearsal} approaches while those that rely on the latter idea are categorized as \textbf{generative replay}
approaches. \\
Exemplar Rehearsal Techniques store data from previous tasks in a so-called \enquote{Replay Buffer} which data is sampled



\subsubsection{Architectural Approaches}
\label{sec:ArchitecturalApproaches}

\section{Model Stealing}
\label{sec:ModelStealing}
% TODO: Add figure with model stealing process
% Model Stealing notes: 
% Taxonomy provided by Jagielski et al. based on attacker's goal 
% Four types of attacks presented by He et al.: model extraction, model inversion, model poisoning and adversarial attack.
% Model extraction is a subfield of adversarial machine learning.
% Look at section 4.3 of Oliynyk et al. paper for taxonomy of adversarial ml.
% Mention some notation as in Olinyk et al. paper in section 5.1.
% Metrics to measure attack in 5.4 of Olinyk et al. paper.
% Have a look at Fig3 of Olinyk et al. paper and use it to tell which attacks we use (e.g. NNPD, Computer Vision, etc.)
% Based on Olinyk et al., our approach falls into the consistency category of section 6.1
% Model Stealing defenses are either used for attack detection or attack prevention.
With the advent of Machine Learning as a Service (MLaaS), an increasing amount of machine learning models are being exposed to the public via 
prediction APIs. The idea of these prediction APIs is that a user can request a prediction from a model by sending a request containing an unlabeled data
point to the API. The response to the request is then the prediction of the model for the data point. Prediction APIs are monetized by charging the user
on a pay-per-query principle. That means that each request has a fixed price for each query which is usually subtracted from his or her account balance on
a monthly basis. Providing public access to a machine learning model is a win-win situation because the model developer is compensated for his or her efforts
on gathering and labeling appropriate training data, choosing a proper model architecture, training the model and fine-tuning its hyperparameters while the user
of the prediction API can benefit from the model's predictions without having to train the model himself or herself. \\
Since the developed model is the intellectual property of the model developer, it is important that only the model's predictions are exposed to the public and
not the model (i.e. its architecture and the model weights)itself. However, in the last years numerous research papers have been published which demonstrate how
several features of a machine learning model can be extracted, i.e. its functionality \cite{tramer2016stealing}, its architecture \cite{oh2019towards} and its
training data \cite{shokri2017membership}. This newly created field of research is called \textbf{Model Stealing} or \textbf{Model Extraction}, and it is a subfield
of \textbf{Adversarial Machine Learning} according to Oliynyk et al. \cite{oliynyk2022know}. Because Model Extraction attacks are so effective, further research works
concerning model stealing defense mechanisms have been published. In the following we will use the taxonomy provided by Oliynyk et al.
\cite{oliynyk2022know} as a basis for our categorization of Model Stealing attacks and defenses. \\

\subsection{Terminology}
\label{sec:ModelStealing:Terminology}
Since Model Stealing is a rather new field it comes with numerous terms that are not commonly used in other fields of machine learning and others that are used 
synonymously. In order to avoid confusion, we will define the terms we use in this section in the following. \\
\textbf{Model Stealing Attack}: The process of maliciously querying a machine learning model in order to extract some or all of the model's features, such as its
architecture, its weights or its training data. Model Stealing Attacks are also known as Model Extraction Attacks or Model Inference Attacks among others. \\
\textbf{Target Model}: The model that is queried via the prediction API and whose features the attacker aims to extract. The target model has also been referred to 
as Oracle, Victim Model or Secret Model. \\



\subsection{Model Stealing Attacks}
\label{sec:ModelStealing:Attacks}

\subsection{Model Stealing Defenses}
\label{sec:ModelStealing:Defenses}
%% LaTeX2e class for student theses
%% sections/evaluation.tex
%% 
%% Karlsruhe Institute of Technology
%% Institute for Program Structures and Data Organization
%% Chair for Software Design and Quality (SDQ)
%%
%% Dr.-Ing. Erik Burger
%% burger@kit.edu
%%
%% Version 1.3.6, 2022-09-28

\chapter{Related Work}
\label{ch:Related_work}
Related work for this thesis can be grouped into three different categories:
The first category is \href{sec:Related_work:Active_Learning}{Active Learning}.
Active Learning is a special form of machine learning where an oracle is present
which can label arbitrary data points. A Machine Learning model trained using Active
Learning iteratively queries the oracle with unlabeled data points, trains a new model
and determines which data should be queried next. The second category is
\href{sec:Related_work:Continual_Learning}{Continual Learning}. Continual learning is a
machine learning technique which aims to make a given machine learning model learn new tasks
without forgetting the knowledge of previous tasks. 
The third category is \href{sec:Related_work:Model_Stealing}{Model Stealing}. Model Stealing is the
process of strategically querying a third-party machine learning model (also referred to as target model)
to train a local model (also referred to as substitute model) which is supposed to approximate the target model as
good as possible.

% use Settles as reference for the general introduction
% Mention that active learning is a specific form of semi-supervised learning (according to Mundt et al.)
\section{Active Learning}
\label{sec:Related_work:Active_Learning}
\subsection{General Introduction}
Active Learning is a specific form of machine learning where the learner can query an oracle to label arbitrary data points.
The motivation behind Active Learning is that nowadays it is not difficult to obtain large amounts of data, but the bottleneck
is assigning labels to them. Active Learning aims to overcome this issue by producing a highly accurate model with little amount
 of labeled data. The idea is that a learner which chooses the data it is trained on should perform better or at least as good
 as a model trained on all available data.
\subsection{Active Learning Approaches used in the Experiments}
\subsubsection{Least Confidence}
\subsubsection{CoreSet}
\subsubsection{BALD}
\subsubsection{Badge}
% Structure: First some general introduction then present
% the different approaches used in the experiments (LC, CoreSet, BALD, Badge)

% Use parisi et al. as reference for the general introduction
% Synaptic Intelligence paper also gives good introduction into approaches and structures them
% into three categories
% Also mention the difference between task-incremental, class-incremental and domain-incremental learning
\section{Continual Learning}
\label{sec:Related_work:Continual_Learning}
\subsection{General Introduction}
Artificial Neural Networks suffer from a problem called \enquote{Catastrophic Forgetting} \cite{mccloskey1989catastrophic}.
Catastrophic Forgetting is the phenomenon that the performance of a neural network previously trained on task $t_{n-k}$
severely decreases when the same neural network is later trained on task $t_n (k>0)$. More briefly, neural networks struggle
to retain the knowledge of previous tasks when learning new tasks. The problem of Catastrophic Forgetting has wide-ranging
implications for the use of neural networks because it limits their use to settings where the data at deployment time is
distributed identically to the data observed at training time. Clearly, this is not the case in most real-world applications.
Research in Continual Learning aims to develop mechanisms which alleviate Catastrophic Forgetting. The Continual Learning approaches
which have been proposed so far can be grouped into three categories according to \cite{mundt2020wholistic} and
\cite{zenke2017continual}. Mundt et al. \cite{mundt2020wholistic} propose to group Continual Learning approaches into
\textbf{regularization}, \textbf{rehearsal} and \textbf{architectural} approaches whereas Zenke et al. \cite{zenke2017continual}
group Continual Learning approaches into \textbf{architectural},\textbf{functional} and \textbf{structural} approaches. In the
following, we will stick to the categorization proposed by Mundt et al. because it is broader and fully encompasses the
categorization by Zenke et al.
\subsubsection{Regularization}
Regularization-based approaches to Continual Learning aim to prevent the forgetting of previous tasks by adding a regularization
term to the model's loss function. The regularization term is used as a proxy for how much the performance of the model on previous
tasks will decrease, i.e. a high regularization term indicates that the model will perform poorly on the old tasks with the current
weights and a low regularization term indicates that the model has not lost much knowledge of the old tasks. The way the
regularization term is computed can further be divided into two groups. There are \textbf{structural} approaches which regularize
based on weight changes to the model and there are \textbf{functional} approaches which regularize based on the output of the model.
Notable examples of structural approaches include Elastic Weight Consolidation (EWC) \cite{kirkpatrick2017overcoming},Memory Aware
Synapses (MAS) \cite{aljundi2018memory}, Incremental Moment Matching (IMM) \cite{lee2017overcoming} as well as Asymmetric Loss
Approximation by Single-Side Overestimation (ALASSO) \cite{park2019continual} which is an extension of Synaptic Intelligence (SI)
\cite{zenke2017continual}. All the structural regularization approaches will be covered in more detail in the section on
\href{sec:Related_work:Continual_Learning:Experiments}{the continual learning approaches used in the experiments}. \\
Functional regularization approaches are inspired by knowledge distillation \cite{hinton2015distilling}. They add a distillation
loss to the objective function which is computed based on the prediction of a data sample stored for future use. These data samples
are called soft targets. Li et al. \cite{li2017learning} compute the distillation loss by using the output of the newly arrived task
given by the model trained on the old tasks. The distillation loss they introduce aims to retain the prediction of the old model on
the new task even if the prediction itself may be inaccurate. The approach of Rannen et al.
\cite{rannen2017encoder}, called Encoder Based Lifelong Learning (EBLL) is based on the approach of Li et al., however in EBLL the
distillation loss is computed based on autoencoder reconstructions of old tasks.
\subsection{Continual Learning Approaches used in the Experiments}
\label{sec:Related_work:Continual_Learning:Experiments}
\subsubsection{EWC}
\subsubsection{MAS}
\subsubsection{ALASSO}
\subsubsection{IMM}

% Structure: First some general introduction then present
% the different approaches used in the experiments (EWC, MAS, ALASSO, IMM, potentially Replay?)


\section{Model Stealing}
\label{sec:Related_work:Model_Stealing}
% Hier Active Thief erwähnen und falls später noch defense strategies verwendet werden das ebenfalls noch erwähnen

\dots
%% ---------------------
%% | / Example content |
%% ---------------------
%\input{sections/content.tex}
%% LaTeX2e class for student theses
%% sections/methodology.tex
%% 
%% Karlsruhe Institute of Technology
%% Institute for Program Structures and Data Organization
%% Chair for Software Design and Quality (SDQ)
%%
%% Dr.-Ing. Erik Burger
%% burger@kit.edu
%%
%% Version 1.3.6, 2022-09-28

\chapter{Methodology}
\label{ch:Methodolody}

%% -------------------
%% | Example content |
%% -------------------
Since our continual active learning approach is, to the best of our knowledge, the first approach to combine pool-based active learning with continual
learning, we explain it in detail in this chapter, including its motivation. We then transfer our approach to the model stealing domain. Because we
build upon the framework of ActiveThief, we describe how our method compares to the original one in detail.

\section{Continual Active Learning}
\label{sec:Methodology:ContinualActiveLearning}
The main contribution of this thesis is combining the two learning paradigms of continual learning and active learning. To motivate the idea of combining
these two paradigms, we will first outline the classic continual learning setting and the classic active learning setting.
Next, we explain common issues with these two learning paradigms and how we aim to overcome these by combining both paradigms. Finally, we describe a custom
Replay strategy, which we use in our experiments.

\subsection{Classic Continual Learning Setting}
\label{sec:Methodology:CLSetting}
In the typical continual learning setting, the model is trained on a sequence of tasks. Each task $T_i = \{x_k,y_k | k \in \{1,\ldots,n\}\}$ is a set
of instances with their respective label. Together, the tasks form a dataset $D = \bigcup\limits_{i=1}^{N} T_i$. It is important to note that
the distribution of two distinct tasks $P(T_i)$  and $P(T_j) (i \neq j)$ are not necessarily the same. Often the tasks are independent, which is why
neural networks struggle to perform well on multiple tasks simultaneously. This also means that the size of two distinct datasets
can be different, i.e., $|T_i| \neq |T_j|$ and so can the number of classes $|\{y_k | \exists x_k: x_k,y_k \in T_i \}| \neq |\{y_l | \exists x_l: x_l,y_l
\in T_j\}|$. When training a model on a sequence of tasks, the model is first fed with the data of the first task $T_1$ and then trained on it. 
After the model was trained on the first task, it can either be trained on the next task or deployed to classify samples stemming from the
distribution of the first task. Next, the model is trained on the second task. After being trained on the second task, the model should now be able to
classify samples from the distribution of the first and second tasks. This process repeats until the model has been trained on all
tasks. At the end of the process, the model should be able to classify samples following the distribution of all the tasks it was trained on.
This workflow is illustrated in Figure \ref{fig:CLWorkflow}. \par
The main difference between the continual learning setting and classic machine learning is that in the classic machine learning setting, the model
not retrained once after deployment. In the continual learning setting, however, the model is retrained whenever a new task arrives.

\begin{figure}[ht]
    \centering
    \includegraphics[width=.9\linewidth]{images/CL_workflow.png}
    \caption[Continual learning workflow]{Example for the classic continual learning workflow. In this example, the model is first trained on different species of
    birds. It is then deployed to differentiate these species. Next, the model is trained on planes. After being deployed again, the model should now differentiate
    the planes as well as the birds.}
    \label{fig:CLWorkflow}
  \end{figure}

\subsection{Pool-based Active Learning Setting}
\label{sec:Methodology:ALSetting}
In the pool-based active learning setting, the model is trained sequentially on the current labeled pool. At first, the labeled pool is empty. Next, it is
initialized by randomly selecting samples from the unlabeled pool to label. On the contrary, the unlabeled pool contains the complete dataset at first and
is emptied in the process. After initializing the labeled pool, the model is trained on it. Next, $b$ samples from the labeled pool are selected until the
total budget is exhausted. The selected samples are the ones determined to be the most informative by the current active learning strategy. The oracle labels
them, and then they added to the labeled pool. Next, the model is trained on the current labeled pool, and the process repeats. We provide a more detailed
description of pool-based active learning in section \ref{sec:PoolBasedActiveLearning}. \par
Active learning contrasts with continual learning because all data stems from the same task. Another difference between pool-based active learning and
continual learning is that a model trained by active learning is trained on all previously selected batches. Consequently, the model is trained on some
data multiple times, which might cause parts of the training to be redundant.

\subsection{Combining Continual and Active Learning}
\label{sec:Methodology:CombiningCLandAL}
The problem with classic active learning is that it is very resource intensive. When training a model using pool-based active learning on a dataset of size $n$,
with batch size $b$, the model will be trained $\frac{n}{b}$ times on the current labeled pool, equating to $\frac{n(n+b)}{2b}$ data points overall (we provide a
short derivation of this number in \ref{sec:appendix:FirstSection}). The problem with the number of data points used for training is that it is dependent on the
batch size $b$. The smaller the batch size, the larger the overall number of data points that the model is trained on. In the extreme case of $b=1$, the model
is trained on $\frac{n(n+1)}{2}$ data points. While \cite{beck2021effective} note that the batch size has a negligible effect on the model performance, a typical
batch size is less than ten percent of the complete training set. Even in this more realistic case, the model is trained on $5.5n$ data points. When
comparing this to the classic continual learning setting, where the model is trained once using $n$ data points, it is clear that active learning comes with
considerable overhead. The overhead of active learning is even more pronounced when considering the running time of the active learning algorithms.
For more details, we refer to chapter \ref{ch:Evaluation} and \ref{ch:Discussion}. \par
On the other hand, a massive problem with classic continual learning is that task order significantly impacts the model performance \cite{bell2022effect}.
Hacohen et al. studied the problem of task ordering previously and showed that training samples in decreasing order of difficulty results in faster learning and
improved generalization error \cite{hacohen2019power}. This motivates us to use active learning to perform task ordering. We should note here that we
assume the free choice of the next task. This assumption is not always realistic, especially when tasks arrive sequentially, but studying the effect of task-ordering
benefits the study of these scenarios, too. Furthermore, insights into task-ordering help towards a more rigorous evaluation of continual learning in 
research because they allow us to assess the influence of task-ordering on experiment results when using classic benchmark datasets. \par
Our approach aims to overcome the issue of the overhead of active learning and the issue of task ordering by combining both learning paradigms. We modify the active
learning process by training only on the currently selected batch instead of the entire labeled pool. This way, the model is trained on $\sum_{i=1}^{\frac{n}{b}} b = n$ 
data points, which fully eliminates the overhead of active learning from a data-centric perspective. Nevertheless, the query time of the active learning algorithm remains
an overhead. From a continual learning perspective, we join all tasks to a single dataset and use this dataset as the unlabeled pool for active learning. In each iteration
of the active learning process, we select a batch $B$ from the unlabeled pool using the given active learning strategy. Next, the oracle labels this batch if the initial
dataset is unlabeled. If the initial dataset is labeled, we omit to query the oracle because the label is already known and add the labels
to the respective data points. We then treat the current batch $B$ as a new task and train our model using only the data points of $B$ with the continual learning strategy of
choice. This process repeats until the unlabeled pool is empty. The full algorithm is described in algorithm \ref{alg:PoolBasedContinualActiveLearning}, where we highlight
the difference between classic pool-based active learning and our new continual active learning approach in bold. \par

\begin{algorithm}
    \caption{Pool-based continual active learning} \label{alg:PoolBasedContinualActiveLearning}
    \begin{algorithmic}[1]
        \Require Unlabeled data $U$, Labeled data $L = \emptyset$, Oracle $O$, Model $M$, budget $B$
        \State Select $k$ data points from $U$ at random, obtain labels by querying $O$ and set $L=\{x_1,\ldots,x_1\}$
        and $U = U \setminus \{x_1,\ldots,x_1\}$ \Comment{Initialization}
        \State Train $M$ on initial labeled set $L$
        \While{label budget $B$ not exhausted}
            \State Select $l$ data points from $U$ predicted to be the most informative by the active learning strategy
            \State Obtain labels $y_i,\ldots,y_l$ by querying $O$ for $x_i,\ldots,x_l$
            \State \textbf{Train $M$ on current labeled batch $\{\mathbf{(x_i,y_i),\ldots,(x_l,y_l)}\}$}
            \State Set $L= L \cup \{x_i,\ldots,x_l\}$ and $U = U \setminus \{x_i,\ldots,x_l\}$
        \EndWhile
    \end{algorithmic}
\end{algorithm}

\subsection{Replay strategy}
\label{sec:Methodology:ReplayStrategy}
In this subsection, we describe a Continual Learning Strategy called Replay. Applying Replay to overcome catastrophic forgetting is not an entirely new idea.
On the contrary, Robins et al. first proposed using Replay in the 1990s \cite{robins1995catastrophic}. The approach we describe modifies the classic Replay
by compressing the replay buffer. To understand the modification, we must first understand the classic Replay strategy. \par
Replay is a continual learning strategy that stores a subset (or all) data points from each task in a so-called replay buffer. When training on a new task
$T_N$, the model is trained on all data from the current task plus a sample from the replay buffer. After training on task $T_N$, the replay buffer is updated
with the data points from the current task. \par
A significant drawback of the classic Replay strategy is that the replay buffer can grow to extreme sizes over time. This is especially true when the replay
buffer stores all data points from each task. It is more desirable to have a continual learning strategy with a fixed memory footprint because this reflects
real-world applications of continual learning where memory is limited. \par
Our proposed modification of the Replay strategy is to compress the replay buffer after each task. Assume we have a replay buffer of size $n$. After
training on task $T_N$, the replay buffer is combined with the data points from the current task to form a new set of data points $P$. We then select $n$
data points from $P$ to create the new replay buffer. These points are selected by performing one iteration of the active learning strategy CoreSet
\cite{sener2017active} with $P$ as the unlabeled pool and $n$ as the batch size. The $n$ data points selected by CoreSet then represents the new replay buffer.
When training on task $T_{N+1}$, the model is trained on all data from the current task plus the full replay buffer.

\section{Continual Active Learning for Model Stealing}
\label{sec:Methodolody:CALMS}
In this section, we describe how we apply the continual active learning approach to model stealing. Using the continual active learning approach
in the model stealing domain allows us to determine if the continual active learning achieves similar performance in the model stealing domain compared to
the standard setup where the oracle returns truthful labels. We motivate transferring our approach to the active learning domain by highlighting the
 differences between the setup mentioned in \ref{sec:Methodology:CombiningCLandAL} and continual active learning for model stealing. \par
The first difference is that the labels returned by the oracle in the model stealing domain have a limited semantic meaning. This is because the oracle
 is another machine learning model. So even if the data point whose label is queried stems from the same distribution as the target model dataset,
the label returned by the oracle is not necessarily correct since machine learning models do not generalize perfectly. Since the target model dataset
is unknown to the attacker, he cannot knowingly construct a thief dataset that overlaps with the target model dataset. Therefore, the data points the
 attacker queries the target model with will most likely be from a different distribution than the target model dataset and the labels returned by the target model
will be incorrect. The example in figure \ref{fig:CalmsWorkflow} highlights the meaning of the associated label. In the given example, the target model 
is trained o classify planes, which is why it associates the label \enquote{Fighter Jet} with a sparrow. \par
The second difference between the standard continual active learning approach and continual active learning for model stealing lies again in the labels
returned by the oracle. In the classic continual active learning approach, the labels of the oracle are not only truthful, but the label is just a
single value, i.e., the respective class. In the model stealing domain, the label returned by the oracle is either a single value or the per-class
probabilities of the output layer of the target model. The latter is the more interesting case for continual active learning because it reveals much more
information about the function learned by the target model. \par
Next, we describe how we apply continual active learning for model stealing. The first step is to use the thief dataset as the initial unlabeled pool for
active learning. The active learning strategy uses the thief dataset in each iteration, just as in classic continual active learning. After selecting the
most informative samples to label, the target model is queried for the labels of the selected samples. Next, the label returned, which is either a single
value or the per-class probabilities of the output layer of the target model, is used as the label of the respective data point throughout the complete
training process. This means that the gradient updates during training are based on the loss computed with a softmax label, allowing for a more fine-grained
optimization of the weights to approximate the target model function. This
process is repeated until the unlabeled pool is empty. A visual example of the continual active learning for model stealing workflow is shown in figure
\ref{fig:CalmsWorkflow}. \par
\begin{figure}[ht]
    \centering
    \includegraphics[width=.7\linewidth]{images/Calms_workflow.png}
    \caption[Continual active learning for model stealing workflow]{Example of continual active learning for model stealing workflow. In this example, the thief
    dataset consists of birds, while the target model was trained to classify planes. In each iteration, a batch of informative samples is selected by the active learning
    strategy. Next, the target model is queried for the labels of the selected samples. Since our thief dataset is composed of \gls{nnpd} data, the associated labels have
    an incorrect semantic meaning. The thief model is then trained on the selected samples from the current batch. This process is repeated iteratively.}
    \label{fig:CalmsWorkflow}
\end{figure}



%% ---------------------
%% | / Example content |
%% ---------------------
%% LaTeX2e class for student theses
%% sections/methodology.tex
%% 
%% Karlsruhe Institute of Technology
%% Institute for Program Structures and Data Organization
%% Chair for Software Design and Quality (SDQ)
%%
%% Dr.-Ing. Erik Burger
%% burger@kit.edu
%%
%% Version 1.3.6, 2022-09-28

\chapter{Experiment Setup}
\label{ch:ExperimentSetup}
% Describe hardware and software specification, datasets used, models used.
% Specifically mention which datasets are target datasets for model stealing
% and which are thief datasets
The findings of this thesis rely heavily on thorough experimentation. To reproduce our findings, we list in detail the
conditions under which we conducted the experiments. This includes training hyperparameters, datasets, Neural Network Architectures
and details of the code and libraries used. In general, our experiments can be divided into two categories: Experiments that explore
the classic Continual Active Learning Setting as in \ref{sec:Methodology:CombiningCLandAL} and experiments that explore Continual 
Active Learning for Model Stealing as in \ref{sec:Methodology:CombiningCLandALforMS}. We perform this differentiation because many 
training hyperparameters, datasets and Neural Network Architectures depend on the whether classic Continual Learning or Continual 
Active Learning for Model Stealing is investigated. First, we mention the general setup, including Hardware, Software, all datasets used
and the training hyperparameters shared between the two categories of experiments. Next, we go into detail about the special setup for
Continual Active Learning and Continual Active Learning for Model Stealing.

\section{General Experiment Setup}
\label{sec:ExperimentSetup:FirstSection}
%TODO: Some introductory words about the general setup

\subsection{Hardware}
\label{sec:ExperimentSetup:Hardware}
All experiments are conducted on the High Performance Computing (HPC) cluster bwUniCluster 2.0 \cite{bwUnicluster}. bwUniCluster 2.0 is an HPC cluster
funded by the Ministry of Science, Research and the Arts Baden-Württemberg and the Universities of the State of Baden-Württemberg. It currently consists
of more than 840 compute nodes with each node falling one of the following categories: \enquote{Thin}, \enquote{HPC}, \enquote{IceLake}, \enquote{Fat},
\enquote{GPUx4}, \enquote{GPUx8}, \enquote{GPUx4 A100}, \enquote{GPUx4 H100} and \enquote{Login}. In our experiments, we use the nodes GPUx4, GPUx8 and
GPUx4 A100. Their hardware specifications can be found in Appendix \ref{sec:Appendix:Specifications}.

\subsection{Software and Libraries}
\label{sec:ExperimentSetup:Software}
All the code used in this thesis is written in version 3.9.4 of the programming language Python \cite{Rossum1995Python}. Furhermore, we use the deep
learning Library PyTorch \cite{paszke2019pytorch} (version 1.13.1) to implement both Active and Continual Learning Algorithms. All further libraries
and their respective versions can be found in Appendix \ref{sec:Appendix:Specifications}.

\subsection{Continual Learning Strategies}
\label{sec:ExperimentSetup:CLStrategies}
In our experiments, we use the Continual Learning Strategies EWC, MAS, IMM and Alasso. For details on these Continual Learning strategies we refer
the reader to section \ref{sec:Related_work:Continual_Learning:Experiments}. As a baseline, we used the naive strategy of performing classic
gradient descent without any regularization. In the following we will refer to this strategy as \enquote{Naive}. \par
Following the paper introducing MAS \cite{aljundi2018memory}, we set the regularization parameter $\lambda$ to 1.0 for MAS. The choice of this parameter
is crucial because it enables a sound evaluation. If we trivially chose to set $\lambda$ to 0, then there would be no difference between MAS and the naive
strategy. On the other hand, fine-tuning neither possible (we use more than 25 combinations of Continual Learning strategies and Active Learning strategies 
on 3 datasets) nor fair (to evaluate the quality and effect of all Continual Learning strategies they should be tested under the same conditions, which
includes the same balance between old and new tasks). We therefore set $\lambda$ in EWC and IMM to 1.0 accordingly. The regularization parameter $c$ in
Alasso, which is equivalent to $\lambda$ in EWC, MAS and IMM is set to 0.5 for all experiments. We tried setting $c$ to 1.0, however we noticed that this
led to divergence during gradient descent despite employing heavy gradient clipping. Since employing stronger gradient clipping was not a viable solution
because it impedes model convergence within reasonable time, we decided to relax the regularization parameter $c$ to 0.5. We assume that the convergence 
problems Alasso faces are due to the fact that $\alpha$ in equation 3.15 makes the parameter space non-convex. \par
As mentioned before, we employ gradient clipping to avoid gradient descent from diverging. Not only is gradient clipping a viable remedy against the exploding
gradient problem, it has also been shown that it can accelerate the training process \cite{zhang2019gradient}. In our implementation we clip gradients by their
$l2$-Norm. Across all Continual Learning procedures, including Naive, we clip the gradients to a maximum $l2$-Norm of 20.0 to accelerate the training process.
When conducting experiments with MAS and Alasso we encountered exploding gradient problems, which we further investigated for MAS and Alasso separately. While we
aimed to eliminate the exploding gradient problem, which is mitigated more by clipping smaller gradient, we did not want to restrict the model too much to hinder 
model convergence. After carefully exploring with different values to clip at, we found setting the threshold for the $l2$-Norm to 2.0 to be effective, mitigating
the exploding gradient problem while simultaneously enabling model convergence. \par
Apart from the parameter weighting the regularization term, the Continual Learning strategies do not share any hyperparameters. EWC in fact does not have
any further hyperparameters and neither does MAS. On the other hand, for IMM we have the choice between mean-IMM and mode-IMM. Furthermore, we can choose
to apply Weight-transfer, L2-transfer as well as Dropout-transfer, and we can choose values for the $\alpha$ Parameter. In our setup we use mean-IMM, Weight-
Transfer and L2-transfer and set the $\alpha$ parameter to [0.45,0.55] (i.e. $\alpha_1 = 0.45, \alpha_2 = 0.55$), as suggested by the authors. For Alasso, we set
the parameter $a$, which controls the overestimation on the unobserved side, to 3.0. Following the recommendation of the authors, we perform parameter decoupling
for the $\Omega$ updates and therefore set $a'$ to 1.5 and $c'$ to 0.25. \par
A detailed summary of the hyperparameters used can be found in Appendix \ref{sec:Appendix:Specifications}.


\subsection{Active Learning Strategies}
\label{sec:ExperimentSetup:ALStrategies}
In our experiments, we use the Continual Learning Strategies Badge, LC, CoreSet and BALD. For details on these Active Learning strategies we refer
the reader to section \ref{sec:Related_work:Active_Learning:Approaches}. For Badge, we use the Euclidean distance in the $k$-means++ algorithm.
For CoreSet we use the Euclidean distance between the activations of the penultimate layer of the neural network as a distance metric for $k$-Center
algorithm. For BALD we use the Monte Carlo dropout with $T=25$ samples. We omit Monte Carlo dropout in cases where the model does not contain dropout
layers as the prediction of such a model is deterministic. LC does not contain any hyperparameters, so we do not have to specify any. \par
Across all Active Learning strategies we use the same initial budget, i.e. the number of points we sample from the training set before the first iter-
ation of the Active Learning algorithm, as the batch size.

\subsection{Datasets}
\label{sec:ExperimentSetup:Datasets}
In our experiments, we use the datasets MNIST \cite{mnist_web}, Fashion-MNIST \cite{xiao2017fashion}, CIFAR-10 \cite{cifar},
CIFAR-100 \cite{cifar}, Tiny-ImageNet \cite{le2015tiny} and a subset of the ILSVRC2012-14 dataset \cite{imagenet}. The subset of the ILSVRC2012-14 dataset that we
use is the first of 10 training batches downloaded from \url{http://www.image-net.org/data/downsample/Imagenet32_32.zip}. From now on, we will refer to this subset
of the ILSVRC2012-14 dataset as \enquote{SmallImagenet}. For all datasets, we use the standard train test split as proposed by Pytorch. We rescale all images to 32x32
pixels because for the success of a model stealing attack the image shape of the thief dataset has to match the image shape accepted by the target model. We will cover
this in more detail in section \ref{sec:Methodology:CALMSsetup}. Moreover, we normalize the train and test split of all datasets by using the mean and standard deviation
of the training set. For all datasets apart from MNIST and FashionMNIST, we further apply Data Augmentation. We use random horizontal flips with a probability of 0.5
followed by random cropping. A detailed list of the datasets used can be found in Appendix \ref{sec:Appendix:Specifications}. \par

\subsection{Neural Network Architectures}
\label{sec:ExperimentSetup:NNArchitectures}
In our experiments, we use the neural network architectures Resnet18 \cite{he2016deep} and CNN Architectures from the ActiveThief paper \cite{pal2020activethief}.
In the following, we will refer to this CNN architecture as \enquote{ActiveThiefConv}. Since there are multiple variations of the architecture, more specifically ones with
2,3 or 4 convolutional blocks we will refer to those as ActiveThiefConv2, ActiveThiefConv3 and ActiveThiefConv4 respectively.
%TODO hier Archtitekturen einfügen und beschreiben


\subsection{Training Hyperparameters}
\label{sec:ExperimentSetup:Hyperparameters}
% TODO: Überlegen, ob es noch mehr Hyperparameter gibt, die wir erwähnen sollten
We use PyTorch's SGD optimizer with a learning rate of 0.1 across all experiments. Furthermore, we schedule the learning rate by a factor of 0.1, but the number of
epochs after which we schedule the learning rate differs between experiments. It is important to note that we re-instantiate the SGD optimizer after each Active Learning
iteration. We will go more into detail on this in the description of the respective experiments. Apart from scheduling the learning rate, we use momentum \cite{cutkosky2020momentum}
of 0.9 and $l2$-regularization of 0.0005. For all experiments we use a batch size of 128 and shuffle the entire training set before each epoch.

\subsection{Randomness}
\label{sec:ExperimentSetup:Randomness}
%TODO: hier über Random number generation sprechen, erwähnen, dass kein fixed seed gewählt wird, und welche NN Initialisierung gewählt wird

\section{Special Setup for Continual Active Learning}
\label{sec:Methodology:CALsetup}
We experiment with Continual Active Learning using Resnet18 as our Neural Network Architecture and CIFAR-10 as our dataset. We perform experiments with a Batch Size of 1000,
2000 and 4000 respectively. Regardless of the batch size, we perform Active Learning until the unlabeled pool is exhausted. This means that we perform Active Learning for
49, 24 and 12 iterations respectively. For batch sizes 1000 and 2000 each query consists of 2000 points, while the last query for batch size 4000 consists of 2000 points.
For batch size 2000 and 4000 we train for 150 epochs per iteration, decaying the learning rate by a factor of 10 after 80 and 120 epochs respectively. For batch size 1000 we
train for 80 epochs and decay the learning rate by 10 after 60 epochs. To compare the performance of Continual Active Learning with pure Active Learning, we computed the results
for Active Learning with the same batch sizes as for Continual Active Learning. In this experiment setup, we use warm start and train for 200 epochs in each iteration. Here, we
decay the learning rate by 10 after 100 and 150 epochs respectively.

\section{Special Setup for Continual Active Learning for Model Stealing}
\label{sec:Methodology:CALMSsetup}
When transferring Continual Active Learning to the Model Stealing domain, we change our setup compared to Continual Active Learning previously. Instead of using Resnet18 as our
model architecture, we use the CNN architecture from the ActiveThief paper \cite{pal2020activethief}, mainly because we want to compare our Continual Active Learning approach to
the framework proposed by the authors of ActiveThief. For most of our experiments, we use ActiveThiefConv3 apart from those experiments which investigate the effect of the model 
architecture on Model Stealing Attacks. We train the target models on the datasets CIFAR-10, CIFAR-100 and MNIST. We train the target models on CIFAR-10 and CIFAR-100 for
150 epochs using momentum of 0.9 and $l2$-regularization of 0.0005 without learning rate decay. When training the target models on MNIST, we change the number of epochs to 50.
For all Model Stealing attacks we do \textit{not} retrain the target model. Instead, we train one target model per dataset, save them to the disk and load them for all Model Stealing
attacks. This way we reduce the uncertainty in our experiments introduced by different weight initializations of the target model. During all Model Stealing attacks, we use SmallImageNet
as our thief dataset. \par
Similar to the classic Continual Learning setting, we compute a baseline with Active Learning for Model Stealing. For this baseline, we use Active Learning with a batch size of 1000
and a total budget of 20000 points. The models are trained using cold start for 200 epochs per iteration. We decay the learning rate by a factor of 10 after 100 and 150 epochs. \par
For Continual Active Learning for Model Stealing, we use a batch size of 2000 and a total budget of 40000. We decided to increase the batch size for Continual Active Learning compared
to pure Active Learning because we noticed a clear correlation between batch size and model performance when training using Continual Active Learning. We train the substitute model for
150 epochs per iteration and decay the learning rate by a factor of 10 after 80 and 120 epochs with momentum of and $l2$-regularization of 0.0005.
% TODO: Mention setup of training target models and the one for stealing

%% LaTeX2e class for student theses
%% sections/evaluation.tex
%% 
%% Karlsruhe Institute of Technology
%% Institute for Program Structures and Data Organization
%% Chair for Software Design and Quality (SDQ)
%%
%% Dr.-Ing. Erik Burger
%% burger@kit.edu
%%
%% Version 1.3.6, 2022-09-28

\chapter{Evaluation}
\label{ch:Evaluation}

In this section we will present the results of our experiments. Our experiments are split into two parts. The first part is the evaluation of our novel
Continual Active Learning Approach and the second part is evaluating the performance our the Continual Active Learning approach when applied to Model Stealing.
In each of these respective subsections we will first describe the experiments schedule for each part and then present the results of the experiments.



\section{Continual Active Learning}
\label{sec:CAL}
In this section we will evaluate the results of our experiments using continual active learning in its classic setting. First, we experiment with regularization-based
continual learning strategies. Next, we evaluate the performance of our custom replay strategy from section \ref{sec:Methodology:ReplayStrategy}. Finally, we
test the combination of exemplar rehearsal continual learning and representation-based active learning.

\subsection{Regularization-based Continual Learning}
\label{sec:Evaluation:Results:CAL:ALRegCL}
We start our experiments by running each combination of the Active Learning strategies \gls{bald},\gls{badge},CoreSet,\gls{lc} and Random with the Continual Learning
strategies Naive, \gls{ewc}, \gls{mas}, \gls{alasso} and \gls{imm}. We use the dataset CIFAR-10, the Neural Network Architecture ResNet18 and a batch size of 4000 for these experiments. For each experiment,
we present the validation accuracy with increasing size of the labeled pool as well as the overall execution time of each experiment. The results by validation accuracy and by execution time are shown in 
Figure \cite{fig:Evaluation:CAL:4000bAcc} and Table \ref{fig:Evaluation:CAL:4000bTime}, respectively.\par

First, we evaluate the results for random sampling. In terms of execution time, there is a large gap between the baseline and the continual learning strategies. \gls{imm} is
the fastest method, followed by \gls{mas}, \gls{ewc} and Naive who perform on the same level. \gls{alasso} is the slowest continual learning strategy, but is still about six times as fast as the baseline. 
In terms of validation accuracy, the baseline outperforms all continual learning strategies by at least ten percentage points. Naive, \gls{ewc} and \gls{imm} demonstrate a similar accuracy
progression, with \gls{ewc} and Naive marginally outperforming \gls{imm}. \gls{mas} has a higher validation accuracy than the three aforementioned strategies in the beginning, but fails to keep up with their
continuous increase in validation accuracy. \gls{alasso} starts of with a higher validation accuracy than the other continual learning strategies, but falls behind at around 8,000 samples due to exploding
gradients. \par
 

Next, we re-run the previous experiment using \gls{lc}. In terms of execution time, the results are similar to the experiment with random sampling. \gls{alasso} is again the slowest strategy,
followed by \gls{mas}, \gls{ewc}, \gls{imm} and Naive. All continual learning strategies are significantly faster than the baseline, with \gls{alasso} being about six times as fast and Naive
about ten times as fast. The gap in validation accuracy between the baseline and the continual learning strategies remains significant and increases during the last 25,000 samples. 
\gls{imm}, \gls{ewc} and Naive perform almost identically across the experiment with \gls{mas} following closely and outperforming the remaining strategies within the last 5000 samples. The four aforementioned methods
experience a decrease in validation accuracy caused by the inferior representativeness of the final batches. \gls{alasso} starts competitively, but suffers from a heavy decrease in validation accuracy at around 8000 samples,
which is again caused by exploding gradients. \par


The third experiment is run with the active learning strategy \gls{bald}. The ranking of execution time of the continual
learning strategies and the baseline is similar to the previous experiments with Random and \gls{lc}. In terms of accuracy, the baseline outperforms the continual learning strategies significantly. The gap between
in validation accuracy is most prominent at roughly 15,000 samples and in the final batch. Naive, \gls{imm} and \gls{ewc} perform similar, showing an
s-shaped validation accuracy curve. The performance of \gls{mas} follows a similar curve, however \gls{mas} performs better in the first half of the experiment, and worse in the second half compared to the
three former strategies. Since we use ResNet18, which does not contain dropout layers, we are unable to run Monte Carlo dropout to accurately estimate $\mathbb{E}_{\theta \sim p(\theta \mid L)} [H[y \mid x, \theta]]$.
This leads to the inferior performance of \gls{bald} compared to the other active learning strategies. \gls{alasso} starts off as
the best continual learning strategy, but its validation accuracy decreases steadily over the course of the experiment, which is again caused by exploding gradients. \par 



We now run the experiment with the identical setup using CoreSet. The gap in execution time between the baseline and the continual learning 
strategies remains significant. \gls{imm} is the fastest continual learning strategy, being about 6 times as fast as the baseline. On the other hand, \gls{alasso} is the slowest Continual Learning strategy,
boasting around one fourth of the execution time of the baseline. In terms of validation accuracy, the accuracy progression strongly resembles the experiment with \gls{lc}. Naive is the best performing method
lacking 10 percentage points to the baseline during the first 25,000 samples. \gls{imm}, \gls{ewc} and \gls{mas} follow in that order. \gls{alasso} starts with a similar validation accuracy
as the other Continual Learning strategies, but falls behind after 8000 samples. All continual learning approaches have a declining validation accuracy in the final 10,000 samples because they are less representative
than the previous samples. \par



Our final experiment in this setting is run with \gls{badge}. Out of all experiments in this series, this one shows the
smallest gap in execution time between the baseline and the continual learning strategies. However, the baseline is still about 800 Minutes or 13 hours slower than all Continual Learning strategies. 
The ranking between the continual learning methods remains similar to previous experiments, with \gls{alasso} being the slowest and \gls{imm} being the fastest. Compared to the experiment with batch size 2000,
the validation accuracy curve of all Continual Learning strategies has become smoother and the gap between the baseline and the Continual Learning strategies has decreased. The only exception to this is \gls{alasso},
which has a negative slope in its validation accuracy curve in the first half of the experiment, followed by a slight increase in validation accuracy and a huge drop in the end. \gls{imm}, \gls{ewc} and Naive perform
similar for the first 45000 samples, where \gls{imm} falls behind the other two strategies. The validation accuracy of \gls{mas} is almost consistent throughout the experiment, with a slight
increase in the first half and a slight decrease in the second half. \par


\begin{figure}[h]
    \centering
    \includegraphics[width=0.32\linewidth]{images/results_CAL/random_4000b_acc.png} \hfill
    \includegraphics[width=0.32\linewidth]{images/results_CAL/lc_4000b_acc.png} \hfill
    \includegraphics[width=0.32\linewidth]{images/results_CAL/bald_4000b_acc.png}
    \\[\smallskipamount]
    \hfill 
    \includegraphics[width=0.32\linewidth]{images/results_CAL/coreset_4000b_acc.png} \hfill
    \includegraphics[width=0.32\linewidth]{images/results_CAL/badge_4000b_acc.png} \hfill
    \caption[Continual Active Learning with \gls{badge} with varying batch size]{Comparison of validation accuracy of regularization-based continual learning strategies
    with batch size 4000.}
    \label{fig:Evaluation:CAL:4000bAcc}
\end{figure}




% \begin{figure}[h]
%     \centering
%     \includegraphics[width=0.32\linewidth]{images/results_CAL/random_4000b_time.png} \hfill
%     \includegraphics[width=0.32\linewidth]{images/results_CAL/lc_4000b_time.png} \hfill
%     \includegraphics[width=0.32\linewidth]{images/results_CAL/bald_4000b_time.png}
%     \\[\smallskipamount]
%     \hfill 
%     \includegraphics[width=0.32\linewidth]{images/results_CAL/coreset_4000b_time.png} \hfill
%     \includegraphics[width=0.32\linewidth]{images/results_CAL/badge_4000b_time.png} \hfill
%     \caption{Comparison of execution time of regularization-based continual learning strategies
%     with batch size 4000.}
%     \label{fig:Evaluation:Results:CAL:4000bTime}
% \end{figure}

%TODO: Decide whether to use table or barplots
\begin{table}[h]
    \centering
    \begin{tabular}{c | c c c c c } 
         & Random & \gls{lc} & \gls{bald} & CoreSet & \gls{badge}\\ 
        \hline
        Naive & 82 & 77 & 78 & 160 & 497 \\
        \gls{ewc} & 82 & 83 & 82 & 145 & 493\\
        \gls{imm} & 76 & 78 & 76 & 129 & 487\\
        \gls{mas} & 81 & 88 & 81 & 145 & 501\\
        \gls{alasso} & 118 & 127 & 120 & 189 & 534\\
        \hline 
        Baseline & 717 & 712 & 721 & 782 & 1311 \\
    \end{tabular}
    \caption{Comparison of execution time of regularization-based continual learning strategies
    with batch size 4000.}
    \label{fig:Evaluation:CAL:4000bTime}
\end{table}



\subsubsection{Influence of Batch Size}
\label{sec:Evaluation:Results:CAL:BatchSize}

\begin{figure}[h]
    \centering
    \includegraphics[width=0.32\linewidth]{images/results_CAL/badge_1000b_acc.png} \hfill
    \includegraphics[width=0.32\linewidth]{images/results_CAL/badge_2000b_acc.png} \hfill
    \includegraphics[width=0.32\linewidth]{images/results_CAL/badge_4000b_acc.png}
    \caption[Continual Active Learning with \gls{badge} with varying batch size]{Comparison of validation accuracy of Continual Learning strategies used with the Active Learning strategy
    \gls{badge}.}
    \label{fig:Evaluation:Results:CAL:VaryBatchSizeAcc}
\end{figure}

% \begin{figure}[h]
%     \centering
%     \includegraphics[width=0.32\linewidth]{images/results_CAL/badge_1000b_time.png} \hfill
%     \includegraphics[width=0.32\linewidth]{images/results_CAL/badge_2000b_time.png} \hfill
%     \includegraphics[width=0.32\linewidth]{images/results_CAL/badge_4000b_time.png}
%     \caption[Continual Active Learning with \gls{badge} with varying batch size]{Comparison of validation accuracy of Continual Learning strategies used with the Active Learning strategy
%     \gls{badge}.}
%     \label{fig:Evaluation:Results:CAL:VaryBatchSizeTime}
% \end{figure}

\begin{table}[h]
    \centering
    \begin{tabular}{c | c c c c c c} 
        Batch Size & Baseline & Naive & \gls{ewc} & \gls{imm} & \gls{mas} & \gls{alasso}\\ 
        \hline 
        4000 & 1311 & 497 & 493 & 487 & 501 & 534 \\
        2000 & 1935 & 523 & 513 & 500 & 515 & 547 \\
        1000 & 3171 & 493 & 486 & 501 & 505 & 509 \\
    \end{tabular}
    \caption{Comparison of execution time of regularization-based continual learning strategies
    combined with \gls{badge}.}
    \label{fig:Evaluation:CAL:BadgeVaryBatchSizeTime}
\end{table}


\subsubsection{Delaying the Start of Continual Learning}
\label{sec:Evaluation:Results:CAL:Hybrid}
After running the experiments in section \ref{sec:Evaluation:Results:CAL:ALCL}, we notice that all of them demonstrate a significant discrepancy in validation accuracy between Active Learning and Continual Active Learning. To further investigate this gap,
we investigate a hybrid approach where we run Active learning for the first $i$ iterations before switching to Continual Active Learning. With these experiments we hope to decrease the gap in validation accuracy to Active Learning. We vary $i$ between 0 and 6,
and perform two sets of experiments, one using the Continual Learning strategy \gls{mas} and the other using \gls{ewc}. In both experiments, we use the Active Learning strategy \gls{badge}. The results of the two sets of experiments can be found in figure 
\ref{fig:Evaluation:Results:CAL:DelayedStart}. For \gls{mas} and \gls{ewc}, the validation accuracy drops immediately after switching from Active Learning to Continual Active Learning. However, in the long run the \gls{mas} strategy retains the validation accuracy better than \gls{ewc}.
We also notice that while the validation accuracy does drop after switching to Continual Active Learning, it is higher at a fixed cycle than when switching to Continual Active Learning in an earlier cycle. \par

%TODO: Hier die Bilder noch ändern sodass sie nicht aus Powerpoint sind
\begin{figure}[h]
    \centering
    \includegraphics[width=0.45\linewidth]{images/results_CAL/delayed_start_badge_ewc.png} \hfill
    \includegraphics[width=0.45\linewidth]{images/results_CAL/delayed_start_badge_mas.png}
    \caption[Continual Active Learning Hybrid approach]{Comparison of validation accuracy for a delayed start of Continual Learning. Left: Accuracy progression for \gls{badge} and \gls{ewc}. Right: Accuracy progression for \gls{badge} and \gls{mas}.}
    \label{fig:Evaluation:Results:CAL:DelayedStart}
\end{figure}

\subsubsection{Varying the initialization of the labeled pool}
\label{sec:Evaluation:Results:CAL:Initialization}
Motivated by the findings from the previous section, we wonder if the initialization of the labeled pool has an impact on the validation accuracy of the Continual Learning strategies. Beck et al.\cite{beck2021effective} show that using a facility location selection 
\cite{iyer2021submodular} yields better validation accuracy when training on the initial labeled pool. We therefore test the effect of an initialization using the facility location selection compared to our random initialization. As our Active Learning strategy,
we use \gls{badge} with a batch size of 4000 and \gls{mas} as our Continual Learning strategy. The results of the experiment can be found in figure \ref{fig:Evaluation:Results:CAL:FLinit}. While the facility location approach performs better than random initialization, the difference is
marginal. Interestingly, the validation accuracy of the facility location approach is lower than the random initialization in the first iteration. The most significant drawback of the facility location initialization is its resource-intensity. Initialization with facility
location takes about 24 hours to complete and requires more than 100GB of memory for CIFAR-10 using the implementation provided by Beck et al. \cite{beck2021effective}. \par

\begin{figure}[h]
    \centering
    \includegraphics[width=0.7\linewidth]{images/results_CAL/Facility_location_init.png}
    \caption[Initialization using Facility Location]{Comparison of validation accuracy for facility location initialization and random initialization. We use a batch size of 4000 and the combination of \gls{badge} and \gls{mas} for the experiments.}
    \label{fig:Evaluation:Results:CAL:FLinit}
\end{figure}


\subsection{Replay Continual Learning}
\label{sec:Evaluation:Results:CAL:Replay}
In section \ref{sec:Methodology:ReplayStrategy} we introduced a custom replay strategy for Continual Active Learning. The use of our replay strategy is motivated by the main finding of section \ref{sec:Evaluation:Results:CAL:ALCL}: the validation accuracy increases with an
increased batch size. In this set of experiments, we use the Active Learning strategy CoreSet and vary the batch size and the size of the replay buffer. Moreover, we investigate the effect of CoreSet selection in the replay buffer compared to random selection. We present our
results in figure \ref{fig:Evaluation:Results:CAL:Replay}. When varying the batch size and buffer size, we notice that increasing the buffer size and the batch size yields a higher validation accuracy. However, our replay strategy does not outperform the Naive approach when using
the same amount of training data (i.e. batch size + replay buffer size for the replay strategy and batch size for the Naive approach). To evaluate the importance of CoreSet selection in the buffer compression process, we compare the validation accuracy of our replay strategy
when using CoreSet selection and random selection. We notice that the validation accuracy remains largely the same for both buffer compression methods, apart from the last 15000 samples, where CoreSet selection outperforms random selection. \par

%TODO: Hier die Bilder noch ändern sodass sie nicht aus Powerpoint sind
\begin{figure}[h]
    \centering
    \includegraphics[width=0.45\linewidth]{images/results_CAL/replay_varying_batch_size.png} \hfill
    \includegraphics[width=0.45\linewidth]{images/results_CAL/replay_buffer_selection.png}
    \caption[Continual Active Learning Custom Replay strategy]{Left: Comparison of validation accuracy of our Replay strategy with different hyperparameters. Right: Comparison of validation accuracy of our Replay strategy when using different buffer compression approaches.
    In both runs, we use a batch size of 4000 and a replay buffer size of 4000.}
    \label{fig:Evaluation:Results:CAL:Replay}
\end{figure}

\subsection{Exemplar Rehearsal Continual Learning and Representation-based Active Learning}
\label{sec:Evaluation:Results:CAL:VAAL_AGEM}
Unsatisfied with the results from previous experiments, we decide to implement further Continual and Active Learning strategies. We perform an extensive literature search, investigating the suitability of representation-based Active Learning strategies and Continual Learning
strategies from the Exemplar Rehearsal category in the summary paper by Mundt et al. \cite{mundt2020wholistic}. The Active Learning strategy which we implement is \gls{vaal} \cite{sinha2019variational} and the Continual Learning strategy is \gls{a-gem} \cite{chaudhry2018efficient}. We decide
to implement \gls{vaal} because it is the only representation-based Active Learning strategy which consistently performs better than random sampling. The reason why we choose to implement \gls{a-gem} as our Continual Learning strategy is that is one of the few Exemplar Rehearsal strategies
applicable to (Semi-) supervised Learning (many other strategies focus on Reinforcement Learning) while being computationally efficient and demonstrating strong performance in the experiments by Chaudhry et al. \cite{chaudhry2018efficient} at the same time. \par
First, we analyze the performance of \gls{vaal} as an Active Learning strategy. We run \gls{vaal} with a batch size of 4000 and compare it to Random and CoreSet. We choose Random because it is the baseline for Active Learning strategies and CoreSet because it is among the best performing Active
we used before. The results can be found in the left plot in figure \ref{fig:Evaluation:Results:CAL:VAAL}. While \gls{vaal} outperforms random sampling by a large margin, it is itself outperformed by about the same margin by Coreset. To ensure that the results are not due to a
suboptimal hyperparameter choice, we run \gls{vaal} training VAE and Discriminator for 20 epochs in one run and 100 epochs in another. Because we are interested in the performance of \gls{vaal} in our Continual Active Learning setting, we use Continual Active Learning with a batch size of 4000
and the Continual Learning strategy Naive. We present the results of the experiment in the right plot of figure \ref{fig:Evaluation:Results:CAL:VAAL}. Surprisingly, the validation accuracy is marginally impacted by the training time of Generator and VAE. If anything, the validation
accuracy is higher when training for 20 epochs. \par

\begin{figure}[h]
    \centering
    \includegraphics[width=\linewidth]{images/results_CAL/VAAL_plots.png}
    \caption[Continual Active Learning Custom Replay strategy]{Left: Comparison of validation accuracy of \gls{vaal} to the Active Learning strategies Random and CoreSet. We use a batch size of 4000 for the experiments. Right: Comparison of validation accuracy when varying the training 
    epochs for \gls{vae} and Discriminator. We use Continual Active Learning with the Naive approach as our Continual Learning strategy and a batch size of 4000.}
    \label{fig:Evaluation:Results:CAL:VAAL}
\end{figure}

Next, we experiment with \gls{a-gem}. \gls{a-gem} has two hyperparameters: $S$, which is the number of samples randomly drawn from the memory to compute the reference gradients and $P$ which is the number of data points stored to the memory from each task. We run \gls{a-gem} with the Active Learning
strategy \gls{lc} and a batch size of 2000. To assess the performance of \gls{a-gem}, we compare its validation accuracy to the Naive approach. We vary $S$ and $P$ and present our results in the right plot of figure \ref{fig:Evaluation:Results:CAL:AGEM}. The validation accuracy increases with
an increased $S$ and $P$ until about 12000 samples, after which the validation accuracy drops for all values of $S$ and $P$. \gls{a-gem} outperforms the Naive approach for the first 15000 samples, but is outperformed by Naive for the remainder of the experiment. \par
After investigating the influence of \gls{a-gem}'s hyperparameters, we compare the combination of \gls{vaal} and \gls{a-gem} to \gls{vaal} and Naive, using a batch size of 4000. The results are shown in the left plot of figure \ref{fig:Evaluation:Results:CAL:AGEM}. Although \gls{a-gem} performs worse than the Naive
approach when using the Active Learning strategy \gls{lc}, it outperforms the Naive approach when using \gls{vaal} as the Active Learning strategy. \par

\begin{figure}[h]
    \centering
    \includegraphics[width=\linewidth]{images/results_CAL/AGEM_plots.png}
    \caption[Continual Active Learning Custom Replay strategy]{Left: Comparison of validation accuracy of \gls{a-gem} to the Naive approach when using \gls{vaal} as the Active Learning strategy We use a batch size of 4000 for the experiments. Right: Comparison of validation accuracy of \gls{a-gem}
     to the Naive approach when using different values of $S$ and $P$. We use the Active Learning strategy \gls{lc} and a batch size of 2000 for the experiments.}
    \label{fig:Evaluation:Results:CAL:AGEM}
\end{figure}


\begin{table}[h]
    \centering
    \begin{tabular}{c | c c c c c c} 
        Batch Size & \gls{vaal}\\ 
        \hline 
        Baseline & 961 \\
        Naive &  \\
        \gls{a-gem} & \\
    \end{tabular}
    \caption{Comparison of execution time of regularization-based continual learning strategies
    combined with \gls{badge}.}
    \label{fig:Evaluation:CAL:VAAL_AGEM_Time}
\end{table}

\section{Continual Active Learning for Model Stealing}
\label{sec:Evaluation:Results:MS}
After extensively studying Continual Active Learning, we shift the focus of our experiments to Model Stealing. Since we build our work on the ActiveThief framework, we first evaluate the performance of the ActiveThief framework, including the influence of Model Architecture and
Thief Dataset as well as the difference between stealing a target model returning the predicted labeled and stealing a target model returning the softmax outputs of the final model layer. After experimenting with the ActiveThief framework, we evaluate the performance of our pro
-posed Continual Active Learning strategy for Model Stealing attacks. We perform the experiments on the MNIST, CIFAR-10 and CIFAR-100 datasets and end with a decisive insight on the effect of data augmentation for Model Stealing attacks. \par


\subsection{Regularization-based Continual Learning}
\label{sec:Evaluation:Results:MS:Regularization}


In this section, we evaluate the success of Model Stealing Attacks using Continual Active Learning. More specifically, we run Model Stealing Attacks with Continual Active Learning on the datasets MNIST, CIFAR-10 and CIFAR-100, using the Active Learning strategies
Random, \gls{lc}, \gls{bald}, \gls{badge} and CoreSet and the Continual Learning strategies Naive, \gls{ewc}, \gls{imm}, \gls{mas} and \gls{alasso}. Furthermore, we differentiate between receiving the softmax output of the target model and solely receiving the top1-label. We use the ActiveThiefConv3 model
as the target and substitute model and perform one model stealing attack for each combination of Active Learning, Continual Learning strategy and target model output. For the baseline runs, we use a batch size of 1000 with a total query budget of 20000. The query budget
is kept for the Continual Active Learning runs, but we increase the batch size to 2000. The numbers reported in the table represent the model agreement at the end of each experiment, i.e. after using up the query budget of 20000. Readers interested in the progression of the
model agreement across the experiments can find the respective plots in appendix \ref{sec:Appendix:Results}. As a baseline we use the Model Extraction using Active Learning with the strategies mentioned before. To evaluate both the Active Learning strategies and the Continual
Learning strategies, we compute the average over the model agreement of all Continual Learning strategies combined with a fixed Active Learning strategy and the average of all Active Learning strategies combined with a fixed Continual Learning strategy. These numbers are given
at the end of each column and row, respectively. The Active Learning strategy and the Continual Learning strategy with the highest average model agreement are highlighted in bold. \par
In the first set of experiments, we perform Model Stealing Attacks using Continual Active Learning with MNIST as our target model and train the substitute model on the softmax output of the target model. The results of this experiment can be found in table
\ref{fig:ModelStealingMNISTSoftmax}. In terms of model agreement, all Continual Active Learning attacks perform significantly worse than the baseline. The best performing attack is the combination of the Active Learning strategy \gls{bald} with the Continual Learning strategy
\gls{mas} with a model agreement of 67.84\% after a query budget of 20000. The best performing Active Learning strategy is \gls{bald} with an average model agreement of 52.52\% across all Continual Learning strategies whereas the best Continual Learning strategy is \gls{mas} with an average
model agreement of 57.73\%. While the Continual Learning strategies struggled to outperform the Naive approach in the classic Continual Active Learning setting, most of them outperform the Naive approach in the Model Stealing setting. The only exception to this is \gls{alasso} which
falls behind the other approaches significantly. \par 

\begin{table}[h]
    \centering
    \begin{tabular}{ c | c c c c c | c } 
         & Random & \gls{lc} & \gls{bald} & CoreSet & \gls{badge} & $\varnothing$\\ 
        \hline
        Naive & 47.74 & 43.02 & 59.61 & 58.36 & 48.89 & 51.52\\
        \gls{ewc} &  59.26 & 53.67 & 59.18 & 56.13 & 50.28 & 55.7\\
        \gls{imm} & 48.18 & 62.95 & 49.71 & 63.43 & 58.76 & 56.61 \\
        \gls{mas} &  64.42 & 50.27 & 67.84 & 55.58 & 50.54 & \textbf{57.73}\\
        \gls{alasso} & 28.04 & 15.5 & 26.28 & 10.39 & 12.8 & 20.6\\
        \hline
        $\varnothing$ & 49.57 & 45.08 & \textbf{52.52} & 48.78 & 44.24 & /\\
        Baseline & 87.91 & 82.39 & 83.64 & 91.22 & 79.68 & 84.97\\
    \end{tabular}
    \caption{Model agreement of continual learning strategies on MNIST using softmax output}
    \label{fig:ModelStealingMNISTSoftmax}
\end{table}

Next, we change the output of the target model to solely return the top1-label and perform the same set of experiments. Table \ref{fig:ModelStealingMNISTLabel} shows the results of these experiments. While the model agreement compared to training with softmax output of the
target model has dropped significantly for the baseline strategies, we do not observed a similar phenomenon for the Continual Active Learning strategies. Nevertheless, we observe a large maring between the Model Agreement of the Active Learning strategies and the Model Agreement
of the Continual Active Learning strategies. However, the combination of \gls{bald} and Naive comes close to its baseline, demonstrating a Model Agreement of 69.18\% which is about 7 percentage points lower than the respective baseline. Overall, the Naive approach performs best with an
average Model Agreement of 51.29\% across all Active Learning strategies. Contrary to the previous setup, no Continual Learning strategy outperforms the Naive approach, with \gls{mas} coming closest. The best Active Learning strategy is \gls{bald} with an average Model Agreement of 55.73\%. \par 

\begin{table}[h]
    \centering
    \begin{tabular}{c | c c c c c | c } 
         & Random & \gls{lc} & \gls{bald} & CoreSet & \gls{badge} & $\varnothing$ \\ 
        \hline
        Naive & 43.44 & 58.84 & 69.18 & 44.88 & 40.12 & \textbf{51.29}\\
        \gls{ewc} &  46.76 & 47.3 & 52.36 & 36.84 & 44.73 & 45.6\\
        \gls{imm} & 47.07 & 10.43 & 58.71 & 51.0 & 47.0 & 42.84\\
        \gls{mas} & 50.52 & 46.79 & 49.51 & 44.96 & 49.47 & 48.25\\
        \gls{alasso} &  10.44 & 46.89 & 48.89 & 16.27 & 10.43 & 26.68\\
        \hline
        $\varnothing$ & 39.65 & 50.97 & \textbf{55.73} & 38.79 & 38.35 & /\\
        Baseline & 67.56 & 80.36 & 76.29 & 81.62 & 76.43 & 76.45\\
    \end{tabular}
    \caption{Model agreement of continual learning strategies on MNIST using the predicted class label}
    \label{fig:ModelStealingMNISTLabel}
\end{table}

We move on to the next dataset which is CIFAR-10. First, we perform Model Stealing attacks using the softmax output of the target model. Our results can be found in Table \ref{fig:ModelStealingCIFAR10Softmax}. Similar to our findings from the MNIST dataset, the Continual Active Learning
are outperformed by the Active Learning strategies, however the gap between is mostly consistent across combinations of Active and Continual Learning strategies at around 10 percentage points. \gls{alasso} is once again an exception to this, performing significantly worse than the other
Continual Learning strategies. The best Continual Active Learning strategy is \gls{ewc} with an average Model Agreement of 60.14\%. At the same time, \gls{ewc} is the only Continual Learning strategy to outperform the Naive approach, albeit the margin between the two is only 0.48 percentage points.
Overall, CoreSet is the most performant Active Learning strategy with an average model agreement of 54.12\% across all Continual Learning strategies. \par

\begin{table}[h]
    \centering
    \begin{tabular}{ c | c c c c c | c } 
         & Random & \gls{lc} & \gls{bald} & CoreSet & \gls{badge} & $\varnothing$\\ 
        \hline 
        Naive & 60.8 & 56.62 & 61.84 & 61.61 & 57.42 & 59.66 \\
        \gls{ewc} & 58.67 & 56.8 & 61.57 & 61.79 & 61.88 & \textbf{60.14}\\
        \gls{imm} & 60.78 & 52.24 & 60.32 & 61.39 & 56.98 & 58.34 \\
        \gls{mas} & 50.32 & 51.45 & 52.09 & 52.23 & 55.68 & 52.35\\
        \gls{alasso} & 17.26 & 24.87 & 28.24 & 33.59 & 32.93 & 27.38\\
        \hline
        $\varnothing$ & 49.57 & 48.4 & 52.81 & \textbf{54.12} & 52.98 & /\\
        Baseline & 71.58 & 70.04 & 71.96 & 71.45 & 71.44 & 71.29\\
    \end{tabular}
    \caption{Model agreement of Continual Learning strategies on CIFAR-10 using softmax output}
    \label{fig:ModelStealingCIFAR10Softmax}
\end{table}

In our next experiment setup, we use the predicted label of the target model to train the substitute model, again with CIFAR-10 as the target model dataset. We present the results of this set of experiments in Table \ref{fig:ModelStealingCIFAR10Label}. Overall,
the Model Agreement is lower than when training with the softmax output of the target model, which is in line with the experiments conducted on the MNIST dataset. The gap between the baseline and the Continual Learning strategies has remained the same however,
resting at around 10 percentage points. \gls{ewc} is again the best performing Continual Learning strategy, outperforming the Naive approach by a large margin this time around. \gls{imm} follows closely, also beating the Naive approach. The main reason for the large gap
between the Naive approach and \gls{ewc} as well as \gls{imm} is the poor performance of \gls{bald} and Naive. Surprised by the poor performance of this combination, we run it again and find that the results are consistent. \gls{mas} and \gls{alasso}, on the other hand, do not manage to
outperform the Naive approach. In terms of Active Learning strategies, CoreSet performs best with an average Model Agreement of 43.57\% across all Continual Learning strategies. \par

\begin{table}[h]
    \centering
    \begin{tabular}{ c | c c c c c | c } 
         & Random & \gls{lc} & \gls{bald} & CoreSet & \gls{badge} & $\varnothing$\\ 
        \hline
        Naive & 49.84 & 30.03 & 10.18 & 48.44 & 45.25 & 36.75\\
        \gls{ewc} & 50.37 & 47.12 & 50.22 & 47.94 & 49.11 & \textbf{48.95} \\
        \gls{imm} & 49.84 & 44.25 & 43.55 & 49.64 & 48.22 & 47.1\\
        \gls{mas} & 45.76 & 33.27 & 36.87 & 39.98 & 40.24 & 32.02\\
        \gls{alasso} & 19.05 & 38.16 & 30.91 & 31.86 & 25.04 & 29.0\\
        \hline
        $\varnothing$ & 42.97 & 38.57 & 34.35 & \textbf{43.57} & 41.57 & /\\
        Baseline & 60.23 & 49.73 & 61.28 & 63.78 & 62.26 & 59.46\\
    \end{tabular}
    \caption{Model agreement of continual learning strategies on CIFAR-10 using the predicted class label}
    \label{fig:ModelStealingCIFAR10Label}
\end{table}

The final target model dataset which we test for our setup is CIFAR-100. Like in the previous experiments, we first test all combination of Continual and Active Learning strategies trained on the softmax output of the target model
and compare them to the baseline of Active Learning. An exception to this are all experiments involving the Active Learning strategy \gls{badge}. We found the estimated runtime of \gls{badge} using ActiveThiefConv as our substitute model and SmallImagenet as our thief dataset
to be in excess of 40 days on our hardware, which is why we were unable to deliver results for this combination. The results for the remaining combinations from this set of experiments can be found in table \ref{fig:ModelStealingCIFAR100Softmax}.
Across the board, Model Agreement is significantly lower than in our previous experiments. The gap between the baseline and the Continual Learning strategies has decreased in absolute terms but increased in relative terms. Like in our previous experiments, \gls{ewc} outperforms
the remaining Continual Learning strategies, and it is one of two Continual Learning strategies which manage to outperform the Naive approach. The other strategy is \gls{imm}, which puts on a performance in between \gls{ewc} and the Naive approach. \gls{mas} follows behind the Naive approach
and \gls{alasso} is left behind once again. The best Active Learning strategy is CoreSet, which outperforms \gls{bald} by 0.57 percentage points. \par

\begin{table}[h]
    \centering
    \begin{tabular}{ c | c c c c | c } 
         & Random & \gls{lc} & \gls{bald} & CoreSet & $\varnothing$\\ 
        \hline
        Naive & 18.46 & 18.48 & 16.8 & 17.69 & 17.86\\
        \gls{ewc} & 19.45 & 17.46 & 20.67 & 19.98 & \textbf{19.39}\\
        \gls{imm} & 18.16 & 17.9 & 20.39 & 18.75 & 18.8\\
        \gls{mas} & 15.75 & 14.85 & 14.72 & 15.45 & 15.19\\
        \gls{alasso} & 5.2 & 4.95 & 6.7 & 10.24 & 6.77\\
        \hline
        $\varnothing$ & 15.4 & 14.73 & 15.85 & \textbf{16.42} & /\\
        Baseline & 25.43 & 26.92 & 28.01 & 27.48 & 26.96 \\
    \end{tabular}
    \caption{Model agreement of continual learning strategies on CIFAR-100 using softmax output}
    \label{fig:ModelStealingCIFAR100Softmax}
\end{table}

Finally, we compute the Model Agreement across multiple combinations of Continual Learning and Active Learning strategies using the predicted label of the target model and CIFAR-100 as our target model dataset. As in the previous set of experiments, we omit the results of
the Active Learning strategy \gls{badge}. Compared to training on the softmax output of the target model, the model agreement has disproportionally decreased for the Continual Learning strategies. While the baseline strategies boast a Model Agreement of 20.42\% on average, the
best Continual Learning strategy, which is \gls{imm}, manages to achieve a Model Agreement of 8.07\% on average. \gls{imm} significantly outperforms both the Naive approach and \gls{ewc}, which demonstrate almost identical performance. We were surprised by the poor performance of \gls{ewc} and \gls{bald},
which is why we conducted this experiment one more time, however we achieved similar results. The remaining Continual Learning strategies, \gls{mas} and \gls{alasso}, perform worse than the Naive approach. Remarkably, \gls{alasso} outperforms \gls{mas}, making this the only set of experiments in which
\gls{alasso} is not the worst performing Continual Learning strategy. Another surprise is the performance of the Random Active Learning strategy, which outperforms the remaining Active Learning strategies. CoreSet, which demonstrated strong performance in previous experiments, falls behind
Random and \gls{lc}, outperforming only \gls{bald} by about one percentage point. \par

\begin{table}[h]
    \centering
    \begin{tabular}{ c | c c c c | c } 
         & Random & \gls{lc} & \gls{bald} & CoreSet & $\varnothing$\\ 
        \hline
        Naive & 7.93 & 7.63 & 4.98 & 4.82 & 6.34\\
        \gls{ewc} & 8.79 & 6.55 & 2.07 & 7.77 & 6.3\\
        \gls{imm} & 8.59 & 8.44 & 7.18 & 8.05 & \textbf{8.07}\\
        \gls{mas} & 5.37 & 5.44 & 5.3 & 4.52 & 5.16\\
        \gls{alasso} & 5.28 & 6.11 & 5.72 & 5.51 & 5.66\\
        \hline
        $\varnothing$ & \textbf{7.19} & 6.83 & 5.05 & 6.13 & /\\
        Baseline & 21.65 & 19.5 & 19.64 & 20.9 & 20.42\\
    \end{tabular}
    \caption{Model agreement of continual learning strategies on CIFAR-100 using the predicted class label}
    \label{fig:ModelStealingCIFAR100Label}
\end{table}

\subsection{Exemplar Rehearsal Continual Learning and Representation-based Active Learning}
\label{sec:Evaluation:CALMS:VAAL_AGEM}

After having conducted our experiments with the Continual Learning strategies Naive, \gls{ewc}, \gls{mas}, \gls{imm} and \gls{alasso} as well as the Active Learning strategies
Random, \gls{bald}, \gls{lc}, CoreSet and \gls{badge}, we test the combination of the Active Learning strategy \gls{vaal} with the Continual Learning strategy \gls{a-gem}. We motivate
this experiment by the performance of \gls{vaal} and \gls{a-gem} in the classic Continual Learning setup, given in Figure \ref{sec:Evaluation:Results:CAL:VAAL_AGEM}.
The experiments are performed in the same manner as the previous experiments in this section. To evaluate the performance of \gls{vaal} and \gls{a-gem}, we compare
the results to the best performing Active Learning strategy, CoreSet, and the combination of the best performing Continual Learning strategy and the
most performant Active Learning strategy, which are \gls{ewc} and CoreSet, respectively. The results of the comparison are given in Table 
\ref{fig:ModelStealingVAALAGEM}. We present the progression in Model Agreement with \gls{vaal} and \gls{a-gem} in Appendix \ref{sec:Appendix:Results}. While \gls{vaal}
and \gls{a-gem} cannot keep up with the performance of the baseline, just as all Continual Active Learning approaches, it performs on par with CoreSet and \gls{ewc},
outperforming them in 3 out of 6 experiments while being outperformed in the remaining 3 experiments. \par
\begin{table}[h]
    \centering
    \begin{tabular}{c | c  c  c  c  c  c } 
        \multirow{2}*{Attack strategy}& \multicolumn{2}{c}{MNIST} & \multicolumn{2}{c}{CIFAR-10} & \multicolumn{2}{c}{CIFAR-100}  \\ 
         & Softmax & Label & Softmax & Label & Softmax & Label \\
        \hline 
        \gls{vaal} \gls{a-gem} & 52.54 & 53.8 & 61.48 & 50.5 & 18.29 & 8.77\\
        CoreSet \gls{ewc} & 56.13 & 36.84 & 61.79 & 47.94 & 19.98 & 7.77 \\
        Coreset Baseline & 90.65 & 77.58 & 71.61 & 61.68 & 27.52 & 20.96\\
    \end{tabular}
    \caption{Comparison of model agreement using \gls{vaal} and \gls{a-gem}}
    \label{fig:ModelStealingVAALAGEM}
\end{table}

%% LaTeX2e class for student theses
%% sections/conclusion.tex
%% 
%% Karlsruhe Institute of Technology
%% Institute for Program Structures and Data Organization
%% Chair for Software Design and Quality (SDQ)
%%
%% Dr.-Ing. Erik Burger
%% burger@kit.edu
%%
%% Version 1.3.6, 2022-09-28

\chapter{Discussion}
\label{ch:Discussion}
% Was war das Ziel? Was wurde (nicht) erreicht?
In this chapter, we will discuss our findings from the experiments in chapter \ref{ch:Evaluation}. Like most chapters in this thesis, we will divide this chapter into two parts, one for continual active learning and one for model stealing. 
The first section will discuss the findings from the first batch of experiments, where we tested Continual Active Learning on the CIFAR-10 dataset using Resnet18. The section on Model Stealing is then used to discuss general findings from
our experiments on Model Stealing as well as findings from applying Continual Active Learning to Model Stealing. 

\section{Continual Active Learning}
\label{sec:Discussion:ContinualActiveLearning}
The goal of this section is to discuss the findings from


\section{Model Stealing}
\label{sec:Discussion:ModelStealing}
%% LaTeX2e class for student theses
%% sections/conclusion.tex
%% 
%% Karlsruhe Institute of Technology
%% Institute for Program Structures and Data Organization
%% Chair for Software Design and Quality (SDQ)
%%
%% Dr.-Ing. Erik Burger
%% burger@kit.edu
%%
%% Version 1.3.6, 2022-09-28

\chapter{Conclusion}
\label{ch:Conclusion}
% Mention drops in accuracy due to Continual Active Learning. Regarding model stealing
% mention that there are cases where the CAL works a lot better than for normal continual
% active learning. Mention correlation with batch size to support the thesis that CAL fails
% because batch size is not large enough. Mention small difference between different active
% learning approaches for CIFAR-10 at least, which are inline with the findings from Yilins
% paper.
This is the conclusion of the thesis.
\dots

\section{Future Work}
\label{sec:Conclusion:FutureWork}

%% --------------------
%% |   Bibliography   |
%% --------------------

%% Add entry to the table of contents for the bibliography
\printbibliography[heading=bibintoc]

%% ----------------
%% |   Appendix   |
%% ----------------
\appendix
%% LaTeX2e class for student theses
%% sections/apendix.tex
%% 
%% Karlsruhe Institute of Technology
%% Institute for Program Structures and Data Organization
%% Chair for Software Design and Quality (SDQ)
%%
%% Dr.-Ing. Erik Burger
%% burger@kit.edu
%%
%% Version 1.3.6, 2022-09-28

\iflanguage{english}
{\chapter{Appendix}}    % english style
{\chapter{Anhang}}      % german style
\label{chap:appendix}


%% -------------------
%% | Example content |
%% -------------------
\section{Proof of Data points trained using Active Learning}
\label{sec:appendix:FirstSection}
\begin{theorem}
Let $X$ be a set of data points of size $n$, $b < n, n \mod b = 0$ be the batch size. Then a machine learning model trained
using pool-based Active Learning is trained $\frac{n}{b}$ times on the current labeled pool of data points. Overall, the model
is trained on $\frac{n(n+b)}{2b}$ data points
\end{theorem}
\begin{proof}
    The first part is trivial, given that the model is trained until the labeled pool is exhausted and in each iteration $b$
    points are queried for their label. To determine the total number of points used, we need to sum up the number of points
    used in each iteration. 
    \begin{equation}
        \sum_{i=1}^{\frac{n}{b}} i \cdot b
    \end{equation}
    Using the formula for the sum of the first $n$ natural numbers, we get
    \begin{equation}
        \sum_{i=1}^{\frac{n}{b}} i \cdot b = b \cdot \sum_{i=1}^{\frac{n}{b}} i = b \cdot \frac{\frac{n}{b} (\frac{n}{b} + 1)}{2}
        = \frac{n (\frac{n}{b} + 1)}{2} = \frac{n(n+b)}{2b}
    \end{equation}
\end{proof}

\section{Experiment Setup}
\label{sec:Appendix:Setup}
In this section, we will list tables and figures containing the specifications of the hardware and software used in our experiments.

\subsection{Hardware}
\label{sec:Appendix:Hardware}
Since we run our experiments on BwUnicluster2.0, we provide a detailed list of the hardware specifications of the nodes we used. Please
note that this table only contains the nodes we used in our experiments. For a more detailed list of all nodes as well as all further
information, see \cite{bwUniclusterHardware}

\newpage

\begin{table}[!htb]
    \begin{tabularx}{\textwidth}{|X | X X X|} 
        \hline
         & GPU x4 & GPU x8 & GPU x4 A100 \\ 
        \hline 
        Processors & Intel Xeon Gold 6230 & Intel Xeon Gold 6248 & Intel Xeon Platinum 8358  \\ 
        Number of sockets & 2 & 2 & 2  \\ 
        Processor frequency (GHz) & 2.1 & 2.6 & 2.5  \\ 
        Total number of cores & 40 & 40 & 64  \\ 
        Main memory & 384 GB & 768 GB & 512 GB  \\ 
        Local SSD & 3.2 TB NVMe & 15TB NVMe & 6.4 TB NVMe  \\ 
        GPUs & 4x NVIDIA Tesla V100 & 8x NVIDIA Tesla V100 & 4x NVIDIA A100  \\ 
        GPU Memory & 32 GB  & 32 GB & 80 GB  \\ 
        Interconnect & IB HDR & IB HDR & IB HDR200  \\ 
        \hline
    \end{tabularx}
    \caption{Hardware configuration for the three nodes used on BWUniCluster2.0. }
    \label{fig:HardwareSpec}
\end{table}


\subsection{Software}
\label{sec:Appendix:Software}
In this section, we provide a comprehensive list of the libraries we used in our experiments as well as their version
and a link to their documentation.

\begin{table}[!htb]
    \centering
    \begin{tabular}{|l | l l |} 
        \hline
        Library Name & Version & Link \\ 
        \hline 
        numpy & 1.23.0 & \url{https://numpy.org/} \\
        tqdm & 4.64.1 & \url{https://tqdm.github.io/}  \\
        torchvision & 0.14.1 & \url{https://pytorch.org/vision/stable/index.html} \\ 
        torch & 1.13.1 & \url{https://pytorch.org/} \\
        PyYAML & 6.0 & \url{https://pyyaml.org/} \\
        scipy & 1.10.1 & \url{https://scipy.org/} \\
        scikit-learn & 1.2.1 & \url{https://scikit-learn.org/stable/index.html} \\
        wget & 3.2 & \url{https://pypi.org/project/wget/} \\
        \hline
    \end{tabular}
    \caption{Software Libraries used for the experiments}
    \label{fig:Libraries}
\end{table}

\subsection{Continual Learning strategies}
\label{sec:Appendix:CLStrategies}
In this section, we extend the parameter description of the Continual Learning strategies given in section 
\ref{sec:ExperimentSetup:CLStrategies} by presenting a table for with the exact hyperparameter configuration
used in the experiments.

\begin{table}[!htb]
    \begin{tabularx}{\textwidth}{|l | X | l |} 
        \hline
        Hyperparameter & Description & Value \\ 
        \hline 
        $\lambda$ & Balances between old and new tasks. A higher value indicates more focus 
        on preserving knowledge from the old task & 1.0  \\ 
        \hline
        Sample size & Relative size of the sample compared to the full training set that is used to 
        compute the Fisher Matrix & 0.05  \\ 
        \hline
        Gradient Clip & Maximum Value of the $l2$-Norm of the gradient. If the current norm is larger,
        the gradient is clipped to that value & 20.0 \\ 
        \hline
    \end{tabularx}
    \caption{Hyperparameter configuration for \gls{ewc}.}
    \label{fig:EWCparams}
\end{table}

\begin{table}[!htb]
    \begin{tabularx}{\textwidth}{| l | X | l |} 
        \hline
        Hyperparameter & Description & Value \\ 
        \hline 
        $\lambda$ & Balances between old and new tasks. A higher value indicates more focus
        on preserving knowledge \newline from the old task & 1.0  \\ 
        \hline
        Gradient Clip & Maximum Value of the $l2$-Norm of the gradient. If the current norm is larger, the
        gradient is clipped to that value & 2.0 \\ 
        \hline
    \end{tabularx}
    \caption{Hyperparameter configuration for \gls{mas}.}
    \label{fig:MASparams}
\end{table}

\begin{table}[!htb]
    \begin{tabularx}{\textwidth}{| l | X | l |} 
        \hline
        Hyperparameter & Description & Value \\ 
        \hline 
        $\lambda$ & Balances between old and new tasks. A higher value indicates more focus
        on preserving knowledge from the old task & 1.0  \\ 
        \hline
        Gradient Clip & Maximum Value of the $l2$-Norm of the gradient. If the current norm is larger, the
        gradient is clipped to that value & 20.0 \\ 
        \hline
        $\alpha$ & Weights the importance of the previous tasks. The sum of all $\alpha$ values must be 1.0 & 0.45,0.55 \\
        \hline
    \end{tabularx}
    \caption{Hyperparameter configuration for \gls{imm}.}
    \label{fig:IMMparams}
\end{table}

\begin{table}[!htb]
    \centering
    \begin{tabularx}{\textwidth}{| l | X | l |} 
        \hline
        Hyperparameter & Description & Value \\ 
        \hline 
        $a$ & Balances between old and new tasks. A higher value indicates more focus
        on preserving knowledge from the old task & 0.5  \\ 
        \hline
        $a'$ & Used to update parameter importances in Equation \hyperref[eq:ALASSO_Small_Omega]{3.23} & 0.25  \\
        \hline
        Gradient Clip & Maximum Value of the $l2$-Norm of the gradient. If the current norm is larger, the
        gradient is clipped to that value & 2.0 \\ 
        \hline
        $c$ & Determines the overestimation of the loss on the unobserved side & 3.0 \\
        \hline
        $c'$ & Used to update parameter importances in Equation \hyperref[eq:ALASSO_Small_Omega]{3.23} & 1.5 \\
        \hline
    \end{tabularx}
    \caption{Hyperparameter configuration for \gls{alasso}.}
    \label{fig:AlassoParams}
\end{table}

\subsection{Datasets}
\label{sec:Appendix:Datasets}

\begin{table}[!htb]
    \centering
    \begin{tabularx}{\textwidth}{|X | X X X X|} 
        \hline
         & MNIST & CIFAR-10 & Tiny \newline ImageNet & Small \newline ImageNet \\ 
        \hline 
        Size of training set & 60000 & 50000 & 100000 & 128116 \\ 
        Size of validation set & 10000 & 10000 & 10000 & 50000\\
        Number of classes & 10 & 10 & 200 & 1000 \\
        Image Dimension & 28x28 & 32x32 & 64x64& 32x32\\
        Number of channels & 1 & 3 & 3 & 3 \\
        Source & torchvision & torchvision & {\small \url{http://cs231n.stanford.edu/tiny-imagenet-200.zip}} & {\small \url{http://www.image-net.org/data/downsample/Imagenet32_32.zip}} \\
        \hline
    \end{tabularx}
    \caption{Information on the datasets used for the experiments.}
    \label{fig:DatasetInformtion}
\end{table}

\subsection{Neural Network Architectures}
\label{sec:Appendix:Architectures}
In our experiments with Model Stealing, we use the same architectures as proposed in \cite{pal2020activethief}. Pal et al. propose a proprietary
\gls{cnn} architecture family which consists of $l$ convolution blocks followed by a single fully connected layer. Each convolution block consists
of 
\begin{enumerate}
    \item a convolutional layer with $k$ filters size $3 \times 3$,
    \item a ReLU activation layer,
    \item a batch normalization layer,
    \item a convolutional layer with $k$ filters size $3 \times 3$,
    \item a ReLU activation layer,
    \item a batch normalization layer,
    \item a max pooling layer with kernel size $2 \times 2$,
\end{enumerate}
where $k=32 \cdot 2^{i-1}$ is the number of filters for block number $i \in \{1,\ldots,l\}$.

\begin{figure}[!htb]
    \centering
    \includegraphics[width=\linewidth]{images/ActiveThiefConvs.png}
    \caption[ActiveThiefConv Architectures]{Example Network Architectures for CIFAR-10.}
    \label{fig:ActiveThiefArchitectures}
\end{figure}



\section{Experimental Results}
\label{sec:Appendix:Results}
This section is intended to provide extended material to selected experiments in section \ref{sec:Evaluation:Results}. We especially
include additional plots for the Continual Active Learning experiments which we were not able to include in the main text due to
space constraints.

\subsection{Target and Substitute Model Comparison}
\label{sec:Appendix:TargetSubstituteComparison}
Here we include the progression of Model Agreement for the experiments conducted in section \ref{sec:Evaluation:Results:MS:ActiveThief}.

\begin{figure}[!htb]
    \centering
    \includegraphics[width=\linewidth]{images/results_CALMS/cifar10_model_comp.png}
    \caption{Agreement Progression for Model Stealing on CIFAR-10 using different ActiveThief models}
    \label{fig:CIFAR10modelComp}
\end{figure}

\begin{figure}[!htb]
    \centering
    \includegraphics[width=\linewidth]{images/results_CALMS/mnist_model_comp.png}
    \caption{Agreement Progression for Model Stealing on MNIST using different ActiveThief models}
    \label{fig:MNISTmodelComp}
\end{figure}


\subsection{Continual Active Learning for Model Stealing}
\label{sec:Appendix:CALMS}
This section contains plots of the full runs of all experiments in section \ref{sec:Evaluation:Results:MS:CAL}.

\subsubsection{MNIST}
\label{sec:Appendix:CALMS:MNIST}
In this section, we present the full runs of all experiments which involve Continual Active Learning using MNIST as a Target Model Dataset.


\begin{figure}[!htb]
    \centering
    \includegraphics[width=0.5\linewidth]{images/results_CALMS/mnist_label_random.png}
    \caption{Agreement Comparison for Model Stealing on MNIST using the predicted class label and the Active Learning strategy Random}
    \label{fig:CALMSMNISTLabelRandom}
\end{figure}

\begin{figure}[!htb]
    \centering
    \includegraphics[width=0.5\linewidth]{images/results_CALMS/mnist_label_lc.png}
    \caption{Agreement Comparison for Model Stealing on MNIST using the predicted class label and the Active Learning strategy \gls{lc}}
    \label{fig:CALMSMNISTLabelLC}
\end{figure}

\begin{figure}[!htb]
    \centering
    \includegraphics[width=0.5\linewidth]{images/results_CALMS/mnist_label_bald.png}
    \caption{Accuracy Comparison for Model Stealing on MNIST using the predicted class label and the Active Learning strategy \gls{bald}}
    \label{fig:CALMSMNISTLabelBALD}
\end{figure}

\begin{figure}[!htb]
    \centering
    \includegraphics[width=0.5\linewidth]{images/results_CALMS/mnist_label_coreset.png}
    \caption{Agreement Comparison for Model Stealing on MNIST using the predicted class label and the Active Learning strategy CoreSet}
    \label{fig:CALMSMNISTLabelCoreSet}
\end{figure}

\begin{figure}[!htb]
    \centering
    \includegraphics[width=0.5\linewidth]{images/results_CALMS/mnist_label_badge.png}
    \caption{Agreement Comparison for Model Stealing on MNIST using the predicted class label and the Active Learning strategy \gls{badge}}
    \label{fig:CALMSMNISTLabelBadge}
\end{figure}

%Softmax
\begin{figure}[!htb]
    \centering
    \includegraphics[width=0.5\linewidth]{images/results_CALMS/mnist_softmax_random.png}
    \caption{Agreement Comparison for Model Stealing on MNIST using the softmax output and the Active Learning strategy Random}
    \label{fig:CALMSMNISTSoftmaxRandom}
\end{figure}

\begin{figure}[!htb]
    \centering
    \includegraphics[width=0.5\linewidth]{images/results_CALMS/mnist_softmax_lc.png}
    \caption{Agreement Comparison for Model Stealing on MNIST using the softmax output and the Active Learning strategy \gls{lc}}
    \label{fig:CALMSMNISTSoftmaxLC}
\end{figure}

\begin{figure}[!htb]
    \centering
    \includegraphics[width=0.5\linewidth]{images/results_CALMS/mnist_softmax_bald.png}
    \caption{Agreement Comparison for Model Stealing on MNIST using the softmax output and the Active Learning strategy \gls{bald}}
    \label{fig:CALMSMNISTSoftmaxBALD}
\end{figure}

\begin{figure}[!htb]
    \centering
    \includegraphics[width=0.5\linewidth]{images/results_CALMS/mnist_softmax_coreset.png}
    \caption{Agreement Comparison for Model Stealing on MNIST using the softmax output and the Active Learning strategy CoreSet}
    \label{fig:CALMSMNISTSoftmaxCoreSet}
\end{figure}

\begin{figure}[!htb]
    \centering
    \includegraphics[width=0.5\linewidth]{images/results_CALMS/cifar_softmax_badge.png}
    \caption{Agreement Comparison for Model Stealing on MNIST using the softmax output and the Active Learning strategy \gls{badge}}
    \label{fig:CALMSMNISTSoftmaxBadge}
\end{figure}

\subsubsection{CIFAR-10}
\label{sec:Appendix:CALMS:CIFAR}
In this section, we present the full runs of all experiments which involve Continual Active Learning using CIFAR-10 as a Target Model Dataset.
%Label
\begin{figure}[!htb]
    \centering
    \includegraphics[width=0.5\linewidth]{images/results_CALMS/cifar_label_random.png}
    \caption{Agreement Comparison for Model Stealing on CIFAR-10 using the predicted class label and the Active Learning strategy Random}
    \label{fig:CALMSCIFAR10LabelRandom}
\end{figure}

\begin{figure}[!htb]
    \centering
    \includegraphics[width=0.5\linewidth]{images/results_CALMS/cifar100_label_lc.png}
    \caption{Agreement Comparison for Model Stealing on CIFAR-10 using the predicted class label and the Active Learning strategy \gls{lc}}
    \label{fig:CALMSCIFAR10LabelLC}
\end{figure}

\begin{figure}[!htb]
    \centering
    \includegraphics[width=0.5\linewidth]{images/results_CALMS/cifar100_label_bald.png}
    \caption{Agreement Comparison for Model Stealing on CIFAR-10 using the predicted class label and the Active Learning strategy \gls{bald}}
    \label{fig:CALMSCIFAR10LabelBALD}
\end{figure}

\begin{figure}[!htb]
    \centering
    \includegraphics[width=0.5\linewidth]{images/results_CALMS/cifar100_label_coreset.png}
    \caption{Agreement Comparison for Model Stealing on CIFAR-10 using the predicted class label and the Active Learning strategy CoreSet}
    \label{fig:CALMSCIFAR10LabelCoreSet}
\end{figure}

\begin{figure}[!htb]
    \centering
    \includegraphics[width=0.5\linewidth]{images/results_CALMS/cifar_label_badge.png}
    \caption{Agreement Comparison for Model Stealing on CIFAR-10 using the predicted class label and the Active Learning strategy \gls{badge}}
    \label{fig:CALMSCIFAR10LabelBadge}
\end{figure}

%Softmax
\begin{figure}[!htb]
    \centering
    \includegraphics[width=0.5\linewidth]{images/results_CALMS/cifar100_softmax_random.png}
    \caption{Agreement Comparison for Model Stealing on CIFAR-10 using the softmax output and the Active Learning strategy Random}
    \label{fig:CALMSCIFAR10SoftmaxRandom}
\end{figure}

\begin{figure}[!htb]
    \centering
    \includegraphics[width=0.5\linewidth]{images/results_CALMS/cifar100_softmax_lc.png}
    \caption{Agreement Comparison for Model Stealing on CIFAR10 using the softmax output and the Active Learning strategy \gls{lc}}
    \label{fig:CALMSCIFAR10SoftmaxLC}
\end{figure}

\begin{figure}[!htb]
    \centering
    \includegraphics[width=0.5\linewidth]{images/results_CALMS/cifar100_softmax_bald.png}
    \caption{Agreement Comparison for Model Stealing on CIFAR-10 using the softmax output and the Active Learning strategy \gls{bald}}
    \label{fig:CALMSCIFAR10SoftmaxBALD}
\end{figure}

\begin{figure}[!htb]
    \centering
    \includegraphics[width=0.5\linewidth]{images/results_CALMS/cifar100_softmax_coreset.png}
    \caption{Agreement Comparison for Model Stealing on CIFAR-10 using the softmax output and the Active Learning strategy CoreSet}
    \label{fig:CALMSCIFAR10SoftmaxCoreSet}
\end{figure}

\begin{figure}[!htb]
    \centering
    \includegraphics[width=0.5\linewidth]{images/results_CALMS/cifar_softmax_badge.png}
    \caption{Agreement Comparison for Model Stealing on CIFAR-10 using the softmax output and the Active Learning strategy \gls{badge}}
    \label{fig:CALMSCIFAR10SoftmaxBadge}
\end{figure}


\subsubsection{CIFAR-100}
\label{sec:Appendix:CALMS:CIFAR100}
In this section, we present the full runs of all experiments which involve Continual Active Learning using CIFAR-100 as a Target Model Dataset.

\begin{figure}[!htb]
    \centering
    \includegraphics[width=0.5\linewidth]{images/results_CALMS/cifar100_label_random.png}
    \caption{Agreement Comparison for Model Stealing on CIFAR-100 using the predicted class label and the Active Learning strategy Random}
    \label{fig:CALMSCIFAR100LabelRandom}
\end{figure}

\begin{figure}[!htb]
    \centering
    \includegraphics[width=0.5\linewidth]{images/results_CALMS/cifar100_label_lc.png}
    \caption{Agreement Comparison for Model Stealing on CIFAR-100 using the predicted class label and the Active Learning strategy \gls{lc}}
    \label{fig:CALMSCIFAR100LabelLC}
\end{figure}

\begin{figure}[!htb]
    \centering
    \includegraphics[width=0.5\linewidth]{images/results_CALMS/cifar100_label_bald.png}
    \caption{Agreement Comparison for Model Stealing on CIFAR100 using the predicted class label and the Active Learning strategy \gls{bald}}
    \label{fig:CALMSCIFAR100LabelBALD}
\end{figure}

\begin{figure}[!htb]
    \centering
    \includegraphics[width=0.5\linewidth]{images/results_CALMS/cifar100_label_coreset.png}
    \caption{Agreement Comparison for Model Stealing on CIFAR-100 using the predicted class label and the Active Learning strategy CoreSet}
    \label{fig:CALMSCIFAR100LabelCoreSet}
\end{figure}


\begin{figure}[!htb]
    \centering
    \includegraphics[width=0.5\linewidth]{images/results_CALMS/cifar100_softmax_random.png}
    \caption{Agreement Comparison for Model Stealing on CIFAR-100 using the softmax output and the Active Learning strategy Random}
    \label{fig:CALMSCIFAR100SoftmaxRandom}
\end{figure}

\begin{figure}[!htb]
    \centering
    \includegraphics[width=0.5\linewidth]{images/results_CALMS/cifar100_softmax_lc.png}
    \caption{Agreement Comparison for Model Stealing on CIFAR-100 using the softmax output and the Active Learning strategy \gls{lc}}
    \label{fig:CALMSCIFAR100SoftmaxLC}
\end{figure}

\begin{figure}[!htb]
    \centering
    \includegraphics[width=0.5\linewidth]{images/results_CALMS/cifar100_softmax_bald.png}
    \caption{Agreement Comparison for Model Stealing on CIFAR-100 using the softmax output and the Active Learning strategy \gls{bald}}
    \label{fig:CALMSCIFAR100SoftmaxBALD}
\end{figure}

\begin{figure}[!htb]
    \centering
    \includegraphics[width=0.5\linewidth]{images/results_CALMS/cifar100_softmax_coreset.png}
    \caption{Agreement Comparison for Model Stealing on CIFAR-100 using the softmax output and the Active Learning strategy CoreSet}
    \label{fig:CALMSCIFAR100SoftmaxCoreSet}
\end{figure}

\subsubsection{VAAL and A-GEM}
\label{sec:Appendix:CALMS:VAALAGEM}
In the following we list the full results for the experiments using \gls{vaal} and \gls{a-gem} on the dataset MNIST, CIFAR-10 and CIFAR-100.
\begin{figure}[!htb]
    \centering
    \includegraphics[width=0.5\linewidth]{images/results_CALMS/cifar_vaal_agem.png}
    \caption{Agreement Comparison for Model Stealing on CIFAR-10 using \gls{vaal} and \gls{a-gem}}
    \label{fig:CALMScifarVAAL_AGEM}
\end{figure}

\begin{figure}[!htb]
    \centering
    \includegraphics[width=0.5\linewidth]{images/results_CALMS/cifar100_vaal_agem.png}
    \caption{Agreement Comparison for Model Stealing on CIFAR-100 using \gls{vaal} and \gls{a-gem}}
    \label{fig:CALMScifar100VAAL_AGEM}
\end{figure}

\begin{figure}[!htb]
    \centering
    \includegraphics[width=0.5\linewidth]{images/results_CALMS/mnist_vaal_agem.png}
    \caption{Agreement Comparison for Model Stealing on MNIST using \gls{vaal} and \gls{a-gem}}
    \label{fig:CALMSmnistVAAL_AGEM}
\end{figure}

\end{document}
