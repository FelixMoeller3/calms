%% LaTeX2e class for student theses
%% sections/methodology.tex
%% 
%% Karlsruhe Institute of Technology
%% Institute for Program Structures and Data Organization
%% Chair for Software Design and Quality (SDQ)
%%
%% Dr.-Ing. Erik Burger
%% burger@kit.edu
%%
%% Version 1.3.6, 2022-09-28

\chapter{Methodology}
\label{ch:Methodolody}

%% -------------------
%% | Example content |
%% -------------------
Since our continual learning approach is, to the best of our knowledge, the first approach to combine pool-based active learning with continual
learning, we explain our approach in detail in this chapter as well as the motivation for it. We then transfer our approach to the domain of model
stealing. Because we build upon the framework of ActiveThief, we describe how our approach compares to the original approach in detail.

\section{Continual Active Learning}
\label{sec:Methodology:ContinualActiveLearning}
A major contribution of this thesis is that we combine the two learning paradigms of continual learning and active learning. To motivate the idea of
combining Continual Learning and Active Learning, we will first outline the classic Continual Learning Setting and the classic Active Learning Setting.
Next we explain common issues with these two learning paradigms and how we aim to overcome these by combining both paradigms. Finally, we will give
a detailed description of our approach. 

\subsection{The classic Continual Learning Setting}
\label{sec:Methodology:CLSetting}
%Mention that CL Setting consists of multiple tasks, task can and usually are independent
% Outline classic Continual Learning Workflow with a figure

\subsection{The classic Active Learning Setting}
\label{sec:Methodology:ALSetting}
%Mention classic pool-based Active Learning setting, what batches are, contrast the continual
% learning setting, saying that the "tasks" are not independent

\subsection{Combining Continual and Active Learning}
% Mention the issues with the classic CL and AL setting, describe how synergising the two approaches
% can help overcome these issues and describe the approach in detail

\section{Continual Active Learning for Model Stealing}
\label{sec:Analysis:SecondSection}
% Transition to model stealing and elaborate why the continual active learning approach is interesting
% in the model stealing domain. Explain how the continual active learning approach has to be modified
% to apply it to model stealing. Best use a figure similar to the figure explaining the CAL approach
% and highlight the parts that are different

\section{Third Section}
\label{sec:Analysis:ThirdSection}

\dots
%% ---------------------
%% | / Example content |
%% ---------------------