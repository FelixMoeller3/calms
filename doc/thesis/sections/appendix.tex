%% LaTeX2e class for student theses
%% sections/apendix.tex
%% 
%% Karlsruhe Institute of Technology
%% Institute for Program Structures and Data Organization
%% Chair for Software Design and Quality (SDQ)
%%
%% Dr.-Ing. Erik Burger
%% burger@kit.edu
%%
%% Version 1.3.6, 2022-09-28

\iflanguage{english}
{\chapter{Appendix}}    % english style
{\chapter{Anhang}}      % german style
\label{chap:appendix}


%% -------------------
%% | Example content |
%% -------------------
\section{Proof of Data points trained using Active Learning}
\label{sec:appendix:FirstSection}
\begin{theorem}
Let $X$ be a set of data points of size $n$, $b < n, b \mod n = 0$ be the batch size. Then a machine learning model trained
using pool-based Active Learning is trained $\frac{n}{b}$ times on the current labeled pool of data points. Overall, the model
is trained on $\frac{n(n+b)}{2b}$ data points
\end{theorem}
\begin{proof}
    The first part is trivial, given that the model is trained until the labeled pool is exhausted and in each iteration $b$
    points are queried for their label. To determine the total number of points used, we need to sum up the number of points
    used in each iteration. 
    \begin{equation}
        \sum_{i=1}^{\frac{n}{b}} i \cdot b
    \end{equation}
    Using the formula for the sum of the first $n$ natural numbers, we get
    \begin{equation}
        \sum_{i=1}^{\frac{n}{b}} i \cdot b = b \cdot \sum_{i=1}^{\frac{n}{b}} i = b \cdot \frac{\frac{n}{b} (\frac{n}{b} + 1)}{2}
        = \frac{n (\frac{n}{b} + 1)}{2} = \frac{n(n+b)}{2b}
    \end{equation}
\end{proof}
% \setcounter{figure}{0}
		
% \begin{figure} [ht]
%   \centering
%   \includegraphics[width=.5\linewidth]{logos/kitlogo_en_cmyk}
%   \caption{A figure}
%   \label{fig:anotherfigure}
% \end{figure}
\section{Specifications}
\label{sec:Appendix:Specifications}

\begin{table}
    \begin{tabular}{c | c c c} 
        \hline
         & GPU x4 & GPU x8 & GPU x4 A100 \\ 
        \hline 
        Processors & Intel Xeon Gold 6230 & Intel Xeon Gold 6248 & Intel Xeon Platinum 8358  \\ 
        Number of sockets & 2 & 2 & 2  \\ 
        Processor frequency (GHz) & 2.1 & 2.6 & 2.5  \\ 
        Total number of cores & 40 & 40 & 64  \\ 
        Main memory & 384 GB & 768 GB & 512 GB  \\ 
        Local SSD & 3.2 TB NVMe & 15TB NVMe & 6.4 TB NVMe  \\ 
        GPUs & 4x NVIDIA Tesla V100 & 8x NVIDIA Tesla V100 & 4x NVIDIA A100  \\ 
        GPU Memory & 32 GB  & 32 GB & 80 GB  \\ 
        Interconnect & IB HDR & IB HDR & IB HDR200  \\ 
        \hline
    \end{tabular}
    \caption{Hardware configuration for the three nodes used on BWUniCluster. This table only contains
    the nodes we used in our experiments. For a more detailed list of all nodes as well as all further
    information, see \cite{bwUniclusterHardware}}
    \label{fig:HardwareSpec}
\end{table}

\begin{table}
    \centering
    \begin{tabular}{c | c l } 
        \hline
        Library Name & Version & Link \\ 
        \hline 
        numpy & 1.23.0 & \url{https://numpy.org/} \\
        tqdm & 4.64.1 & \url{https://tqdm.github.io/}  \\
        torchvision & 0.14.1 & \url{https://pytorch.org/vision/stable/index.html} \\ 
        torch & 1.13.1 & \url{https://pytorch.org/} \\
        PyYAML & 6.0 & \url{https://pyyaml.org/} \\
        scipy & 1.10.1 & \url{https://scipy.org/} \\
        scikit-learn & 1.2.1 & \url{https://scikit-learn.org/stable/index.html} \\
        wget & 3.2 & \url{https://pypi.org/project/wget/} \\
        \hline
    \end{tabular}
    \caption{Software Libraries used for the experiments}
    \label{fig:Libraries}
\end{table}

\begin{table}
    \begin{tabular}{c | c c } 
        \hline
        Hyperparameter & Description & Value \\ 
        \hline 
        $\lambda$ & Balances between old and new tasks. A higher value indicates more focus
        on preserving knowledge from the old task & 1.0  \\ 
        Sample size & Relative size of the sample compared to the full training set that is used to 
        compute the Fisher Matrix & 0.05  \\ 
        Gradient Clip & Maximum Value of the $l2$-Norm of the gradient. If the current norm is larger, the
        gradient is clipped to that value & 20.0 \\ 
        \hline
    \end{tabular}
    \caption{Hyperparameter configuration for Elastic Weight Consolidation.}
    \label{fig:EWCparams}
\end{table}

\begin{table}
    \begin{tabular}{c | c c } 
        \hline
        Hyperparameter & Description & Value \\ 
        \hline 
        $\lambda$ & Balances between old and new tasks. A higher value indicates more focus
        on preserving knowledge from the old task & 1.0  \\ 
        Gradient Clip & Maximum Value of the $l2$-Norm of the gradient. If the current norm is larger, the
        gradient is clipped to that value & 2.0 \\ 
        \hline
    \end{tabular}
    \caption{Hyperparameter configuration for Memory Aware Synapses.}
    \label{fig:MASparams}
\end{table}

\begin{table}
    \begin{tabular}{c | c c } 
        \hline
        Hyperparameter & Description & Value \\ 
        \hline 
        $\lambda$ & Balances between old and new tasks. A higher value indicates more focus
        on preserving knowledge from the old task & 1.0  \\ 
        Gradient Clip & Maximum Value of the $l2$-Norm of the gradient. If the current norm is larger, the
        gradient is clipped to that value & 20.0 \\ 
        $\alpha$ & Weights the importance of the previous tasks. The sum of all $\alpha$ values must be 1.0 & 0.45,0.55 \\
        \hline
    \end{tabular}
    \caption{Hyperparameter configuration for Incremental Moment Matching.}
    \label{fig:IMMparams}
\end{table}

\begin{table}
    \centering
    \begin{tabularx}{\textwidth}{X | X X } 
        \hline
        Hyperparameter & Description & Value \\ 
        \hline 
        $a$ & Balances between old and new tasks. A higher value indicates more focus
        on preserving knowledge from the old task & 1.0  \\ 
        $a'$ & Used to update parameter importances in Equation TODO & 0.5  \\
        Gradient Clip & Maximum Value of the $l2$-Norm of the gradient. If the current norm is larger, the
        gradient is clipped to that value & 2.0 \\ 
        $c$ & Determines the overestimation of the loss on the unobserved side & 3.0 \\
        $c'$ & Used to update parameter importances in Equation TODO & 1.5 \\
        \hline
    \end{tabularx}
    \caption{Hyperparameter configuration for Alasso.}
    \label{fig:AlassoParams}
\end{table}

% TODO: Finish this table.
% Add preprocessing and size of train and test split.
\begin{table}
    \centering
    \begin{tabularx}{\textwidth}{c | c c } 
        \hline
         FashionMNIST & CIFAR-10 & TinyImageNet \\ 
        \hline 
        $a$ & Balances between old and new tasks. A higher value indicates more focus
        on preserving knowledge from the old task & 1.0  \\ 
        $a'$ & Used to update parameter importances in Equation TODO & 0.5  \\
        Gradient Clip & Maximum Value of the $l2$-Norm of the gradient. If the current norm is larger, the
        gradient is clipped to that value & 2.0 \\ 
        $c$ & Determines the overestimation of the loss on the unobserved side & 3.0 \\
        $c'$ & Used to update parameter importances in Equation TODO & 1.5 \\
        \hline
    \end{tabularx}
    \caption{Dataset Information}
    \label{fig:DatasetInformtion}
\end{table}

% TODO: Finish this table.
% Parameters, source, etc.
\begin{table}
    \centering
    \begin{tabularx}{\textwidth}{c | c c } 
        \hline
         FashionMNIST & CIFAR-10 & TinyImageNet \\ 
        \hline 
        $a$ & Balances between old and new tasks. A higher value indicates more focus
        on preserving knowledge from the old task & 1.0  \\ 
        $a'$ & Used to update parameter importances in Equation TODO & 0.5  \\
        Gradient Clip & Maximum Value of the $l2$-Norm of the gradient. If the current norm is larger, the
        gradient is clipped to that value & 2.0 \\ 
        $c$ & Determines the overestimation of the loss on the unobserved side & 3.0 \\
        $c'$ & Used to update parameter importances in Equation TODO & 1.5 \\
        \hline
    \end{tabularx}
    \caption{Neural Network information}
    \label{fig:NNArchitectures}
\end{table}


\begin{figure}[ht]
    \centering
    \includegraphics[width=\linewidth]{images/results_CALMS/cifar10_model_comp.png}
    \caption[Accuracy Progression for Model Stealing on CIFAR-10 using different ActiveThief models]{TODO: Nice text here}
    \label{fig:CIFAR10modelComp}
\end{figure}

\begin{figure}[ht]
    \centering
    \includegraphics[width=\linewidth]{images/results_CALMS/mnist_model_comp.png}
    \caption[Accuracy Progression for Model Stealing on MNIST using different ActiveThief models]{TODO: Nice text here}
    \label{fig:MNISTmodelComp}
\end{figure}