\chapter*{Notation}
\label{ch:notation}

% Mention b as batch size
%TODO: Review the text here
This chapter is intended as a quick overview of the mathematical notation used in this work.
The symbols largely follow contemporary works, while simultaneously trying to avoid ambiguity.
Furthermore, the notation refrains from assigning semantics to different font weights, in order to stay consistent across multiple media.

\begin{tabularx}{\textwidth}{l X}
    \toprule
    Symbol & Definition \\
    \midrule
    $x_i$ & The $i$-th data point in a dataset. \\ \addlinespace
    $y_i$ & The label of the given data point $x_i$, also known as $y(x_i)$. \\ \addlinespace
    $X$ & The set of data points in a dataset. \\ \addlinespace
    $Y$ & The set of labels in a dataset. \\ \addlinespace
    $b$ & The number of samples within a batch in Active Learning \\ \addlinespace
    %$T$ & Number of ... \\ \addlinespace
    $U$ & The unlabeled pool in pool-based active learning\\ \addlinespace
    $L$ & The labeled pool in pool-based active learning\\ \addlinespace
    $\mathcal{L}$ & The loss function \\ \addlinespace
    $P$ & Patterns per experience, a hyperparameter of the \gls{a-gem} algorithm \\ \addlinespace
    $S$ & Samples drawn from memory to compute the reference gradients in \gls{a-gem} \\ \addlinespace
    \bottomrule
\end{tabularx}
\label{tab:notation}