%% LaTeX2e class for student theses
%% sections/evaluation.tex
%% 
%% Karlsruhe Institute of Technology
%% Institute for Program Structures and Data Organization
%% Chair for Software Design and Quality (SDQ)
%%
%% Dr.-Ing. Erik Burger
%% burger@kit.edu
%%
%% Version 1.3.6, 2022-09-28

\chapter{Related Work}
\label{ch:Related_work}
Related work for this thesis can be grouped into three different categories:
The first category is \href{sec:Related_work:Active_Learning}{Active Learning}.
Active Learning is a special form of machine learning where an oracle is present
which can label arbitrary data points. A Machine Learning model trained using Active
Learning iteratively queries the oracle with unlabeled data points, trains a new model
and determines which data should be queried next. The second category is
\href{sec:Related_work:Continual_Learning}{Continual Learning}. Continual learning is a
machine learning technique which aims to make a given machine learning model learn new tasks
without forgetting the knowledge of previous tasks. 
The third category is \href{sec:Related_work:Model_Stealing}{Model Stealing}. Model Stealing is the
processing of strategically querying a third-party machine learning model (also referred to as target model)
to train a local model (also referred to as substitute model) which is supposed to approximate the target model as
good as possible.

% use Settles as reference for the general introduction
\section{Active Learning}
\label{sec:Related_work:Active_Learning}
\subsection{General Introduction}
Active Learning is a specific form of machine learning where the learner can query an oracle to label arbitrary data points.
The motivation behind Active Learning is that nowadays it is not difficult to obtain large amounts of data, but the bottleneck
is assigning labels to them. Active Learning aims to overcome this issue producing a highly accurate model with little amount
 of labeled data. The idea is that a learner which chooses the data it is trained on should perform better or at least as good
 as a model trained on all available data.
\subsection{Active Learning Approaches used in the Experiments}
\subsubsection{Least Confidence}
\subsubsection{CoreSet}
\subsubsection{BALD}
\subsubsection{Badge}
% Structure: First some general introduction then present
% the different approaches used in the experiments (LC, CoreSet, BALD, Badge)

% Use parisi et al. as reference for the general introduction
% ALASSO paper also give good introduction into approaches and structures them
% into three categories
\section{Continual Learning}
\label{sec:Related_work:Continual_Learning}

\subsection{General Introduction}
\subsection{Continual Learning Approaches used in the Experiments}
\subsubsection{EWC}
\subsubsection{MAS}
\subsubsection{ALASSO}
\subsubsection{IMM}

% Structure: First some general introduction then present
% the different approaches used in the experiments (EWC, MAS, ALASSO, IMM, potentially Replay?)


\section{Model Stealing}
\label{sec:Related_work:Model_Stealing}

\dots
%% ---------------------
%% | / Example content |
%% ---------------------