%% LaTeX2e class for student theses
%% sections/conclusion.tex
%% 
%% Karlsruhe Institute of Technology
%% Institute for Program Structures and Data Organization
%% Chair for Software Design and Quality (SDQ)
%%
%% Dr.-Ing. Erik Burger
%% burger@kit.edu
%%
%% Version 1.3.6, 2022-09-28

\chapter{Conclusion}
\label{ch:Conclusion}
% Mention drops in accuracy due to Continual Active Learning. Regarding model stealing
% mention that there are cases where the CAL works a lot better than for normal continual
% active learning. Mention correlation with batch size to support the thesis that CAL fails
% because batch size is not large enough. Mention small difference between different active
% learning approaches for CIFAR-10 at least, which are inline with the findings from Yilins
% paper.
% Don't use data augmentation for model stealing!!!
In this thesis we proposed a novel approach called Continual Active Learning, which combines Active Learning and Continual Learning. 
We implement multiple Active Learning and Continual Learning strategies, namely BALD, LC, Badge and CoreSet as well as EWC, IMM, Alasso and MAS,
and combine them to create a Continual Active Learning strategies. Next, we evaluate them on the dataset CIFAR-10 and the Neural Network Architecture
ResNet18. Our results indicate that Continual Active Learning strategies have significantly lower runtime than Active Learning strategies, but fail
to achieve their accuracy. We decide to implement the Active Learning strategy VAAL along with the Continual Learning strategy A-GEM to see if they
can outperform the other Continual Active Learning strategies but receive similar results. Our analysis of the results suggests that the main reason
why Continual Active Learning achieves sub-par validation accuracy is that the number of training points used in each Active Learning iteration is too
small. Further analysis of the results from our experiments with a custom Replay strategy support this theory. \par
Before applying Continual Active Learning to the Model Stealing domain, we rigorously analyze the setting of the ActiveThief paper \cite{pal2020activethief}.
While we agree with most findings of the paper, we find that the architecture of the target and the substitute model has a greater influence of the Model Agree-
ment achieved than the Authors of ActiveThief suggest. Furthermore, we find that the choice of the thief dataset has a significant influence on Model Agreement
and so does (not) using data augmentation on the thief dataset. \par
Finally, we perform Model Stealing Attacks with Continual Active Learning, using MNIST, CIFAR-10 and CIFAR-100 as target model datasets. The results from these
experiments closely resemble those from the classic Continual Active Learning setting, i.e. we find that Continual Active Learning achieves lower Model Agreement
than Model Extraction using Active Learning. \par
In general, we find that our approach is (resource-) \textit{efficient} but not \textit{effective} because it is significantly outperformed by the baseline,
both in the classic Continual Active Learning setting and in the Model Stealing setting, in terms of accuracy and agreement, respectively. However, it is important
to note that the validity of our results is limited. First, the list of Continual and Active Learning strategies we evaluated is not exhaustive, although we made
every effort to include the most popular strategies. Second we only evaluated our approach on a single Neural Network Architecture and a single dataset in the 
classic Continual Active Learning setting and one Neural Network plus three datasets in the Model Stealing setting.
\section{Future Work}
\label{sec:Conclusion:FutureWork}
% Future work could be to use the same approach in other types of continual learning settings,
% i.e. task-incremental settings, or class-incremental settings 
% Another approach is to use data distillation for replay to minimize the data overhead of active learning
% One could also use different Active Learning and Continual Learning approaches to see if the results are similar
% Another way is to combine the Active Learning and Continual Learning approaches, i.e. to combine the gradients used
% in AGEM for the selection by gradients using Badge
% Look if CAL is relevant to any security problem like e.g. malware analysis
The approach we proposed in this thesis is not limited to the settings we evaluated them in. While the most straight-forward extension would be to evaluate the approach
on new datasets and Neural Network Architectures in the settings we used them in, we believe that, from a research perspective, it would be more interesting to apply 
Continual Active Learning to unrelated settings. In our methodology, we had an Active-Learning-centric view on Continual Active Learning, i.e. our major goal was to
improve the accuracy of Active Learning while reducing its runtime. Future work might be interested in approaching Continual Active Learning from a Continual Learning
perspective, for example by using Active Learning to perform task ordering in a task-incremental setting. Another idea could be to leverage the information of Active
Learning for Continual Learning and vice-versa. In our current framework Active Learning and Continual Learning work independently, and we see potential for improve-
ment by introducing interdependence between the two. \par
Broadly speaking, our approach can be used in any setting where the data distribution of the task changes and labeling data is expensive. In the computer security domain,
our approach could be used for malware analysis. Previous works have demonstrated the effectiveness of using Machine Learning for Malware Analysis \cite{nath2014static}
\cite{ijaz2019static}. Malware analysis is a task that is well suited for our approach because the features that classify malware change over time and determining if a
given piece of software contains malware is time-consuming. 