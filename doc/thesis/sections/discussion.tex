%% LaTeX2e class for student theses
%% sections/conclusion.tex
%% 
%% Karlsruhe Institute of Technology
%% Institute for Program Structures and Data Organization
%% Chair for Software Design and Quality (SDQ)
%%
%% Dr.-Ing. Erik Burger
%% burger@kit.edu
%%
%% Version 1.3.6, 2022-09-28

\chapter{Discussion}
\label{ch:Discussion}
% Was war das Ziel? Was wurde (nicht) erreicht?
In this chapter, we will discuss our findings from the experiments in chapter \ref{ch:Evaluation}. Like most chapters in this thesis, we will divide this chapter into two parts, one for continual active learning and one for model stealing. 
The first section will discuss the findings from the first batch of experiments, where we tested Continual Active Learning on the CIFAR-10 dataset using Resnet18. The section on Model Stealing is then used to discuss general findings from
our experiments on Model Stealing as well as findings from applying Continual Active Learning to Model Stealing. 

\section{Continual Active Learning}
\label{sec:Discussion:ContinualActiveLearning}
The goal of this section is to discuss the findings from section \ref{sec:Evaluation:Results:CAL}. When developing the Continual Active Learning approach proposed in this thesis, our initial aim was to improve both the overall performance, measured
by validation accuracy, of the model and the training time in comparison to Active Learning. By studying the results we observe that we were not able to outperform Active Learning in any of the experiments. On the contrary, Active Learning, even naive
Active Learning using random sampling, significantly outperforms Continual Active Learning in terms of validation accuracy. However, we managed to achieve the second goal of improving the training time. Before we analyze the reasons for the poor performance 
of Continual Active Learning in terms of validation accuracy, we will first discuss the performance in terms of execution time.

\subsection{Execution time}
\label{sec:Discussion:ExecutionTime}

In terms of training time, we observe that Continual Active Learning outperforms Active Learning in all experiments. The main reason for the superior performance of Continual Active Learning runtime-wise is that the model is trained on significantly less data
in the training process. In section \ref{sec:Methodology:CombiningCLandAL} we mentioned that a model trained with a total budget of $n$ and a batch size of $b$ is trained on $\frac{n(n+b)}{2b}$ data points. When we use this equation with the parameters given in
our experiments, namely $n=50000$ and $b \in \{ 1000, 2000, 4000\}$, we see that the total number of points trained increases almost proportionally with a decreasing batch size. This phenomenon is shown in table \ref{fig:NumberOfTrainingPoints}, where we computed
the total number of training points for the batch sizes used in the experiments. In our case, we train on almost four times as many data points when using a batch size of 1000 compared to a batch size of 4000. Since the total number of training points is constant
for Continual Active Learning with varying batch sizes, we do not observe the same phenomenon. \par

\begin{table}[h]
    \centering
    \begin{tabular}{| c | c |} 
        \hline
        $b$ & Total number of training points \\
        \hline 
        1000 & 1275000 \\
        2000 & 650000 \\
        4000 & 362000 \\
        \hline
    \end{tabular}
    \caption[Total number of data points trained on in the Continual Active Learning experiments]{Total number of training points for varying batch sizes on the CIFAR-10 dataset, which has a train set of consisting of 50000 elements.}
    \label{fig:NumberOfTrainingPoints}
\end{table}

The total number of training points perfectly explains the execution time of the Active Learning strategies Random and LC. In the experiments, the execution time of Random is at approximately 2500 Minutes for batch size 1000, 1300 for batch size 2000 and 700 for
batch size 4000 with LC showing comparable results. On the other hand the Continual Active Learning strategies using Random and LC show similar runtime for batch size 2000 and 4000 but the execution time for batch size 1000 is merely half the execution time when
using the previous batch sizes. This discrepancy in execution time can be explained by the number of epochs trained in each iteration of Active Learning. For batch size 1000, we train for 80 epochs per iteration while we train for 150 epochs per iteration for
batch size 2000 and 4000. \par
While we have analyzed how the runtime of LC and Random can be explained we still need to analyze the runtime of Badge, CoreSet and BALD. LC and Random have a very low query time, which is negligible with a dataset of the size of CIFAR-10 and a model like ResNet18.
In this specific setting BALD exhibits a runtime which is almost identical to LC and Random. This is because ResNet18 does not contain any dropout layers. Since our implementation of BALD dynamically detects whether the model contains dropout layers and only performs
Monte Carlo dropout if it does, we only run one dropout iteration in this setting. In theory, we could run multiple iterations of Monte Carlo dropout, however this increases query time, and runtime for that matter, without improving the performance of BALD. Using only
one dropout iteration in our setting is sufficient since the predictions of a Neural Network without activated dropout layers are deterministic. In general however, the query time of BALD depends on the number of dropout iterations, which is a hyperparameter, the size
of the unlabeled pool and the prediction speed of the model. \par
CoreSet and Badge are Active Learning strategies which incorporate diversity into their sampling. With CoreSet being entirely diversity-based and Badge being both diversity-based and uncertainty-based, these strategies compute a cover of the data points in the pool using
a k-center greedy algorithm and $k$-Means++ respectively. Both strategies use distance computation in their sampling which becomes more expensive computationally with an increasing number of data points and a larger feature space. This explains the higher query time of
CoreSet and Badge compared to LC, BALD and Random. Furthermore, it is worth noting that BALD and LC are rank-based, meaning that they always compute their respective importance measure for the complete unlabeled pool and then return the $b$ most informative data points
according to their importance measure. Therefore, a query with batch size $b_1$ while always be a subset of a query with batch size $b_2$ if $b_1 < b_2$. CoreSet and Badge on the other hand are not rank-based, meaning that they compute a cover of a feature space. The
optimal cover depends on the batch size. It is therefore theoretically possible that a query with batch size $b_1$ and a query with batch size $b_2$ have no data points in common, if $b_1 \neq b_2$. For CoreSet, and even more so for Badge, the query time becomes a
bottleneck in our experiments. While the total query time in our experiment is not dependent on the batch size, because we always query the complete train set over the course of the experiment, it is still a considerable portion of the total runtime. The main reason why
Continual Active Learning using Badge or CoreSet do not outperform the respective baseline by a factor as large as the one of Continual Active Learning straegies using BALD, LC or Random is that with increasing batch size the training time becomes a smaller fraction of
the execution time while the query time remains constant. \par
Next, we investigate the contribution of the Continual Learning strategies in terms of runtime. Overall, we observe that the overhead introduced by the Continual Learning strategies is negligible compared to the overhead introduced by the Active Learning strategies. The
biggest increase in runtime is caused by Alasso, which increases the execution time by 30 to 60 Minutes in our experiments. The reason for the increase in runtime is most likely the backpropagation step. Alasso needs two backward passes through the model to compute the
complete gradients: The first is used to compute the unregularized gradients which are needed for the computation of $\Omega$ in Equation \ref{eq:ALASSO_Omega} and the second is used to compute the gradients for the surrogate loss. MAS, IMM and EWC only need one backward
pass to compute the gradients, just as the Naive approach, because they do not require saving the unregularized gradients. \par
Apart from the overhead introduced by backpropagation, the Continual Learning strategies introduce overhead by estimating weights for the parameters of the Neural Network at the end of each task. Alasso computes its weights by using equation \ref{eq:ALASSO_Omega}, MAS
uses the gradients of the validation step whereas EWC and IMM compute the diagonals of the Fisher Information Matrix. Overall, this overhead is minimal however, as can be seen from the execution times presented in section \ref{sec:Evaluation:Results:CAL}. Surprisingly,
IMM and EWC demonstrate minimally lower runtime than the Naive approach in a few experiments. While this is not possible in theory because the Naive approach does not compute any importance weights, we believe that the superior performance of IMM and EWC when using LC,
IMM and BALD can be explained by factors that are beyond our control, such as operating system overhead or disk access speed. At this stage, we would like to point out that we do not intend to give a perfect ranking of the Continual Learning strategies in terms of
execution time, but we rather aim to give a rough estimation of the speedup introduced by using Continual Active Learning over Active Learning. For CoreSet and Badge, the difference in runtime between Naive, EWC and IMM can be explained by the fact that the query time
is not deterministic for these Active Learning strategies, because computing their query results requires computing a set cover like in CoreSet or a $k$-Means++ initialization like in Badge. \par
% Further explanations in terms of runtime: BALD and LC are rank based, i.e. their query time does not depend on the batch size. CoreSet and Badge are batch size dependent since they compute a cover of the data points and more data points means more points in the cover.
% For continual learning strategies mention that one overhead is the computation time during gradient computation which is needed for backpropagation and for some strategies the importance weights need to be computed.



% In terms of accuracy: mention that CAL does not work as expected because the CL strategies were not designed for the scenario which we are using them in, i.e. training on the same dataset. Mention that batch size plays a major role as can be seen from the experiments.
\subsection{Validation Accuracy}
\label{sec:Discussion:ValidationAccuracy}
After analyzing the results in terms of execution time, we will now focus on the validation accuracy. As mentioned before, the Continual Active Learning strategies are significantly outperformed by the Active Learning strategies in terms of validation accuracy. Before we
analyze the difference in performance between Active Learning and Continual Active Learning, we will first investigate them on their own. \par
When investigating the validation accuracy curves of the Active Learning strategies we see that, apart from random sampling, they allow follow the same pattern: First, the validation accuracy increases rapidly until the size of the labeled pool has reached about 20\% of
the full labeled set. When the labeled pool comprises 20\% to 50 \% of the full labeled set, the validation accuracy increases moderately until it reaches its peak at 50\% of the labeled set. Between 50\% and 100\% of the labeled set, the validation accuracy
decreases slightly until it reaches the level it achieved with 20\% of the full training set. Before we analyze why this behavior occurs, we point out that the batch size has a negligible effect on the validation accuracy progression in our experiments, which is in line
with the findings of Beck et al. \cite{beck2021effective}. The validation accuracy curve of Active Learning can be explained as follows: First, the model is trained on the most informative samples which leads to a rapid increase in validation accuracy. However, the larger
the size of the labeled pool, the less informative samples exist and the less informative are even the most informative samples of the unlabeled pool. This is the reason for the moderate increase in validation accuracy between 20\% and 50\% of the labeled pool. At the point
where the labeled pool comprises 50\% of the full labeled set, the model has seen all the informative samples and the validation accuracy starts to decrease as the newly labeled samples are uninformative. Nevertheless, the validation accuracy does not experience a sharp decline
because the labeled pool still contains the informative samples which where labeled in the beginning. \par
The validation accuracy curves of the Continual Learning strategies are more diverse than the ones of the Active Learning strategies. When analyzing these validation accuracy curves, it is important to keep in mind that a single Continual Active Learning strategy consists of a
combination of an Active Learning strategy and a Continual Learning strategy, and therefore we need to analyze the contribution of both the Active Learning strategy and the Continual Learning strategy to the validation accuracy. Since we used more than 25 combinations of Continual
and Active Learning strategies in our experiments, we will not be able to analyze each combination in detail. This is not necessary however, because the contribution of Active Learning and Continual Learning to the validation accuracy can be isolated mostly. \par
First, we will analyze the Contribution of the Continual Learning strategies on the validation accuracy of Continual Learning. We start off with Alasso, which is clearly demonstrates the most inferior performance of all Continual Learning strategies. A major problem of Alasso is
its erratic behavior. While Alasso does not perform significantly worse than the other Continual Learning strategies when the accuracy curves are averaged, it has by far the highest variance in its validation accuracy. Furthermore, the previous statement only holds for experiments
using a batch size of 1000. For batch size 2000 and 4000, Alasso becomes practically unusable after about 10000 samples because it shows a steady decline in validation accuracy. We believe that this is due to the fact that the overestimation of the loss on a single side of the 
loss function creates an uneven loss function. The problem is that overestimation is done on a per-parameter basis, meaning that the loss could be overestimated for one parameter and correctly estimated for another. The exploding gradient problem we observe in our experiments
supports the thesis that parameter optimization using gradient descent is hard when the loss is estimated as it is done with Alasso. While using gradient clipping did mitigate the exploding gradient problem, it did not solve it completely. When observing the loss progression for
Alasso, we notice that the loss dramatically increases for the first epoch of each Active Learning iteration. Subsequently, the loss decreases but is unable to recover from the gradient shift performed in the first epoch. We tried clipping the gradient at even smaller values of
the $l_2$-Norm to eradicate the loss increase in the first epoch, however this impeded model convergence. \par
Next, we will analyze IMM and EWC jointly. The reason for the joint analysis is that they perform on a comparable level and they both estimate parameter importance similarly. Overall, IMM and EWC demonstrate a validation accuracy that is on par with the Naive approach. We believe
that this is because they employ regularization less stringently than Alasso and MAS. This is supported by the experiments where the validation accuracy for Naive, IMM and EWC drops of similarly while MAS manages to maintain its validation accuracy. Our experiment with the delayed
start of Continual Active Learning in Figure \ref{fig:Evaluation:Results:CAL:DelayedStart} provides further evidence for this thesis. \par
Third, we investigate the behavior of MAS. MAS performs significantly better than Alasso in terms of validation accuracy but is outperformed by IMM, EWC and Naive. We believe that this is caused by the fact that MAS restricts parameter updates more than EWC and IMM. When observing
the validation accuracy progression of MAS, we note that MAS does not exhibit the steady increase in validation accuracy that EWC, IMM and Naive do, but it also does not experience the sharp decline in validation accuracy within the final 10000 samples. While it is desirable that
MAS effectively manages to preserve previously learn knowledge, we argue that it lacks the ability to adapt to new tasks. \par
Finally, we will analyze the Naive approach, which also enables us to bridge between the contribution of Continual Learning and Active Learning to the validation accuracy. The Naive approach performs significantly better than Alasso and MAS in the first 40000 samples, which is due
to the fact that it does not restrict parameter updates at all. Since the accuracy of the Naive approach increases for the first 25000 samples approximately, we assume that these samples are the ones that are informative enough to further make the model further learn the task it is
trained for. After about 20000-30000 samples the accuracy of the Naive approach starts to decline, which is caused by the fact that the samples in this period are less informative than before, although they are still informative enough to impede forgetting. The final 5000 samples how-
ever introduce a significant reduction in validation accuracy which is interesting because it demonstrates that a task can be unlearned to some extent when using data that belongs to the very same task. \par
Moving on to the contribution of Active Learning to the validation accuracy, we first analyze the performance of the Active Learning strategy LC. Continual Active Learning strategies using LC generally follow the pattern described in the previous section. To add to the previous
section, we believe that for LC the drop in validation accuracy within the final 10000 samples of the experiments is caused by an uneven class distribution in the final queries. We observed the class distribution for LC over multiple iterations and noticed that it became increasingly
inhomogeneous over time. \par
Next, we investigate the contribution of BALD to the validation accuracy of Continual Active Learning. We note that there is a large variance in the validation accuracy, especially when using a batch size of 1000. With larger batch sizes, the variance in validation accuracy decreases
but BALD is still not able to demonstrate comparable performance with other Active Learning strategies. This is due to the special circumstances of the experiment setup. Since we use ResNet18, which does not contain dropout layers, we are unable to run Monte Carlo dropout to accurately
estimate $\mathbb{E}_{\theta \sim p(\theta \mid L)} [H[y \mid x, \theta]]$. \par
Regarding Badge and CoreSet, we notice that their show matching performance, similar to EWC and IMM in the Continual Learning setting. Badge and CoreSet follow the typical Continual Active Learning curve, described in the paragraph for the Naive strategy, the most. This is due to the
fact that they incorporate diversity into their sampling strategies which smooths the validation accuracy curve except for the final 5000 samples and experiments with batch size 10000. We believe that the final 5000 samples generally do not represent the task the model is trained for
well and therefore even a subset that is representative of these samples cannot be used to improve the validation accuracy. For the setting with 1000 batch size, the issue is rather that these 1000 points are not enough to represent the unlabeled pool well. \par
Finally, we briefly discuss the performance of VAAL and A-GEM. VAAL as an Active Learning strategy by itself is outperformed by the remaining Active Learning strategies which demonstrates that is unable to select the most informative samples to query. However, this is an advantage when
using VAAL in conjunction with A-GEM. In our experiment with LC and A-GEM, we noticed that A-GEM outperformed LC in terms of validation accuracy within the first 15000 samples but suffered from a heavy drop in validation accuracy thereafter. It seems that A-GEM struggles to retain the
knowledge of previously learned samples when their informativeness changes rapidly. Ironically, the fact that VAAL is not as performant as the other Active Learning strategies helps A-GEM increase its validation accuracy when combined with VAAL. \par

We close this section by aiming to answer the most important question: Why does Continual Active Learning not outperform Active Learning? We believe that this is mainly due to the number of training points being used in each iteration. While Active Learning strategies use the
complete labeled pool in each iteration ($i \cdot b$ data points where $i$ is the number of iterations), Continual Active Learning strategies only use the data points that have been labeled in the current task, i.e. $b$ points. It is worth pointing out that the correlation between
data points trained on, and validation accuracy achieved is not perfect by any means. However, especially if the number of data points trained on is less than 20\% of the full size of the training set, just one or two percentage points can make a big difference in terms of validation
accuracy. This is supported by the observation that the validation accuracy increases with an increasing batch size. Our Replay strategy further demonstrates this behavior. For varying sizes of the replay buffer and the batch size, we observe identical performance as long as the sum
of replay buffer size and batch size is identical. Analysis of the second experiment with our Replay strategy shows that for the first 25000 samples, it is irrelevant how we select the data we replay, which further supports the thesis. \par
A second component to the answer of the question is the fact that the Continual Learning strategies we used, or any Continual Learning strategy for that matter, are not designed for a setting in which all the training data belong to the same task. The Continual Learning strategies have
been developed for and evaluated on task-incremental and class-incremental learning settings.


\section{Model Stealing}
\label{sec:Discussion:ModelStealing}
In this section we will discuss the results of the experiments which involve Model Stealing, i.e. the experiments conducted in section \ref{sec:Evaluation:Results:MS:CAL}. First, we will analyze the general observations regarding Model Stealing attack which we made and compare them
to the findings of Pal et al. in ActiveThief as well as Orekondy et al. in Knockoff Nets. Next, we will elaborate on the results of our Continual Active Learning Attacks strategy for Model Stealing Attacks.

\subsection{General Model Stealing Observations}
\label{sec:Discussion:ModelStealing:General}
We make three decisive observations for Model Stealing which we will present in this subsection. The first observation is that the success of a Model Stealing Attack, which is measured by Model Agreement in our case, is highly dependent on the thief dataset. From our experiment in 
Figure \ref{fig:Evaluation:Results:CAL:EffectDataset}, we see that the Model Agreement at the end of the attack is approximately 85\% for CIFAR-10, 80\% for Small ImageNet and merely 20\% for Tiny ImageNet. While previous works such Knockoff Nets state that \enquote{any large diverse set
of images makes for a good transfer set}, we come to a different conclusion. Tiny ImageNet consists of 100000 images which should be enough to steal a model trained on a simple dataset like MNIST is. The strong performance of CIFAR-10 in this experiment contradicts the claims from
Orekondy et al. even more because CIFAR-10 is arguably less diverse than Tiny ImageNet but significantly outperforms it in this setting. \par
The second observation we make is that the success of a model stealing does depend on the architecture of both target and substitute model. In our experiments, ActiveThiefConv3 and ActiveThiefConv4 perform significantly better than ActiveThiefConv2 when being used as a target model
and on the other hand, model stealing attacks extracting ActiveThiefConv2 achieve higher Model Agreement than those extracting ActiveThiefConv3 or ActiveThiefConv4. This is in line with the findings of Orekondy et al. while it contradicts the findings of Pal et al. We do note however,
that Pal et al. most likely compared Model Agreement between Neural Network architectures when training on the full thief datasets where we compared Model Agreement after using Active Learning for Model Extraction with a total budget of 20000. We believe that our findings are more
representative because they do not assume training on a full dataset. \par
Finally, we observed that data augmentation can significantly effect the success of Model Stealing attacks. Unfortunately however, there is no general rule whether data augmentation is beneficial or not. In our experiments with MNIST as target model dataset, we observed that
renouncing data augmentation significantly increased Model Agreement whereas the opposite was the case for CIFAR-10 as target model dataset. We believe that learning to classify MNIST is a much easier task than learning to classify CIFAR-10 and therefore, using data augmentation
when extracting a model trained on MNIST only obscures the function learned by the target model. Since most real world datasets are more similar to CIFAR-10 than to MNIST, we believe that performing data augmentation on the thief dataset generally is beneficent for Model Stealing
in the real world. The effect of data augmentation proves another point however: Model Stealing is a domain where a single hyperparameter can have a significant effect on the outcome of an experiment.

\subsection{Continual Active Learning for Model Stealing}
\label{sec:Discussion:ModelStealing:CALMS}
Next, we will turn to the results of the experiments where we applied Continual Active Learning to perform Model Stealing Attacks. We will discuss the results on the three different target model datasets separately and then mention any findings that apply to all three datasets. \par
Regarding our experiment for the MNIST dataset, we notice that the gap in performance of Continual Active Learning to Active Learning is even larger than in the classic Continual Active Learning setting, both when training using the softmax output of the target model and when using 
only the predicted class label. We assume that the main reason for the poor performance of Continual Active Learning in this setting is the fact that we used data augmentation for the thief dataset. In Figure \ref{fig:Evaluation:Results:CAL:EffectAugmentation} we presented the results
of our experiments performing Model Stealing with and without data augmentation on the thief dataset. We see that for the MNIST dataset, renouncing data augmentation significantly increases Model Agreement. When using data augmentation with MNIST as target model dataset, it is more
challenging to learn the function of the target model, especially when training solely on 2000 data points at a time. \par
For the CIFAR-10 dataset, we noticed that the margin between Continual Active Learning and Active Learning was considerably lower compared to MNIST. Again, we believe that using data augmentation is the main reason for this. Figure \ref{fig:Evaluation:Results:CAL:EffectAugmentation}
shows that using data augmentation when extracting a target model trained on CIFAR-10 increases Model Agreement. The results from the experiments using softmax output and the predicted label suggest that with data augmentation, the feature space of CIFAR-10 is demanding to learn to a
full extend (hence to low Model Agreement of the baseline compared to MNIST), but simpler to learn partially. This benefits the performance of Continual Active Learning in this setting. \par
Finally, we analyze the results of our experiments using CIFAR-100 as target model dataset. In absolute terms the gap between the performance of Continual Active Learning and Active Learning is the smallest for this dataset. This is expected however because CIFAR-100 is a challenging
dataset, where even the Active Learning Attacks achieve below 30\% Model Agreement. In relative terms however, the performance is similar to the performance of Continual Active Learning on MNIST. For the experiment with softmax output of the target model, the average agreement of the
best Continual Learning strategy is about 72\% of the baseline and when training on the target model, the average agreement is a mere 40\% of the baseline. We believe that the reason for the poor performance is that the validation accuracy of the target model is already very low at 42.9\%.
Accurately learning the function of the target model is especially difficult in this setting when training using the label, because the model has not fully learned to classify the target model dataset itself. The low validation accuracy of the target model in combination which caused by
the complexity of the CIFAR-100 dataset makes training on a significantly large dataset crucial in order to achieve high Model Agreement. This is again where the fact that Continual Active Learning only uses data points from the current batch to train on is a disadvantage compared to
Active Learning. \par
Generally, we observed that BALD demonstrates significantly improved performance compared to the classic Continual Active Learning setup. The reason for this is that the substitute model in this experiment, which is ActiveThiefConv3, contains multiple dropout layers, enabling a more
accurate estimation of $\mathbb{E}_{\theta \sim p(\theta \mid L)} [H[y \mid x, \theta]]$ using Monte Carlo dropout. Although BALD shows improved performance in the Model Stealing experiments, it is outperformed by CoreSet. Since CoreSet is a diversity-based sampling strategy, this is
in line with the findings of \cite{ash2019deep}, who state that diversity sampling is more effective in the first Active Learning iterations. For our experiments we use a total budget of 20000 samples, which is less than 16\% of the total thief dataset (128116 samples), meaning that we
operate at the beginning of the Active Learning cycle for the full experiment. Surpisingly, Badge underperforms compared to the classic Continual Active Learning setup. This is most likely due to the fact that the gradient embedding of ActiveThiefConv3 is twice the size of the
gradient embedding of ResNet18 and the thief dataset is more than twice the size of CIFAR-10 (50000 samples vs 128116 samples). Having a large embedding means that distance computation is harder due to the Curse of Dimensionality \cite{koppen2000curse}. In addition to the poor performance
of Badge, we need to mention its runtime. Due to the factors mentioned previously, the embedding space is significantly larger which causes an increase in runtime. On our hardware, the Model Stealing Attacks for CIFAR-10 and MNIST which used Badge had a query time of about 4 days, which is
significantly more than all the other Active Learning strategies other Active Learning strategies. \par
Initially, we planned to compute the same set of experiments using ImageNet as our target model dataset. In the end, we decided against running the experiments because our thief dataset is a subset of ImageNet which is an unrealistic setting in the real world. Furthermore, the validation
accuracy of the target model would most likely be even lower than for CIFAR-100, making it even more challenging to evaluate the different Continual Active Learning strategies because their performance would not differ significantly.

% Am ende ein paar Sätze zu jeder CL und AL Strategie
%  For model stealing. Mention first of all that this is a setting where there is an abundance of hyperparameters and therefore: 1) it is hard to find the optimal hyperparameters and 2) it is hard to compare different approaches since they might not be using the same hyperparameters
% and approaches with different hyperparameters might yield completely different experiments. Then mention that we were not able to reproduce the findings of activethief when using different substitute and target model architectures for MNIST. Mention that this is most probably because
% we used data augmentation for the stealing process which we showed not to be helpful in the end. Regarding the experiments with MNIST, CIFAR-10 and CIFAR-100 talk about the single Active Learning and Continual Learning strategies (i.e. say that for CIFAR-100, CoreSet does not work well
% as expected).