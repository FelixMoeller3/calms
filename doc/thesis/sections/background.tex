%% LaTeX2e class for student theses
%% sections/content.tex
%% 
%% Karlsruhe Institute of Technology
%% Institute for Program Structures and Data Organization
%% Chair for Software Design and Quality (SDQ)
%%
%% Dr.-Ing. Erik Burger
%% burger@kit.edu
%%
%% Version 1.3.6, 2022-09-28

\chapter{Background}
\label{ch:Background}

In this chapter we will give an overview of the background knowledge that is needed to understand the following chapters. The topic
of this thesis is at the intersection of the fields of Active Learning, Continual Learning and Model Stealing and therefore we will
give an introduction to each of these fields. 

\section{Active Learning}
\label{sec:ActiveLearning}

%TODO: Hier learner-Beschreibung anpassen
Active Learning is a subfield of Machine Learning that focuses on the problem of how to select the most informative data points to label.
Research in Active Learning is motivated by the fact that data acquisition is often easy because it can be automated whereas labeling is
difficult and time-consuming. Therefore, the goal of research in Active Learning is to develop methods which maximize model performance
with minimal amount of labeled data. A typical Active Learning scenario comprises a learner, an oracle, unlabeled data and labeled data.
The learner is the actual algorithm which selects which data points to label by the oracle. Let $I$ be the instance space (i.e. the set
of all possible data points) and $L$ be the label space (i.e. the set of all possible labels). The oracle $O$ can be seen as the function
\begin{equation}
    O: I \rightarrow L, x \mapsto y(x)
\end{equation}
where $y(x)$ is the true label of the data point $x$, often just referred to as $y$. The unlabeled data $U$ is a subset of $I$ just as the set
of labeled data $L$. At the beginning there are no labeled data points. Often a few labeled data points are randomly sampled from $U$ and labeled
by the oracle to initialize the learning process. A detailed overview on research activities within the Active Learning Domain is presented in
\cite{settles2009active}. Note however that more recent works are not contained in this literature survey because it was last updated in 2009.
In general, Active Learning methods can be divided into three categories:
\begin{itemize}
    \item \textbf{Query Synthesis Active Learning}
    \item \textbf{Pool-based Active Learning}
    \item \textbf{Stream-based Active Learning}
\end{itemize}
The taxonomy mentioned above is composed in a data-centric way. While researchers generally agree on the taxonomy mentioned above,
it is important to note that the effectiveness of an active learning strategy also depends on the type of Machine Learning Model that is used
(e.g. (Convolutional) Neural Networks, Support Vector Machines, etc.) and the type of data that is used (e.g. images or text).
For example, the pool-based Active Learning strategy CoreSet \cite{sener2017active} was specifically designed for CNNs.

\subsection{Query Synthesis Active Learning}
\label{sec:QuerySynthesisActiveLearning}
Query Synthesis Active Learning, also known as Membership Query Synthesis Active Learning, was among the first active learning scenarios
that were proposed \cite{angluin1988queries}. The idea behind this approach is to synthesize data points from the input space, rather than
to sample real data points. Nowadays, this is done by training a generative model (e.g a Generative Adversarial Network) \cite{zhu2017generative}
which learns the distribution of unlabeled data. However, earlier works relied on statistical models such as Gaussian Mixture Models 
\cite{cohn1996active}. The generated queries are then labeled by the oracle and can be used to train the Machine Learning Model. Note that
Query Synthesis is not limited to classification tasks. For example, \cite{cohn1996active} proposed a method to predict the absolute coordinates
of a robot hand when given the joint angles of its mechanical arms as inputs. When the oracle is a human annotator, Query Synthesis Active Learning
researchers have encountered problems in labeling them. Because the generated queries do not show any class-discriminative features, human annotators
struggled to assign any class to them in the survey of \cite{baum1992query}.

% TODO: Hier ein Bild wie QS Active Learning funktioniert einfügen

\subsection{Pool-based Active Learning}
\label{sec:PoolBasedActiveLearning}
Pool-based Active Learning is the most widely studied and used type of Active Learning. The idea behind pool-based Active Learning approaches
is to iteratively select the most informative data points from the current unlabeled pool, query the oracle for their labels and add them to the
labeled pool. Next, the ML model is trained on the current labeled pool. This process is repeated until the query budget is exhausted. A more detailed
explanation can be found in \href{alg:PoolBasedActiveLearning}{Algorithm 1}. Pool-based Active Learning strategies share this structure among each other.
The main difference between them is the informative measure, i.e. the criterion with which they select which data points to label next. Within the 
Pool-based Active Learning category, there are two subcategories: \textbf{Uncertainty-based} sampling and \textbf{Diversity-based} sampling. 
Uncertainty-based sampling strategies select the data points that the model is most uncertain about. Diversity-based sampling strategies on the other
hand aim to select data points that best represent the data distribution in the unlabeled pool. All the active learning strategies that we will use
are pool-based Active Learning Strategies, stemming both from the Uncertainty-based and Diversity-based subcategories.

\begin{algorithm}
    \caption{Pool-based Active Learning} \label{alg:PoolBasedActiveLearning}
    \begin{algorithmic}[1]
        \Require Unlabeled data $U$,Labeled data $L = \emptyset$:, Oracle $O$, Model $M$, budget $B$
        \State Select $k$ data points from $U$ at random, obtain labels by querying $O$ and set $L=\{x_1,\ldots,x_1\}$
        and $U = U \setminus \{x_1,\ldots,x_1\}$ \Comment{Initialization}
        \State Train $M$ on initial labeled set $L$
        \While{label budget $B$ not exhausted}
            \State Select $l$ data points from $U$ predicted to be the most informative by the Active Learning strategy
            \State Set $L= L \cup \{x_i,\ldots,x_l\}$ and $U = U \setminus \{x_i,\ldots,x_l\}$
            \State Train $M$ on labeled set $L$
        \EndWhile
    \end{algorithmic}
\end{algorithm}

\subsection{Stream-based Active Learning}
\label{sec:StreamBasedActiveLearning}
Stream-based Active Learning is closer to Pool-based Active Learning than Query Synthesis Active Learning. It was first introduced by Cohn et al. 
\cite{cohn1994improving}. The main difference between Stream-based Active Learning and Pool-based Active Learning is that data arrives sequentially
instead of having a batch of data points at once. In the stream-based Active Learning scenario, the learner draws a data point from the data source
one at a time. For each data point, the learner can then decide to query the oracle for its label or to discard it. The decision whether to label a
data point can either be made on the basis of its informativeness \cite{dagan1995committee} or its location within the instance space \cite{cohn1994improving}.
In the latter case the learner would label the data point if it is located in a region of the instance space that the learner is not confident about.

\section{Continual Learning}
\label{sec:ContinualLearning}
Continual Learning is a subfield of Machine Learning that aims to solve the problem of catastrophic forgetting. To elaborate further on this problem,
it is important to remember that most machine learning services are deployed in an environment where constant changes occur. To adapt to this, it is
necessary for Machine Learning Models to learn continually, i.e. to learn new tasks without forgetting the knowledge they have acquired in the past.
This is a common case for humans. For example, if a child has once learned to walk, it does not forget how to walk when it learns to ride a bike.
In contrast to human behavior, the performance of Machine Learning Models on old tasks rapidly decreases when they are trained on new tasks. This phenomenon
is known as \enquote{Catastrophic Forgetting} and was already discovered in the early days of Machine Learning research \cite{mccloskey1989catastrophic}.
In more general terms, research in Continual Learning is looking not only to alleviate Catastrophic Forgetting, but to solve the 
\enquote{stabiliy / plasticity dilemma} \cite{carpenter1988art}. The stability / plasticity dilemma refers to the fact that Machine Learning Models
should ideally be stable enough to retain their performance on old tasks while simultaneously being plastic enough to adapt to new tasks. This is a dilemma
because both properties are designed even though they are in conflict with each other. In practice, Machine Learning Models are rather plastic than stable.
This is especially true for Deep Neural Networks, but generally for all Machine Learning Models that are trained by greedily updating their parameters using
gradient descent \cite{mundt2020wholistic}. \\
Continual Learning is mainly used in scenarios where new tasks arrive over time or where the data distribution changes over time. Despite being useful
in these classic scenarios, Continual Learning approaches can also be used in cases where the data cannot be stored for legal reasons or due to memory
constraints, i.e. when the normal batch learning approach with a large training set cannot be applied. 


\subsection{Continual Learning Scenarios}
\label{sec:ContinualLearningScenarios}
Continual Learning is a rapidly evolving research field and terminology and taxonomy are still being established. Among the most important factors to distinguish
is the way in which new tasks arrive. In the following, we will introduce the three typical Continual Learning Scenarios as presented in \cite{van2022three}. 

\subsubsection{Task-Incremental Continual Learning}
\label{sec:TaskIncrementalContinualLearning}
%TODO: Add a figure that shows the difference between the three scenarios. For Task-Incremental learning mention tuple with task id.
In the Task-Incremental setting, a model is informed about which task it will be trained on or which task the data whose label it is supposed to predict belongs to.
The task-information is supplied via a task identifier which is usually an integer. Because the model does not have to infer the task which it is supposed to predict
or learn, it is possible to have task-specific components in the model. For neural networks, this means that there is one output layer per task and for each task the
respective output layer is used while all other layers are shared between all tasks. These classifiers are also known as \enquote{multi-head} classifiers \cite{van2018generative}.
An example would be the following: The first task is to classify images of cats and dogs. The second task is to classify cows and sheep. The model would have one output layer 
used to classify cats and dogs and a second output layer to classify cows and sheep. When training for the first task, only the first output layer is used. When the second task
arrives the model is retrained with the second output layer.  The mapping that is learned is 
\begin{equation}
    f: X \times T \rightarrow Y
\end{equation}
Where $X = \mathbb{R}^{N x N x C}$ is the input space (i.e. all possible input images of size $N x N$ with $C$ channels), $T = \{1,2,\ldots\}$ is the task space (i.e. all possible
tasks that the model can be trained on) and $Y = \mathbb{Z}_{k}$ is the label space with $k$ being the number of possible classes.
% Figure with decision tree for the continual learning scenarios like in https://www.nature.com/articles/s42256-022-00568-3
\subsubsection{Domain-Incremental Continual Learning}
\label{sec:DomainIncrementalContinualLearning}
In the Domain-Incremental setting, task identities are not given during evaluation. Since the underlying task does not change, this is also not necessary. While the structure of the task
stays the same it is rather the distribution of the data that changes. An example of Domain-Incremental Continual Learning would be the classification of digits between 0 and 9 (such as
those in the MNIST dataset) where the digits for one subtask are green and blue for the other. The mapping that is learned is 
\begin{equation}
    f: X \rightarrow Y
\end{equation}
Where $X = \mathbb{R}^{N x N x C}$ is the input space (i.e. all possible input images of size $N x N$ with $C$ channels), and $Y = \mathbb{Z}_{k}$ is the label space with $k$ being the
number of possible classes.

\subsubsection{Class-Incremental Continual Learning}
\label{sec:ClassIncrementalContinualLearning}
Class-Incremental Continual Learning is the most challenging scenario. Like in the Domain-Incremental setting, task identities are not provided at evaluation time, however, this time they
need to be inferred by the model. In this scenario, a classifier would be incrementally exposed to multiple tasks which contain different classes. An example would be again the classification
of digits between 0 and 9, where this time each task contains a disjoint subset of digits. The classifier would be trained on the first task, containing the digits 0 and 1, the second task,
containing the digits 2 and 3 and so on. During evaluation, the classifier would not only have to classify the digits correctly, but also to infer which task they belong to. The mapping that
is learned is 
\begin{equation}
    f: X \rightarrow Y \times T
\end{equation}
Where $X = \mathbb{R}^{N x N x C}$ is the input space (i.e. all possible input images of size $N x N$ with $C$ channels), $T = \{1,2,\ldots\}$ is the task space (i.e. all possible
tasks that the model can be trained on) and $Y = \mathbb{Z}_{k}$ is the label space with $k$ being the number of possible classes.

\subsection{Continual Learning Approaches}
\label{sec:ContinualLearningApproaches}
The Continual Learning approaches which have been proposed so far can be grouped into three categories according to, \cite{parisi2019continual} \cite{mundt2020wholistic} and \cite{zenke2017continual}.
Parisi et al. \cite{parisi2019continual} propose to group Continual Learning approaches into \textbf{regularization}, \textbf{rehearsal} and \textbf{architectural} approaches whereas Zenke et al.
\cite{zenke2017continual} group Continual Learning approaches into \textbf{architectural},\textbf{functional} and \textbf{structural} approaches. In the following, we will stick to the
categorization proposed by Parisi et al. because it is broader and fully encompasses the categorization by Zenke et al. Furthermore, Parisi et al.'s classification has been adopted by
recent continual learning reviews \cite{mundt2020wholistic}.

\subsubsection{Regularization Approaches}
\label{sec:RegularizationApproaches}
Regularization-based approaches to Continual Learning aim to prevent the forgetting of previous tasks by adding a regularization
term to the model's loss function. The regularization term is used as a proxy for how much the performance of the model on previous
tasks will decrease, i.e. a high regularization term indicates that the model will perform poorly on the old tasks with the current
weights and a low regularization term indicates that the model has not lost much knowledge of the old tasks. The way the
regularization term is computed can further be divided into two groups. There are \textbf{structural} approaches which regularize
based on weight changes to the model and there are \textbf{functional} approaches which regularize based on the output of the model.
Notable examples of structural approaches include Elastic Weight Consolidation (EWC) \cite{kirkpatrick2017overcoming},Memory Aware
Synapses (MAS) \cite{aljundi2018memory}, Incremental Moment Matching (IMM) \cite{lee2017overcoming} as well as Asymmetric Loss
Approximation by Single-Side Overestimation (ALASSO) \cite{park2019continual} which is an extension of Synaptic Intelligence (SI)
\cite{zenke2017continual}. All the structural regularization approaches will be covered in more detail in the section on
\href{sec:Related_work:Continual_Learning:Experiments}{the continual learning approaches used in the experiments}. \\
Functional regularization approaches are inspired by knowledge distillation \cite{hinton2015distilling}. They add a distillation
loss to the objective function which is computed based on the prediction of a data sample stored for future use. These data samples
are called soft targets. Li et al. \cite{li2017learning} compute the distillation loss by using the output of the newly arrived task
given by the model trained on the old tasks. The distillation loss they introduce aims to retain the prediction of the old model on
the new task even if the prediction itself may be inaccurate. The approach of Rannen et al. \cite{rannen2017encoder}, called Encoder
Based Lifelong Learning (EBLL) is based on the approach of Li et al., however in EBLL the distillation loss is computed based on
autoencoder reconstructions of old tasks.

\subsubsection{Rehearsal Approaches}
\label{sec:RehearsalApproaches}
Rehearsal approaches to Continual Learning aim to prevent catastrophic forgetting by fitting a model's parameters to the distribution
of an incoming task and all previous tasks simultaneously. Within Rehearsal approaches it is important to remember the trade-off between
model performance and computational cost. In general, the more data from previous tasks is used to train the model, the better the accuracy is. 
However, in order to train on more data, the data has to be stored and fed into the training process which is costly in terms of memory and This can either be done by replaying stored data from previous tasks or by
retraining on generated data drawn from the distribution of previous tasks. Continual Learning Methods that rely on the former idea are
categorized as \textbf{exemplar rehearsal} approaches while those that rely on the latter idea are categorized as \textbf{generative replay}
approaches. \\
Exemplar Rehearsal Techniques store data from previous tasks in a so-called \enquote{Replay Buffer} which data is sampled



\subsubsection{Architectural Approaches}
\label{sec:ArchitecturalApproaches}

\section{Model Stealing}
\label{sec:ModelStealing}
% TODO: Add figure with model stealing process
% Model Stealing notes: 
% Taxonomy provided by Jagielski et al. based on attacker's goal 
% Four types of attacks presented by He et al.: model extraction, model inversion, model poisoning and adversarial attack.
% Model extraction is a subfield of adversarial machine learning.
% Look at section 4.3 of Oliynyk et al. paper for taxonomy of adversarial ml.
% Mention some notation as in Olinyk et al. paper in section 5.1.
% Metrics to measure attack in 5.4 of Olinyk et al. paper.
% Have a look at Fig3 of Olinyk et al. paper and use it to tell which attacks we use (e.g. NNPD, Computer Vision, etc.)
% Based on Olinyk et al., our approach falls into the consistency category of section 6.1
% Model Stealing defenses are either used for attack detection or attack prevention.
With the advent of Machine Learning as a Service (MLaaS), an increasing amount of machine learning models are being exposed to the public via 
prediction APIs. The idea of these prediction APIs is that a user can request a prediction from a model by sending a request containing an unlabeled data
point to the API. The response to the request is then the prediction of the model for the data point. Prediction APIs are monetized by charging the user
on a pay-per-query principle. That means that each request has a fixed price for each query which is usually subtracted from his or her account balance on
a monthly basis. Providing public access to a machine learning model is a win-win situation because the model developer is compensated for his or her efforts
on gathering and labeling appropriate training data, choosing a proper model architecture, training the model and fine-tuning its hyperparameters while the user
of the prediction API can benefit from the model's predictions without having to train the model himself or herself. \\
Since the developed model is the intellectual property of the model developer, it is important that only the model's predictions are exposed to the public and
not the model (i.e. its architecture and the model weights)itself. However, in the last years numerous research papers have been published which demonstrate how
several features of a machine learning model can be extracted, i.e. its functionality \cite{tramer2016stealing}, its architecture \cite{oh2019towards} and its
training data \cite{shokri2017membership}. This newly created field of research is called \textbf{Model Stealing} or \textbf{Model Extraction}, and it is a subfield
of \textbf{Adversarial Machine Learning} according to Oliynyk et al. \cite{oliynyk2022know}. Because Model Extraction attacks are so effective, further research works
concerning model stealing defense mechanisms have been published. In the following we will use the taxonomy provided by Oliynyk et al.
\cite{oliynyk2022know} as a basis for our categorization of Model Stealing attacks and defenses. \\

\subsection{Terminology}
\label{sec:ModelStealing:Terminology}
Since Model Stealing is a rather new field it comes with numerous terms that are not commonly used in other fields of machine learning and others that are used 
synonymously. In order to avoid confusion, we will define the terms we use in this section in the following. \\
\textbf{Model Stealing Attack}: The process of maliciously querying a machine learning model in order to extract some or all of the model's features, such as its
architecture, its weights or its training data. Model Stealing Attacks are also known as Model Extraction Attacks or Model Inference Attacks among others. \\
\textbf{Target Model}: The model that is queried via the prediction API and whose features the attacker aims to extract. The target model has also been referred to 
as Oracle, Victim Model or Secret Model. \\



\subsection{Model Stealing Attacks}
\label{sec:ModelStealing:Attacks}

\subsection{Model Stealing Defenses}
\label{sec:ModelStealing:Defenses}